

\chapter{Prefácio}\label{prefuxe1cio}

Se uma pessoa em um centro urbano qualquer ficasse isolada nos últimos
dez anos e fosse reinserida de repente nos dias atuais, ficaria chocada
com a ubiquidade dos dispositivos de acesso a internet e principalmente
com a frequência de seu uso. Para além da disseminação dos dispositivos,
a web sofreu uma gradual, porém profunda, mudança nesses anos. O insight
vem de um proeminente blogueiro iraniano que ficou preso entre 2008 e
2014 por conta de seu ciberativismo. Ao ser liberado a dinâmica do
ambiente virtual que Houssein Derakhshan reencontrou era muito diferente
daquela que ele se habituara no começo da primeira década do século. Se
antes de sua prisão a web era principalmente uma rede de documentos
ligados por hiperlinks em que a experiência do usuário era construída
dinamicamente em sua navegação, hoje ela é melhor descrita como um
conjunto de canais, controlados por um pequeno grupo de grandes
empresas, em que os usuários consomem e produzem fluxos constantes de
dados {[}Derakhshan 2015{]}.

Uma vez notada essa mudança, possivelmente não tivesse sido tão
surpreendente as denúncias de Edward Snowden em 2013. A concentração de
tanta informação em tão poucas bases de dados cria o que os
especialistas em segurança da informação chamam de pontos únicos de
falha que, para piorar a situação, estão muito concentrados em uma única
região: o vale do Silício.

Talvez por isso muitos desses técnicos não receberam com grande surpresa
as denúncias do empregado terceirizado que prestava serviços à NSA de
que a agência de segurança estadunidenses vigia bilhões de pessoas no
mundo todo interceptando suas comunicações pessoais e principalmente
mapeando suas redes de contatos. Politicamente a notícia teve impacto
ainda maior do que os vazamentos do Wikileaks alguns anos antes. No
Brasil, aonde inclusive a presidente teve suas comunicações
interceptadas, as declarações criaram uma animosidade temporária com a
maior potência do mundo, aceleraram a aprovação do Marco Civil da
Internet e fomentaram a formulação de projetos de leis de proteção de
dados pessoais. De uma perspectiva menos institucional, encontros sobre
segurança da informação, como as criptorraves, organizadas por ativistas
e acadêmicos, proliferaram e têm atraído um público maior a cada ano.

Este livro pretende oferecer um primeiro contato a esses temas complexos
a quaisquer interessados. Ele contém seis textos sobre três assuntos
distintos, mas intrinsecamente relacionados: privacidade, proteção de
dados pessoais e vigilância. Quatro são traduções de textos de autores
consagrados na área e dois são textos originais de autores nacionais.
Julgamos que essa coletânea possa vir a ser útil em cursos de graduação
e pós-graduação.

O livro começa com um texto escrito por seus organizadores com reflexões
provocativas. Os autores argumentam que a construção da infraestrutura
de vigilância em massa foi possível graças a hegemonização do modelo de
negócios baseado na publicidade direcionada que se estabeleceu em parte
como resposta às reivindicações dos movimentos de cultura-livre. Apesar
de nesse modelo os usuários produzirem informações que são usadas para
melhor classificá-los, o papel desses como trabalhadores voluntários é
questionado. Para os autores, a privacidade deve ser compreendida não
apenas como proteção à intimidade, mas como um direito civil
fundamental. A garantia desse direito, porém, requer não apenas um marco
regulatório, mas principalmente mobilização e conscientização que
incentive o desenvolvimento e a adoção de padrões mais elevados de
segurança.

Privacidade é um conceito polissêmico de difícil captura. Tentativas
modernas de conceitualizá-lo inevitavelmente caem em uma lista de
diferentes significados com relações pouco claras. O segundo texto do
livro é uma das primeiras dessas tentativas e foi escolhido
principalmente por sua influência e importância histórica. O texto de
1890 foi escrito por dois colegas da faculdade de direito de Boston:
Samuel Warren e Louis Brandeis. O ano em que foi escrito coincide
curiosamente com o ano em que máquinas Hollerith, precursoras dos
primeiros computadores, eram usadas pela primeira vez para processar o
censo nos Estados Unidos. A preocupação dos advogados, porém, era com o
desenvolvimento de outra tecnologia: a máquina fotogŕafica. Para os
autores, a fotografia instantânea e o jornalismo invadiram ``o recinto
sagrado da vida privada''' ferindo o direito individual de ``ser deixado
em paz'' e por vezes submetendo as pessoas a uma forma de ``sofrimento
mental''. Assim, a dupla argumenta que o Direito Comum no EUA deveria se
adaptar a essas mudanças e alargar seu escopo, como tantas outras vezes
antes, para incluir proteção à privacidade.

Na Europa a preocupação com a regulamentação da proteção de dados
pessoais toma força nos anos 1970 quando o desenvolvimento computacional
combinado com o interesse do estado em melhor aplicar políticas públicas
incentivam a produção de grandes bases de dados pessoais. Vários países
europeus formularam leis específica sobre o tema. Viktor
Meyer-Schenberger argumenta que a grande similaridade entre essas leis
torna uma abordagem geracional mais útil do que uma comparação entre
elas. Assim o autor apresenta quatro gerações de leis formuladas entre
1970 e o final dos anos 1990. Cada geração apresenta novidades em
relação a anterior para melhor se adaptar ao desenvolvimento tecnológico
ou incorporar os debates e os limites aprendidos na geração anterior.
Assim, a legislação focada na regulamentação técnica de grandes bases de
dados centralizadas, evoluiu para uma legislação voltada para as
liberdades civis centrada no consentimento informado - mais adaptada ao
ambiente distribuído dos mini-computadores -, para uma legislação focada
na participação do titular dos dados e finalmente em uma legislação mais
paternalista com autoridades com papel decisório para deliberar contra
violações.

Hellen Nissembaum é uma das principais críticas do paradigma regulatório
centrado no consentimento informado. Em seu influente trabalho, a autora
apresenta uma teoria segundo a qual a privacidade deve ser avaliada
quanto ao respeito à integridade contextual dos fluxos de informação.
Assim, uma violação ocorre não quando um dado privado é tornado público,
mas quando ele é tirado de seu contexto. Para a autora, a internet teria
mediado o rompimento da integridade de fluxos informacionais em uma
escala e variedade sem precedentes, mas nem por isso nossa abordagem
deveria ser diferente nesse meio. Grande parte dos serviços na web são
análogos a serviços offline e portanto, segundo a autora, devem seguir a
mesma abordagem regulatória. Além do texto selecionado nesta coletânea,
a autora possui uma vasta quantidade de outros textos recentes sobre o
tema como \emph{Respecting Context to Protect Privacy: Why Meaning
Matters} de 2015, em que a autora articula seu conceito de integridade
contextual a noção de ``contexto'' como apresentada na lei de
privacidade endossada por Obama em 2012.

Vigilância é um tema correlato ao da privacidade e da proteção de dados
pessoais. Foucalt descreve como no século XVIII a vigilância aos poucos
substitui o suplício como forma de controle social. As instituições
modernas, escolas, exército, hospitais etc., vigiam e disciplinam os
corpos e, assim, produzem indivíduos obedientes. Para Deleuze, a
descrição da sociedade disciplinar é adequada para os séculos XVIII e
XIX, mas depois da segunda guerra esse modelo começou a ser
gradativamente substituído pelo que ele chamou de ``sociedade do
controle''. Na sociedade do controle os indivíduos não passam mais de
uma instituição disciplinar para outra cada qual nos modelando, mas são
intermitentemente modulados por um processo contínuo. Esses dois modelos
de sociedade nos ajudam a perceber diferentes formas como a vigilância
reproduz estruturas de poder. Como bem nos lembra o autor do quinto
texto deste livro, há um recorte de classe sobre quem é alvo da
vigilância disciplinar e quem é alvo da vigilância do controle.

David Lyon é provavelmente o principal autor contemporâneo a tratar do
tema da vigilância. O texto ``Vigilância como triagem social'' é a
introdução de um livro homônimo de artigos sobre vigilância
contemporânea. Para o autor, uma tendência-chave da vigilância, definida
como ``atenção sistemática a detalhes pessoais, com finalidade de
gerenciar ou influenciar pessoas e grupos'', é a utilização de bases de
dados pesquisáveis para diversos fins como verificar e monitorar o
comportamento, influenciar pessoas e prever riscos que pode ser
observada nas práticas de publicidade direcionada e policiamento, na
proliferação dos circuitos fechados de TV (CCTV) e na crescente gama de
dispositivos de localização. Os alvos da vigilância são classificados em
um processo denominado pelo autor de "triagem social". Essa triagem é
cada vez mais mediada por códigos que "congelam" retóricas, discursos
políticos e preconceitos e cada vez mais ubíqua e móvel.

O livro encerra com um ensaio escrito por Silvia Viana para uma
conferência de 2015. A autora se indaga sobre a estrutura que configura
o desenvolvimento técnico que confere forma à vigilância. A autora
critica a posição de que o desenvolvimento tecnológico é neutro e
posteriormente instrumentalizado. A imagem de que existe uma disputa de
poder e contrapoder na rede ignora as disparidades políticas e
econômicas que dão grande vantagem às empresas e ao Estado capturando
informações a posteriori ou influenciando ideologicamente os fluxos de
informação. Para a autora, as tecnologias da informação, como produção
capitalista, são a realização de duas dimensões: controle e
participação. Assim, é falsa a disputa entre esses supostos pólos que na
verdade obedecem ambos ao imperativo da produção. Sílvia argumenta que
produzimos dados para os sites de redes sociais em um ritual que
repetimos pelo medo do desaparecimento simbólico, ritual que mimetiza a
lógica da concorrência horizontal no mercado de trabalho.

\chapter{Quatro reflexões sobre a privacidade e a
vigilância}\label{quatro-reflexuxf5es-sobre-a-privacidade-e-a-vigiluxe2ncia}

\section{A pirataria e o movimento de cultura livre foram co-responsáveis
pela montagem da infraestrutura da vigilância eletrônica em massa}

Quando a internet foi privatizada e aberta para acesso público, em
meados dos anos 1990, ela já estava povoada por serviços universitários
gratuitos e participativos, de nascentes páginas web a listas de e-mail
e grupos de notícias da Usenet. As empresas de comunicação -- que
tentaram explorar economicamente esse novo e emergente mercado --
buscaram impor a esse novo meio a dinâmica de uso a que estavam
acostumadas nos meios de comunicação de massa, como a TV e as revistas
semanais.

Foi a era dos grandes portais de internet com produção de conteúdo
concentrada, elaborada por profissionais de mídia reunidos numa redação
jornalística. Neste modelo, a bidirecionalidade da rede era aproveitada
apenas para que os usuários fizessem uso de serviços como os de e-mail
ou para que clicassem em \emph{hyperlinks} como se vira uma página de
revista ou se troca de canal de TV. O modelo que tinham em mente para
financiar esses portais de produção centralizada era o da venda de
assinaturas, como faziam os jornais, as revistas e a TV a cabo. Esse
modelo foi desafiado durante os primeiros anos da internet privada de
duas maneiras: enquanto modelo de produção concentrado e em sua
viabilidade econômica como modelo de negócios.

No que diz respeito ao modelo concentrado, vários experimentos na
segunda metade dos anos 1990 buscaram transpor para a Web a cultura de
comunicação participativa que vinha da internet universitária, na qual
cientistas discutiam sobre assuntos acadêmicos em listas de e-mail ou se
entretinham participando em grupos de notícia de ficção científica na
Usenet. Alguns desses experimentos como a Wikipedia ou a rede Indymedia
terminaram virando modelos para a comunicação que veríamos anos depois
na chamada Web 2.0 e nas redes sociais.

Mas, os portais foram desafiados também em sua dimensão econômica.
Quando foram lançados, eles tiveram que competir com todo o ecossistema
universitário gratuito que já estava implantado desde antes da internet
se abrir ao público mais amplo. Próximo dele, havia também o movimento
de software livre que se transmutou no final dos anos 1990 e no início
dos anos 2000 num mais amplo movimento de cultura livre com produtores
de notícias, textos literários, músicas e vídeos produzindo livremente
sem exigir autorização ou cobrança de royalties. E, além deles, um
massivo movimento de desobediência civil digital passou a distribuir
livremente todo tipo de conteúdo das indústrias culturais sem
autorização ou pagamento aos titulares. Juntos, a internet
universitária, o movimento de cultura livre, mas, sobretudo a pirataria
organizada colocaram-se como competidores extra-mercantis que tornaram
praticamente impossível para os operadores comerciais venderem com êxito
qualquer tipo de produto ou serviço na internet\footnote{Sobre
  a competição entre produtos mercantis e extra-mercantis: Carlotto, M.
  C; Ortellado, P. Activist-driven innovation: uma história
  interpretativa do software livre. \emph{Revista Brasileira de Ciências
  Sociais}, v. 26, n. 76, 2011. p. 77-102.}. Como Chris Anderson havia
argumentado, o grátis era um preço mágico, praticamente impossível de
ser derrotado no mercado\footnote{Anderson. \emph{Free:
  the future of a radical price}. Nova Iorque: Hyperion, 2009.}.

Sob pressão da competição de iniciativas não mercantis, de um lado e das
plataformas participativas, de outro, a indústria foi empurrada para um
modelo baseado na exploração da publicidade, como já fazia a TV aberta e
passou a experimentar, pouco a pouco, serviços de comunicação
participativos que ficaram conhecidos, num primeiro momento, como a Web
2.0\footnote{O`Reilly, T. What Is Web 2.0. 2005.
  Disponível em
  \textless{}http://www.oreilly.com/pub/a/web2/archive/what-is-web-20.html\textgreater{}}.
Um importante manual da economia da informação\footnote{Shapiro,
  C.; Varian, H. \emph{Information Rules: A Strategic Guide to the
  Network Economy}. Boston: Harvard Business Review Press, 1999.} já
sugeria, no final dos anos 1990, que o caminho da indústria era
aproveitar o grande manancial de informações produzidas pelos usuários
que permitiriam aprimorar o modelo da publicidade dirigida. Ao invés de
fazer publicidade de massa para todos os milhares de assinantes de um
jornal, os meios digitais permitiam fazer uma publicidade bem mais
dirigida, especialmente se, com a adoção dos serviços participativos,
pudesse recolher muitas informações pessoais dos usuários.

Nos anos 2000, um tipo novo e particular de ativismo se difundiu: o
ativismo digital. Enquanto as empresas da indústria cultural travavam as
guerras do direito autoral, perseguindo piratas digitais pelo mundo e
endurecendo o \emph{enforcement} das leis de propriedade intelectual por
meio de mecanismos internacionais como os tratados da OMPI e os
Relatórios Especiais 301\footnote{Para um balanço
  histórico rico e detalhado da formação destes instrumentos
  internacionais, veja: Susan Sell. \emph{Private Power, Public Law: The
  Globalization of Intellectual Property Rights.} Cambridge: Cambridge
  University Press, 2003 e também Peter Drahos. \emph{Information
  Feudalism: Who Owns the Knowledge Economy}. Nova Iorque: New Press,
  2003.}, jovens de todo mundo se alistavam na luta por uma cultura
livre. Projetos sofisticados de compartilhamento se difundiram
anunciando uma nova era de liberdade e abundância digital. Mas quanto
mais a gratuidade da pirataria e da cultura livre pressionava
economicamente a indústria, sufocando os modelos pagos, mais rapidamente
o modelo baseado em exploração da publicidade dirigida a partir dos
dados pessoais aparecia como a tábua de salvação. Involuntariamente, o
ativismo da cultura livre terminou trocando a gratuidade de serviços
como os da Google e do Facebook pela mercantilização da nossa vida
privada: o conteúdo das nossas conversas, nossas buscas na Web, nosso
deslocamento na cidade. E foi sobre esse valiosíssimo arsenal de
conteúdos privados que a NSA montou seu monstruoso sistema de vigilância
em massa. Terá valido a pena?

*

\section{Você é mercadoria e não trabalhador no modelo de negócio das
empresas de internet}

Quando o Youtube foi comprado pela Google em 2006 por mais de um bilhão
e meio de dólares, apenas pouco mais de um ano após sua fundação, muitos
se perguntaram como uma empresa tão nova e principalmente tão pequena -
na época ela possuía menos de uma centena de funcionários - poderia
valer tanto. A explicação para essa pergunta oferecida por uma série
autores e que acabou formando o senso-comum é a de que o valor da
empresa era produto do trabalho não remunerado dos seus milhões de
usuários\footnote{Dantas, Marcos. ``Mais valia 2.0:
  produção e apropriação de valor nas redes do capital''
  \emph{https://dialnet.unirioja.es/servlet/revista?codigo=20212}(\emph{Eptic
    online}: revista electronica internacional de economia política da
  informaçao, da comuniçao e da cultura), ISSN-e 1518-2487,
  \emph{https://dialnet.unirioja.es/ejemplar/392385}{{Vol. 16, Nº. 2,
  2014}}, págs. 85-108}.

Vinte e cinco anos antes, Dallas Smythe inaugurou essa polêmica linha de
argumentação em um outro contexto. Estudando os grande meios de mídia
tradicional no capitalismo monopolista, Smythe defende que o produto das
mídias de massa é o que ele chama de \emph{força da audiência}
(\emph{audience power}) em uma analogia à força de trabalho. A força da
audiência seria produzida pelos trabalhadores em seu tempo de lazer ao
assistirem programas de televisão ou lerem revistas e jornais
financiados por anúncios comerciais. O acesso à audiência seria vendido
pelos donos dos veículos de mídia aos industriais em busca de
compradores para seus produtos\footnote{Smythe, Dallas
  W. "On the audience commodity and its work." \emph{Media and cultural
  studies: Keyworks} (1981): 230-56.}.

No contexto atual, além de produzir a audiência, os usuários de
plataformas de redes sociais, como o mencionado Youtube e o Facebook por
exemplo, participam ativamente da produção do próprio conteúdo criativo
que atrai outros usuários para as plataformas - para produzir mais
audiência e conteúdo. Ademais, em muitos casos esses conteúdos refletem
as preferências dos usuários permitindo que sejam traçados perfis de
consumo. Assim, aquilo que se produz no atual modelo de negócios de
grande parte das empresas na internet seria um tipo mais valioso de
audiência, uma audiência segmentada, organizada e engajada.

Duas das críticas a essa analogia entre a comunicação por meio das redes
sociais e o trabalho não remunerado é a falta de consciência do
trabalhador sobre essa relação e a falta de controle de sua atividade
pela empresa\footnote{Caraway, Brett. "Audience labor
  in the new media environment: A Marxian revisiting of the audience
  commodity." \emph{Media, Culture \& Society} 33.5 (2011): 693-708.}.
Se pretendemos insistir na analogia, precisaríamos em primeiro lugar
defender que os termos de serviço das plataformas de redes sociais
servem como uma espécie de contrato em que os usuários consentem em
trocar o produto de seu trabalho pelo acesso que a plataforma oferece ao
conteúdo produzido por outros usuários. Em segundo lugar, é preciso
ampliar a noção de controle exercido sobre o trabalhador. Alguns autores
argumentam que os funcionários das empresas de internet moldam o
comportamento dos usuários ao estabelecerem protocolos que restringem
sua gramática de comunicação de forma a maximizar as formas aonde seria
mais fácil extrair lucro. Nesse sentido, os funcionários dessas empresas
seriam como gerentes buscando maximizar a extração do lucro da produção
dos usuários\footnote{Beverungen, Armin, Steffen Böhm,
  and Chris Land. "Free labour, social media, management: Challenging
  Marxist organization studies." (2015): 473-489.}.

O problema de toda essa linha de argumentação, porém, é que, ao
interpretar a interação dos usuários nas redes sociais como trabalho
semiótico produtivo, admitimos que todo o tempo de reprodução dos
trabalhadores, todo seu tempo livre, está subjugado ao tempo de
produção. Em outras palavras, nessa visão de mundo não há qualquer
subjetividade possível, apenas alienação. Se nos atermos, porém, com
mais cuidado ao problema, veremos que no fundo não são as nossas
interações que são mercantilizadas, mas nossa atenção. As empresas que
controlam as plataformas de comunicação estão interessadas apenas muito
indiretamente no conteúdo do que produzimos ao nos comunicarmos por meio
dessas ferramentas. Por isso o pouco controle sobre nossa atividade. Por
outro lado, há uma acirrada disputa sobre nossa atenção, especialmente
durante nosso tempo de lazer.

Ou seja, não é exatamente nosso trabalho que é transformado em
mercadoria nessas plataformas, mas nossa atenção. Nesse sentido é mais
fácil compreender o controle fino dos protocolos e algoritmos que
decidem quais informações terão mais ou menos destaque num ajuste
delicado entre aquilo que nos mantém mais tempo usando a plataforma -
como as fotos de um primo distantes recém nascido ou um comentário
sensível de uma amigo de infância -, aquilo que se busca vender - desde
um novo modelo de tênis até um candidato à presidência - ou de
preferência ambos ao mesmo tempo.

*

\section{A literatura sobre privacidade enfatiza a proteção da intimidade
em detrimento da proteção dos direitos políticos}

Os problemas em torno da privacidade são muito antigos e os usos do
conceito parecem ter muitos sentidos. Num famoso artigo sobre a
polissemia do conceito de privacidade, Daniel Solove propôs que
olhássemos para a diversidade de seus significados e usos por meio do
conceito wittgensteiniano de "semelhança de família", na qual o que os
une são traços em comum que podem unir apenas alguns, sem que haja
necessariamente traços que sejam comuns a todos,, como a semelhança que
se nota numa foto em que aparecem muitos primos\footnote{Solove,
  D. Conceptualizing privacy. \emph{California Law Review}. Vol. 90, n.
  4, julho de 2002.}. Por isso, seria inútil buscar um conceito simples
e comum a todos os usos que explicasse o que "verdadeiramente" ou
"essencialmente" é a privacidade.

Entre esses muitos sentidos que não podem ser unificados, dois deles
parecem particularmente divergentes: aquele que entende a privacidade
como proteção à vida íntima e aquele que entende privacidade como um
fundamento dos direitos políticos.

Quando Warren e Brandeis, no clássico artigo que inaugurou o discurso
jurídico americano sobre o direito à privacidade, definiram-na como o
"direito de ser deixado em paz", deram origem a uma série de reflexões
derivadas que lidam com o problema da proteção da vida íntima e do
controle sobre a imagem própria. Essas reflexões dominam toda uma
corrente no direito e na sociologia que enfrentam questões reais sobre a
exposição indevida da intimidade e o controle do desenvolvimento da
personalidade. Embora os problemas sejam reais e relevantes, essa
corrente tem sobrepujado e colocado em segundo plano questões
pertinentes aos direitos políticos e o papel da privacidade como um dos
seus fundamentos.

Desde os conflitos que opuseram as classes emergentes às aspirações
absolutistas da monarquia na Inglaterra do século XVII, a privacidade
concretamente entendida como o direito de não ter a sua residência ou a
sua correspondência violada é um fundamento da liberdade e dos direitos
políticos. Proteger a privacidade é não permitir que o Estado monitore e
assim se antecipe à organização daqueles que o opõe a partir da
sociedade civil. Protegê-la é, portanto, um fundamento do direito de se
organizar politicamente. Por isso, a privacidade dos cidadãos deveria
ser preservada contra a interferência de qualquer poder político, como
já havia lembrado em 1776, o parlamentar whig William Pitt, numa famosa
passagem de um discurso ao parlamento: "O mais pobre dos homens," disse
ele, "desafia, em sua cabana, todas as forças da Coroa. Ela pode ser
frágil, seu teto pode tremer, o vento pode nela soprar, a tempestade
pode entrar, a chuva pode entrar, mas o rei da Inglaterra não pode
entrar."\footnote{Citado em Albert M. Bendich. Privacy,
  Poverty, and the Constitution. \emph{Caiforrnia Law Review}, v. 54, n.
  2, 1966. }

A ideia de que os direitos se estruturam em camadas que fundamentam
outras camadas foi classicamente exposto no influente artigo de T. H.
Marshall, "Cidadania e classes sociais"\footnote{\emph{Citizenship
  and Social Class}. Londres: Pluto, 1987.}. Nele, Marshall defende que
a experiência histórica da Inglaterra desenvolveu o sistema de direitos
de cidadania como gerações mais ou menos bem distribuídas entre os
séculos. Assim, no século XVIII se consolidaram os direitos civis, no
XIX, os direitos políticos e no XX, os direitos sociais. Essa sucessão,
no entanto, não foi apenas uma sucessão \emph{cronológica}, mas também
uma sucessão \emph{lógica} na qual cada geração de direitos deu
sustentação à geração de direitos seguinte. Foi fazendo uso dos direitos
civis, como a liberdade de reunião e a liberdade de imprensa que os
trabalhadores se organizaram e conquistaram os direitos políticos; e foi
fazendo uso de seus direitos políticos que conquistaram os direitos
sociais.

Quando olhamos as coisas assim, direitos civis como o direito à
privacidade não são apenas um direito a mais: são o fundamento de todo o
sistema de direitos. Por isso, quando descobrimos, como aconteceu após
os vazamentos de Edward Snowden, que sua violação é sistemática e
corriqueira, é todo o conjunto dos direitos de cidadania que está em
jogo.

*

\section{Não há saída regulatória para o problema da vigilância em massa}

Para muitos técnicos e pesquisadores em segurança da informação, as
revelações de Edward Snowden em 2013 sobre o esquema de vigilância
global operado pela Agência de Segurnça Nacional dos EUA, apenas
confirmaram uma desconfiança de longa data. Olhando do ponto
privilegiado em que nos encontramos agora, parece evidente que tal
desconfiança tinha embasamento. O motivo da desconfiança é que a
hegemonia na internet do modelo de negócios baseado na extração de dados
pessoais para construção de perfis de usuários, criou o que esses
técnicos chamam de \emph{pontos únicos de falha}. Por maior esforço que
as grandes empresas da internet façam para evitar vazamentos e ataques,
suas bases de dados se tornaram grandes e valiosas demais para uma vasta
gama de atores com diferentes poderes computacional e de influência.

Do ponto de vista regulatório, desde a década de 1970 diversos países
têm aprovado leis específicas de proteção de dados pessoais. Seguindo a
análise geracional proposta por Meyer-Schönberger, desde meados dos anos
80 no centro deste debate está o conceito de \emph{autodeterminação
informacional}. As legislações em geral buscam essencialmente garantir
que o titular dos dados pessoais tenha ciência de como seus dados serão
usados e algum grau de controle sobre eles. Em outras palavras, as leis
regulamentam os contratos de troca entre prestadores de serviços e
titulares de dados pessoais. Nos anos noventa, o debate se voltou para
necessidade em se diminuir a granularidade desse tipo de troca - para
evitar situações onde ou o usuário aceita entregar seus dados para
virtualmente qualquer fim ou não tem acesso ao serviço -, e em aumentar
a interação dos usuários no processo de consentimento. Em outra frente,
há uma importante disputa sobre a importância de uma autoridade garante
forte e independente, capaz de fiscalizar os corretores de dados
(\emph{data-brokers}) e proteger os cidadãos de possíveis abusos. De
qualquer forma, o \emph{consentimento informado} continua sendo a pedra
de toque do debate legislativo.

Não é possível exagerar na importância de tal regulamentação. Leis que
impeçam o uso não consentido de dados pessoais são imprescindíveis para
evitar uma série de abusos. Porém, a saída regulatória não toca, e nem
poderia tocar, no ponto principal: o modelo de negócios que permitiu a
construção da infraestrutura de vigilância em massa. Enquanto o modelo
de negócios hegemônico na internet for a propaganda direcionada, baseada
no processamento massivo de dados pessoais, a disputa pela maior
concentração desses dados estimulará a existência dos ditos pontos
únicos de falha. Em outras palavras, a regulação pode coibir usos
abusivos, mas as grandes bases de dados continuarão existindo e
continuarão a ser exploradas pelos serviços de inteligência - que operam
em muitos casos numa zona cinzenta entre legalidade e ilegalidade - para
coibir dissidências.

As denúncias de 2013 sobre o esquema de vigilância global, além de
estimularem o debate sobre a legislação de proteção de dados pessoais,
reacenderam preocupações técnicas em relação à infraestrutura de
segurança na comunicação online. No caso da comunicação interpessoal, em
particular, muitos começaram a questĩonar a solução amplamente adotada
de se criptografar o tráfego das mensagens mantendo as chaves em posse
dos operadores dos serviços. A preocupação por mais segurança, aqueceu
um mercado de aplicações de comunicação em que a chave da comunicação
criptografada é compartilhada apenas entre remetente e destinatário e
não com os intermediários - o que no jargão técnico é chamado de
\emph{criptografia ponta a ponta}. Além de alavancar uma série de
pequenas empresas, a preocupação com privacidade tem incentivado grandes
companhias a elevarem seus padrões de segurança\footnote{Apenas
  em três meses de 2016, de abril a junho, a Open WhisperSystems
  (https://whispersystems.org/blog/ ), empresa que desenvolve o
  aplicativo de comunicação Signal, anunciou que a integração de seu
  protocolo de segurança com o Whatsapp estava completa e anunciou
  parcerias com o Facebook e com o Google para integração do mesmo
  protocolo de segurança em outras duas plataformas de comunicação, o
  Allo e o Messenger }.

Talvez a maior comprovação do sucesso da adoção da criptografia ponta a
ponta veio com o vazamento de documentos da CIA em março de 2017. As
informações reveladas pelo Wikileaks indicam que, diferente do modelo
exposto quatro anos antes, a forma de vigilância usada pela CIA explora
vulnerabilidades nos dispositivos dos usuários e não nas bases de dados
das grandes empresas de comunicação online. Essa aparente mudança, fruto
da adoção massiva da criptografia ponta a ponta, representa um
deslocamento do modelo de vigilância em massa para a vigilância contra
alvos específicos. No momento em que escrevemos este texto, por exemplo,
a pressão dos usuários por mais segurança na comunicação interpessoal,
de um lado, e o interesse dessas empresas nas informações que as
permitam construir melhores perfis de consumos, de outro, encontraram um
equilíbrio no ponto em que o conteúdo das mensagens é protegido, mas seu
contexto - os \emph{metadados} no jargão técnico - continua explorado
comercialmente e, consequentemente, segue vulnerável à vigilância em
massa.

Ainda há muito o que se fazer para proteger o direito fundamental dos
cidadãos à privacidade. Do ponto de vista regulatório, o Brasil ainda
não possui legislação específica sobre o tema o que abre brechas para
uma série de abusos. A saída regulatória, porém, não é suficiente sem o
esforço em se promover e desenvolver ferramentas de comunicação seguras.
Essa combinação de abordagens parece ser a estratégia que tem dado mais
frutos.

\chapter{O direito à privacidade\footnote{Originalmente publicado na \emph{Harvard
  Law Review}, v. IV, n. 5, dez. 1890.}}\label{o-direito-uxe0-privacidade}

Samuel D. Warren\\
Louis D. Brandeis

Poderia ser feito somente com base nos princípios de justiça privada,
aptidão moral e conveniência pública que, quando aplicados a um assunto
novo, fazem o Direito Comum sem um precedente; ainda mais quando
recebidos e aprovados pelo uso.

Willes, J., em Millar \emph{versus} Taylor, 4 Burr. 2303, 2312.

Que o indivíduo deve ter proteção integral em pessoa e em propriedade é
um princípio tão antigo quanto o Direito Comum; todavia, tem-se achado
necessário, de tempos em tempos, definir de uma nova maneira a exata
natureza e a extensão de tal proteção. Mudanças políticas, sociais e
econômicas implicam o reconhecimento de novos direitos e o Direito
Comum, em sua eterna juventude, cresce para atender às novas demandas da
sociedade. Assim, em tempos muito antigos, a lei oferecia reparação
somente para a interferência física na vida e propriedade, para
transgressões \emph{vi et armis}. Nessa época, o ``direito à vida''
servia apenas para proteger o sujeito de agressões em suas diversas
formas; liberdade significava liberdade de constrangimentos concretos; e
o direito à propriedade assegurava ao indivíduo suas terras e seu gado.
Mais tarde, houve um reconhecimento da natureza espiritual do homem, de
seus sentimentos e seu intelecto. Gradualmente o escopo desses direitos
legais foi ampliado; e agora o direito à vida veio a significar o
direito de desfrutar a vida --- o direito de ser deixado em paz; o
direito à liberdade garante o exercício de amplos privilégios civis; e o
termo ``propriedade'' cresceu para incluir todas as formas de posse ---
intangíveis tanto quanto tangíveis.

Com o reconhecimento do valor jurídico das sensações, assim, a proteção
contra lesões verdadeiramente corporais foi estendida para proibir meras
tentativas de se produzir tais danos; ou seja, infundir no outro o medo
de tal injúria. Da ação de lesão corporal surgiu a de tentativa de
agressão.\footnote{\emph{Year Book,} Lib. Ass., folio 99, pl. 60 (1348
  ou 1349) parece ser o primeiro caso relatado de reparação de danos por
  tentativa de agressão civil.} Muito depois, surgiu uma proteção
qualificada para o indivíduo contra ruídos e odores ofensivos, poeira e
fumaça e vibração excessiva. Estava criada a lei da
perturbação.\footnote{Tais perturbações são, tecnicamente, danos à
  propriedade; todavia, o reconhecimento do direito de se ter a
  propriedade livre da interferência de tais perturbações também envolve
  o reconhecimento do valor das sensações humanas.} A consideração pelas
emoções humanas, portanto, logo estendeu o âmbito da imunidade pessoal
para além do corpo do indivíduo. Sua reputação, sua posição perante seus
semelhantes, passou a ser considerada e surgiram as leis de calúnia e
difamação.\footnote{\emph{Year Book}, Lib. Ass., folio 177, pl. 19
  (1356), (2 \emph{Finl. Reeves Eng. Law}, 395) parece ser o mais antigo
  caso relatado de uma ação por calúnia.} As relações familiares do
homem tornaram-se uma parte da concepção legal de sua vida e a alienação
do afeto de uma esposa passou a ser tida como compensável.\footnote{Winsmore
  \emph{versus} Greenbank, Willes, 577 (1745).} Ocasionalmente, a lei
era suspensa, como recusa em reconhecer a intrusão por sedução sobre a
honra da família. Mas mesmo nessas situações as demandas da sociedade
eram atendidas. Recorreu-se a uma cruel ficção, a ação \emph{per}
\emph{quod servitium amisit}\footnote{Nota do tradutor: ``No ordenamento
  jurídico norte-americano, os danos não patrimoniais dos familiares da
  vítima mortal ou de lesão corporal grave enquadram-se na ação por loss
  of consortium, que teve como antecedente histórico a ação medieval per
  quod servitium amisit, através da qual o senhor podia exigir uma
  indemnização ao terceiro que agredisse um dos seus servos, sempre que
  de tal agressão resultasse a impossibilidade de o senhor beneficiar da
  força de trabalho desse servo. Por analogia com essa ação
  desenvolveu-se uma outra, a per quod consortium amisit, que reconhecia
  ao marido a faculdade de demandar o terceiro responsável pela lesão da
  sua mulher, sempre que tal determinasse a impossibilidade de o
  primeiro beneficiar do trabalho doméstico, das relações sexuais e/ou
  da companhia antes proporcionadas pela mesma. Esta ação passou a
  designar-se, mais tarde, por loss of

  consortium. Em Inglaterra, tal ação foi abolida pela secção 2 do
  Admnistration of Justice Act de 198286 87. No entanto, nos Estados
  Unidos a figura teve um grande desenvolvimento, mantendo-se atual.
  Neste país, a partir da segunda metade do século XX, a ação passou a
  ser atribuída também às mulheres, sendo, nos dias de hoje, reconhecida
  na maioria dos Estados, discutindo-se se a mesma não deve ser também
  reconhecida aos filhos do lesado (loss of parental consortium) e aos
  membros de uniões de facto.''. In: CASCAREJO, Guilherme. \textbf{DANOS
  NÃO PATRIMONIAIS DOS FAMILIARES DA VÍTIMA DE LESÃO CORPORAL GRAVE}:
  DANOS REFLEXOS OU DANOS DIREITOS? 2014. Tese de Doutorado.
  UNIVERSIDADE DO PORTO.}, e ao se permitir a indenização por danos aos
sentimentos dos pais, uma reparação adequada foi garantida.\footnote{O
  cerne da ação está na impossibilidade dos serviços; mas foi dito que
  ``não temos notícia de nenhum caso relatado, movido por um pai, em que
  o valor de tais serviços fosse considerado como a medida dos danos.''
  Cassoday, J., in Lavery \emph{versus} Crooke, 52 Wis. 612, 623 (1881).
  Primeiramente, foi inventada a ficção de serviço construtivo; Martin
  \emph{versus} Payne, 9 John. 387 (1812). Só então os sentimentos do
  pai, a desonra sobre ele e sobre a sua família, foram aceitos como o
  elemento mais importante do dano. Bedford \emph{versus} McKowl, 3 Esp.
  119 (1800); Andrew \emph{versus} Askey, 8 C. \& P. 7 (1837); Phillips
  \emph{versus} Hoyle, 4 Gray, 568 (1855); Phelin \emph{versus}
  Kenderdine, 20 Pa. St. 354 (1853). A permissão de tais danos pareceria
  reconhecer que a invasão à honra da família é uma injúria à pessoa do
  pai, já que, normalmente, a mera injúria a sentimentos parentais não é
  um elemento de dano, \emph{e.g.}, o sofrimento do pai no caso de uma
  injúria física à criança. Flemington \emph{versus} Smithers, 2 C. \&
  P. 292 (1827); Black \emph{versus} Carrolton R. R. Co., 10 La. Ann. 33
  (1855); Covington Street Ry. Co. \emph{versus} Packer, 9 Bush, 455
  (1872).} Similar à expansão do direito à vida foi o crescimento da
concepção legal de propriedade. Da propriedade corpórea surgiram os
direitos incorpóreos, partindo dela; e então se abriu um amplo domínio
da propriedade intangível, nos produtos e processos mentais,\footnote{``A
  noção do Sr. Ministro Yates, de que nada que não possa ser marcado e
  recuperado por devolução ou compensação constitua propriedade, pode
  ser verdadeira para um estágio inicial da sociedade, quando a
  propriedade existe em sua forma simples e eram simples as reparações
  para a sua violação; todavia, não é verdadeira para um estado mais
  civilizado, quando as relações da vida e dos interesses que dela
  surgem se complicam.'' Erle, J., in Jefferys \emph{versus} Boosey, 4
  H. L. C. 815, 869 (1854).} como obras de literatura e arte,\footnote{Os
  direitos autorais parecem ter sido reconhecidos como uma espécie de
  propriedade privada pela primeira vez na Inglaterra em 1558.
  \emph{Drone} sobre \emph{Copyright}, 54,61.} patrimônios de
marca,\footnote{Gibblett \emph{versus} Read, 9 Mod. 459 (1743) é
  provavelmente o primeiro reconhecimento do patrimônio de marca como
  propriedade.} segredos comerciais e marcas registradas.\footnote{Hogg
  \emph{versus} Kirby, 8 Ves. 215 (1803). Até 1742, o Lorde Hardwicke se
  recusava a tratar uma marca registrada como uma propriedade por cuja
  violação pudesse ser garantida medida cautelar. Blanchard
  \emph{versus} Hill, 2 Atk. 484.}

Tal evolução da lei era inevitável. A intensa vida intelectual e
emocional e a elevação das sensações que vieram junto com o avanço da
civilização deixaram claro para os homens que apenas uma parte da dor,
do prazer e do usufruto da vida reside em coisas físicas. Pensamentos,
emoções e sensações exigiam reconhecimento legal e a bela capacidade de
crescimento que caracteriza o Direito Comum habilitou juízes a
fornecerem a proteção necessária sem a interposição da legislatura.

Invenções e métodos empresariais recentes chamam atenção para o próximo
passo que deve ser tomado para a proteção da pessoa e para garantir ao
indivíduo o que o juiz Cooley chama de o direito ``de ser deixado em
paz''.\footnote{Cooley sobre \emph{Torts}, 2ª ed., p. 29.} Fotografias
instantâneas e corporações de imprensa invadiram o recinto sagrado da
vida privada e doméstica; e numerosos dispositivos mecânicos ameaçam ter
sucesso em concretizar a previsão de que ``o que é sussurrado na
dispensa há de ser proclamado dos telhados''. Por anos tem havido um
sentimento de que a lei deve fornecer algum tipo de reparação para a
circulação não autorizada de retratos de pessoas privadas;\footnote{\emph{Amer.
  Law Reg.} N. S. I. (1869); 12 \emph{Wash. Law Rep.} 353 (1884); 24
  Sol. J. \& Rep. 4 (1879).} e o mal da invasão da privacidade pelos
jornais, profundamente sentido há tempos, só recentemente foi discutido
por um hábil escritor.\footnote{\emph{Scribner's Magazine}, Julho, 1890.
  \emph{``The Rights of the Citizen: To his Reputation''}, por E. L.
  Godkin, Esq., pp. 65, 67.} Os fatos alegados em um caso, algo notório,
trazido para um tribunal inferior em Nova York há alguns
meses,\footnote{Marion Manola \emph{versus} Stevens \& Myers, \emph{N.Y.
  Supreme Court}, \emph{New York Times} de 15, 18 e 21 de junho de 1890.
  A autora da ação alegou que, enquanto atuava no Teatro da Broadway em
  um papel que exigia a sua aparição usando meias-calças, fora
  fotografada com o uso de \emph{flash} sub-repticiamente e sem o seu
  consentimento desde um dos camarotes pelo réu Stevens, administrador
  da companhia \emph{Castle in the Air}, e pelo réu Myers, um fotógrafo,
  e exigia que os réus fossem impedidos de fazer uso da fotografia que
  tiraram. Emitiu-se uma medida cautelar preliminar \emph{ex parte} e
  foi estipulado um tempo para a discussão do requerimento segundo o
  qual a medida cautelar deveria se tornar permanente, mas ninguém
  surgiu então em oposição.} envolveram diretamente a consideração do
direito de circulação de retratos; e a questão de se nossa lei irá
reconhecer e proteger o direito à privacidade nesse e em outros aspectos
deve ser levada à consideração dos tribunais em breve.

Sobre a conveniência --- aliás, a necessidade --- de tal proteção,
acredita-se não existirem dúvidas. A imprensa está ultrapassando, em
todos os sentidos, os limites óbvios da propriedade e da decência. A
fofoca já não é um recurso dos desocupados ou maliciosos, mas se tornou
um negócio, buscado com empenho e descaramento. Para satisfazer um gosto
lascivo, detalhes de relações sexuais são transmitidos e espalhados nas
colunas dos jornais diários. Para ocupar os indolentes, colunas e
colunas são preenchidas com fofocas inúteis, que só podem ser colhidas
por meio da intrusão no círculo doméstico. A intensidade e a
complexidade da vida, concomitantemente com os avanços da civilização,
tornaram necessário algum afastamento do mundo; e o homem, sob a
influência aperfeiçoadora da cultura, tornou-se mais sensível à
publicidade, de modo que a solidão e a privacidade fizeram-se mais
essenciais ao indivíduo; porém a empresa e a invenção modernas, por meio
de invasões a sua privacidade, submeteram-no a dor e sofrimento mental
muito maiores do que poderia ser infligido por meras lesões corporais.
Tampouco o dano trazido por tais invasões se restringe ao sofrimento
daqueles que podem virar temas da imprensa ou de outro tipo de empresa.
Neste, como em outros ramos do comércio, a oferta cria a demanda. Cada
safra de fofoca indecente, assim que colhida, torna-se a semente de
outras e, na proporção direta de sua circulação, resulta na redução de
padrões sociais e de moralidade. Mesmo fofocas aparentemente
inofensivas, circuladas ampla e persistentemente, são potentes para o
mal. Menosprezam tanto quanto pervertem. Menosprezam ao inverter a
importância relativa das coisas, diminuindo assim os pensamentos e
aspirações de um povo. Quando a fofoca pessoal atinge a dignidade da
imprensa, ocupando todo o espaço disponível para assuntos de real
interesse para a comunidade, não é de se espantar que o ignorante e o
leviano se enganem a respeito de sua importância relativa. De fácil
compreensão, apelando para aquele lado fraco da natureza humana que
nunca é totalmente abatido pelos infortúnios e fragilidades dos nossos
vizinhos, não surpreende a ninguém que a fofoca usurpe o lugar de
interesse em cérebros capazes de outras coisas. A trivialidade destrói
ao mesmo tempo a robustez do pensamento e a delicadeza do sentimento.
Nenhum entusiasmo pode florescer, nenhum impulso generoso pode
sobreviver sob sua influência corrosiva.

É nosso propósito considerar se a lei existente proporciona um princípio
que pode ser devidamente invocado para proteger a privacidade do
indivíduo; e, se sim, qual a natureza e a extensão dessa proteção.

Devido à natureza dos instrumentos pelos quais a privacidade é invadida,
a lesão infligida possui uma semelhança superficial com as injustiças
enfrentadas pela lei de calúnia e difamação, uma vez que uma compensação
legal para tal lesão parece envolver o tratamento de meros sentimentos
feridos como uma causa substantiva de ação. O princípio em que se
assenta a lei de difamação abrange, no entanto, uma classe de efeitos
radicalmente diferente daqueles para os quais a atenção é agora
requisitada. Ele trata apenas dos danos à reputação, da lesão causada ao
indivíduo nas suas relações externas com a comunidade, diminuindo-lhe na
estima de seus companheiros. A matéria publicada sobre ele, ainda que
amplamente divulgada e ainda que inadequada para a publicidade, deve, a
fim de ser acionável, ter uma tendência direta a feri-lo em suas
relações com os outros e, mesmo se por escrito ou impressa, deve
sujeitá-lo ao ódio, ao ridículo ou ao desprezo de seus semelhantes --- o
efeito da publicação sobre sua autoestima e seus próprios sentimentos
não configuram um elemento essencial na causa da ação. Em suma, as
injustiças e correlativos direitos reconhecidos pela lei de calúnia e
difamação são, em sua natureza, mais materiais que espirituais. Esse
ramo da lei simplesmente estende a proteção que envolve a propriedade
física para englobar algumas das condições necessárias ou úteis à
prosperidade mundana. Por outro lado, a nossa lei não reconhece nenhum
princípio a partir do qual uma compensação possa ser concedida pela mera
lesão aos sentimentos. Por mais doloridos que sejam os efeitos mentais
causados em outrem por um ato, seja ele puramente gratuito ou mesmo
malicioso, se o ato em si não configurar uma atividade ilícita, o
sofrimento infligido é \emph{dannum absque injuria}. A lesão dos
sentimentos pode de fato ser levada em consideração na determinação do
montante dos danos causados ao se tratar de algo reconhecido como
injúria legal;\footnote{Embora o valor legal dos ``sentimentos'' agora
  seja geralmente reconhecido, distinções têm sido formuladas entre os
  diversos tipos de casos nos quais a compensação pode ou não ser
  obtida. Assim, o medo ocasionado por uma tentativa de agressão
  constitui causa para uma ação, mas o medo ocasionado pela negligência
  não. Desse modo, o medo associado à injúria física oferece fundamento
  para danos qualificados; mas, normalmente, o medo não seguido de lesão
  física não pode ser considerado como um elemento de dano, mesmo quando
  exista uma causa válida para uma ação, como na transgressão
  \emph{quare clausum fregit.} Wyman \emph{versus} Leavitt, 71 Me. 227;
  Canning \emph{versus} Williamstown, 1 Cush. 451. A indenização por
  danos aos sentimentos dos pais, no caso de sedução, abdução de uma
  criança (Stowe \emph{versus} Heywood, 7 All. 118) ou remoção do
  cadáver da criança de seu sepulcro (Meagher \emph{versus} Driscoll, 99
  Mass. 281) são tidos como exceções à regra geral. Por outro lado, a
  injúria contra sentimentos é reconhecida como elemento de dano nas
  ações por calúnia e difamação e por acusação maliciosa. Tais
  distinções, entre os casos em que a injúria contra sentimentos
  constitui ou deixa de constituir causa para ação ou elemento legal de
  dano, não são lógicas, mas certamente funcionam bem como regras
  práticas. Acredita-se que, após o exame das autoridades, chegar-se-á à
  conclusão de que, onde quer que um sofrimento mental substancial tenha
  sido o resultado natural e provável da ação, concedeu-se a compensação
  por danos aos sentimentos; e que nos casos em que nenhum sofrimento
  mental tenha, normalmente, resultado, ou, se resultarem, eles sendo
  naturalmente nada mais que insignificantes, e ausentes os sinais
  visíveis de lesão, dando vazão a uma vasta gama de doenças
  imaginárias, lá os danos não terão sido abonados. As decisões a esse
  respeito ilustram bem a sujeição, em nossa lei, à lógica do senso
  comum.} porém nosso sistema, ao contrário do Direito Romano, não
oferece compensação para o sofrimento mental resultante de mera
blasfêmia e insulto, de uma violação intencional e injustificada da
``honra'' de outros.\footnote{``Injúria, \emph{stricto sensu}, é toda e
  qualquer violação intencional e ilegal da honra, \emph{i.e.,} de toda
  a personalidade do outro.'' ``Agora, comete-se uma ofensa não só ao se
  atingir um homem com um punho, digamos, ou com uma clava, ou mesmo ao
  fustigá-lo, mas também ao se utilizar contra ele de uma linguagem
  abusiva.'' Salkowski, \emph{Roman Law}, pp. 668-9, Nº 2}

Isso não é necessário, contudo, para que se mantenha a visão de que o
Direito Comum reconhece e assegura um princípio aplicável aos casos de
invasão de privacidade, para invocar a analogia que não passa de
superficial, às lesões sofridas tanto por um ataque à reputação ou pelo
o que os civis chamam de uma violação da honra; as doutrinas jurídicas
relativas a infrações contra o que normalmente se denomina direito
consuetudinário à propriedade intelectual e artística são, acredita-se,
apenas instâncias e aplicações de um direito geral à privacidade, que,
devidamente compreendido, concede uma compensação para os males em
consideração.

O Direito Comum assegura a cada indivíduo o direito de determinar,
normalmente, até que ponto seus pensamentos, sentimentos e emoções devem
ser comunicados a outros.\footnote{``É certo que cada homem tenha o
  direito de manter seus próprios sentimentos para si, se ele assim o
  quiser. Ele tem certamente o direito de avaliar se os tornará
  públicos, ou se os confiará apenas à vista de seus amigos.'' Yates,
  J., in Millar \emph{versus} Taylor, 4 Burr. 2303, 2379 (1769).} Sob o
nosso sistema de governo, ele nunca pode ser compelido a expressá-los
(exceto quando na condição de testemunha); e mesmo que sua escolha seja
expressá-los, ele geralmente mantém o poder de estabelecer os limites da
publicidade que lhes deve ser dada. A existência desse direito não
depende do método específico de expressão adotado. É irrelevante que
seja expresso por palavras\footnote{Nicols \emph{versus} Pitman, 26 Cb.
  D. 374 (1884).} ou por sinais,\footnote{Lee \emph{versus} Simpson, 3
  C. B. 871, 881; Daly \emph{versus} Palmer, 6 Blatchf. 256} em
pintura,\footnote{Turner \emph{versus} Robinson, 10 Ir. Ch. 121; S. C.
  ib. 510.} por escultura, ou em música.\footnote{Drone sobre
  \emph{Copyrights}, 102.} A existência desse direito tampouco depende
da natureza ou do valor do pensamento ou emoção, nem da excelência dos
meios de expressão.\footnote{``Sendo esta a lei, qual é o fundamento a
  esse respeito? Não se refere, penso, a nenhuma consideração
  peculiarmente literária. Aqueles com os quais se originou o nosso
  dDireito Ccomum provavelmente não tinham entre os seus muitos méritos
  o de serem patronos das letras; porém eles sabiam do dever e da
  necessidade de se proteger a propriedade, e, com esse objeto geral,
  instituíram regras providencialmente expansivas -- regras capazes de
  se adaptar, elas mesmas, às diversas formas e modos de propriedade que
  a paz e o cultivo poderiam descobrir e introduzir.

  ``O produto do trabalho mental, pensamentos e sentimentos, registrados
  e preservados pela escrita, tornaram-se, conforme o conhecimento
  progrediu e se espalhou, e a cultura do entendimento humano avançou,
  um tipo de propriedade impossível de ser ignorado, e a interferência
  da legislação moderna nesse assunto, através dopelo \emph{stat.} 8
  Anne Estatuto da Rainha Ana, professando pelo seu título ser 'Pelo
  estímulo ao aprendizado', e usando as palavras 'tomada a liberdade',
  no preâmbulo, independentemente de aumentar ou diminuir os direitos
  privados de autores, deixando-os em certa medida na mesma situação,
  chegou à conclusão de que o Direito Comum, ao velar pela proteção à
  propriedade, velava também pela sua segurança, pelo menos antes de uma
  publicação geral com o consentimento do escritor.'' Knight Bruce, V.
  C., in Príncipe Albert \emph{versus} Strange, 2 DeGex \& Sm. 652m 695
  (1849).} A mesma proteção é concedida a de uma carta informal ou uma
nota num diário ao mais valioso poema ou ensaio, de um rascunho ou um
rabisco a uma obra-prima. Em todos esses casos, o indivíduo tem o
direito de decidir se aquilo que é dele deve ser dado ao
público.\footnote{``A questão, no entanto, não está na forma ou no
  tamanho do prejuízo ou vantagem, perda ou ganho. O autor de
  manuscritos, seja ele famoso ou obscuro, pequeno ou grande, tem o
  direito de dizer a seu respeito, se inocente, que independentemente de
  interessantes ou tediosos, leves ou pesados, vendáveis ou não
  vendáveis, eles não serão publicados sem o seu consentimento.'' Knight
  Bruce, V. C., in Príncipe Albert \emph{versus} Strange, 2 DeGex \& Sm.
  652, 694.} Nenhuma outra pessoa tem o direito de publicar suas
produções de nenhuma forma sem consentimento. Esse direito é totalmente
independente do material ou meios nos quais o pensamento, sentimento ou
emoção são expressos. Ele pode existir independentemente de qualquer ser
corpóreo, como em palavras ditas, numa canção cantada, na representação
de um drama. Ou, se expresso em qualquer material, como em um poema por
escrito, o autor pode ter rasgado o papel sem perder qualquer direito
sobre a propriedade da composição em si. Perde-se o direito apenas
quando o próprio autor comunica sua produção ao público --- em outras
palavras, publica-a.\footnote{Duque de Queensberry \emph{versus}
  Shebbeare, 2 Eden, 329 (1758); Bartlett \emph{versus} Crittenden, 5
  McLean, 32, 41 (1849).} Isso é inteiramente independente das leis de
direito autoral e de sua extensão para o domínio da arte. O objetivo
desses estatutos é garantir ao autor, compositor ou artista os lucros
totais decorrentes da publicação; mas a proteção do Direito Comum
permite-lhe controlar absolutamente o ato de publicação e, no exercício
de sua própria discrição, decidir se haverá ou não qualquer
publicação.\footnote{Drone sobre \emph{Copyrights}, pp. 102, 104; Parton
  \emph{versus} Prang, 3 Clifford, 537, 548 (1872); Jefferys
  \emph{versus} Boosey, 4 H. L. C. 815, 867, 962 (1834).} O direito
estatutário não possui qualquer valor a menos que haja uma publicação; o
direito consuetudinário é perdido \emph{assim que} haja uma publicação.

Qual a natureza e a base desse direito para impedir a publicação de
manuscritos ou obras de arte? Afirma-se ser a aplicação de um direito de
propriedade;\footnote{``A questão será se o projeto de lei constatou
  fatos nos quais a corte poderá se aprofundar, como um caso de
  propriedade civil que ela deve proteger. A medida cautelar não pode
  ser mantida sobre qualquer princípio deste tipo, de que se uma carta
  tiver sido escrita no decorrer dae uma amizade, tanto a continuidade
  quanto aou interrupção da amizade oferecem um motivo para a
  interferência da cortedos tribunais''. Lord Eldon in Gee \emph{versus}
  Pritchard, 2 Swanst. 402, 413 (1818).

  ``É baseado no princípio, portanto, da proteção à propriedade que o
  Direito Comum, nos casos de inexistência de ajuda ou interferência
  denão assistidos ou lesados por estatutos, dá guarida à privacidade e
  à seclusão do pensamento e dos sentimentos confiados à escrita, e cujo
  autor deseja que permaneçam pouco conhecidos.'' Knight Bruce, V. C.,
  in Príncipe Albert \emph{versus} Strange, 2 DeGex \& Sm. 652, 695.

  ``Assumindo que razões de conveniência e políticas públicas
  nuncajamais podem servir como base única da jurisdição civil, a
  questão sobre se com alguma acerca das possíveis fundamentaçãoões,
  sobre as quais o autor da ação estaria apto à compensação que ele
  exigeexigida fica por ser respondida; e nos parece que existehá apenas
  uma fundamentação sobre a qual seu direito de demandar -- e o da nossa
  jurisdição de oferecer - para a exigência, para o oferecimento de
  garantias da parte de nossa jurisdição e para o estabelecimento da
  compensação pode ser colocada. Devemos nos satisfazer com o fato de
  que a publicação de cartas particulares, sem o consentimento do autor,
  é uma invasão a um direito exclusivo de propriedade que permanece no
  autor, mesmo quando as cartas foram destinadas ao correspondente e
  ainda se encontram em sua possessão.'' Duer, J., in Woolsey
  \emph{versus} Judd, 4 Duer, 379, 384 (1855).} e nenhuma dificuldade
surge em se aceitar essa visão, desde que tenhamos que lidar apenas com
a reprodução das composições literárias e artísticas. Elas certamente
possuem muitos dos atributos da propriedade comum: são transferíveis;
têm um valor; e a publicação ou reprodução é um uso pelo qual esse valor
é concretizado. Mas quando o valor da produção encontra-se não no
direito de se obter os lucros decorrentes da publicação, mas na paz de
espírito ou no alívio proporcionado pela capacidade de impedir qualquer
tipo de publicação, é difícil se considerar esse direito como de
propriedade, na aceitação comum desse termo. Um homem registra em uma
carta a seu filho, ou em seu diário, que ele não jantou com sua esposa
em um determinado dia. Ninguém em cujas mãos caíssem esses papéis
poderia publicá-los para o mundo, mesmo que a posse dos documentos
tivesse sido obtida legalmente; e a proibição não estaria restrita à
publicação de uma cópia da carta em si ou da nota no diário; a restrição
estende-se também à publicação dos conteúdos. Qual é a coisa que está
protegida? Certamente, não o ato intelectual de se registrar o fato de
que o marido não jantou com sua esposa, mas esse fato em si. Não é o
produto intelectual, mas a ocorrência doméstica. Um homem escreve uma
dúzia de cartas para pessoas diferentes. Ninguém teria a permissão para
publicar uma lista das cartas escritas. Se as cartas ou os conteúdos dos
diários fossem protegidos como composições literárias, o âmbito da
proteção concedida deveria ser o mesmo garantido pela lei de direito
autoral a uma publicação escrita. Mas a lei de direito autoral não
impediria uma enumeração das cartas ou a publicação de alguns dos fatos
nelas contidos. Os direitos autorais de uma série de pinturas ou
gravuras impediria uma reprodução das pinturas como fotos; mas não
impediria a publicação de uma lista ou mesmo uma descrição
delas.\footnote{``Uma obra publicada legitimamente, no sentido comum do
  termo, figura a esse respeito, creio, de um modo diferente de uma obra
  que nunca esteve nessa situação. A primeira pode estar sujeita a
  traduções, adaptações, análises, exibições de suas partes,
  complementações e tratamentos distintos, de um modo que a segunda não
  está.

  Suponha, no entanto, -- ao invés de uma tradução, uma adaptação, ou
  uma resenha -- o caso de um catálogo; suponha que um homem tenha
  composto uma variedade de obras literárias ('inocentes'', para usar a
  expressão de Lorde Eldon) que ele jamais tenha imprimido ou publicado,
  ou tenha perdido o direito de impedir sua publicação; suponha que o
  conhecimento dessas obras tenha sido indevidamente obtido por alguma
  pessoa inescrupulosa, que imprime, visando a circulação, um catálogo
  descritivo, ou até mesmo uma mera lista dos manuscritos, sem
  autoridade ou consentimento -- a lei permite isso? Eu espero e
  acredito que não. Os mesmos princípios que previnem formas mais
  francas de pirataria devem, creio eu, também reger um caso como esse.

  Ao se publicar a respeito de um homem, que ele escreveu para pessoas
  particulares ou sobre assuntos particulares, pode-se expô-lo não
  apenas ao sarcasmo, mas até à ruína. Pode haver em sua posse cartas
  devolvidas, que ele escreveu para antigos correspondentes, e cujas
  relações, por mais inofensivas que fossem, podem não ser motivo de
  orgulho em sua vida subsequente; ou seus escritos podem ainda, de
  outro modo, não casar perfeitamente com seus hábitos externos e
  posição social. Ainda hoje há ocupações nas quais é perigoso ser
  condenado por literatura, embora tal perigo seja às vezes evitado.

  Mais uma vez, os manuscritos podem pertencer a um homem tal que, em
  virtude de seu nome, uma mera lista venha a ser objeto de curiosidade
  geral. Quantas pessoas não poderiam ser mencionadas cujo catálogo de
  escritos não publicados poderia, ao longo de suas vidas ou depois
  delas, garantir uma venda imediata!'' Knight Bruce, V. C., in Príncipe
  Albert \emph{versus} Strange, 2 DeGex \& Sm. 652, 693.} Ainda no
famoso caso do Príncipe Albert \emph{vs.} Estranho, o tribunal
considerou que a regra consuetudinária proibia não apenas a reprodução
das gravuras que o demandante e a rainha Vitória tinham feito para seu
próprio prazer, mas também ``a publicação (pelo menos por impressão ou
escrita), ainda que não por cópia ou semelhança, sua descrição mais ou
menos limitada ou resumida, seja na forma de um catálogo ou de outra
forma''.\footnote{``Uma cópia ou impressão das gravuras seria apenas um
  meio de transmitir conhecimento e informação acerca do original e não
  uma listagem e descrição do mesmo? Os meios são diferentes, mas o
  objeto e o efeito são similares; pois em ambos o objeto e efeito é
  trazer ao conhecimento do público algo da obra e da composição não
  publicadas pelo autor, as quais ele tem o direito de reservar em sua
  totalidade para o seu uso e prazer particulares, e de ocultar
  completamente, ou o quanto quiser, do conhecimento de outrem. Casos
  baseados em simplificações, traduções, citações e críticas de obras
  publicadas não têm nenhuma relevância para a presente questão; todos
  eles dependem da extensão do direito sob os atos relativos aos
  direitos autorais, e não possuem qualquer analogia com os direitos
  exclusivos do autor de composições não publicadas, que dependem
  inteiramente do direito à propriedade do Direito Comum.'' Lorde
  Cottenham in Príncipe Albert \emph{versus} Strange, 1 McN. \& G. 23,
  43 (1849). ``O Sr. Ministro Yates, em Millar \emph{versus} Taylor,
  afirmou que o caso de um autor era idêntico ao de um inventor de uma
  nova máquina mecânica; que ambas invenções originais se sustentam
  sobre o mesmo fundamento em matéria de propriedade, seja o caso
  mecânico ou literário, um poema épico ou um planetário; que a
  imoralidade de se piratear a invenção de outrem era tão grande quanto
  a de roubar suas ideias. A propriedade em obras mecânicas ou obras de
  arte, executadas por um homem para o seu próprio entretenimento,
  instrução ou uso, tem o direito de subsistir, certamente, e pode,
  antes de sua publicação pelo autor, ser invadida, não apenas pela
  cópia, mas pela descrição ou catálogo, conforme me parece. Um catálogo
  de tais obras pode por si só ser valioso. Ele também pode mostrar as
  maneiras e os movimentos da mente, os sentimentos e o gosto do
  artista, especialmente quando não profissional, de modo tão efetivo
  quanto uma lista de seus papéis. O portfólio ou o estúdio têm tanto a
  declarar quanto a escrivaninha. Um homem pode se empregar no privado
  de uma maneira bastante inofensiva, mas que, exposta à sociedade, pode
  destruir o conforto de sua vida ou até mesmo o seu sucesso nela. Cada
  pessoa, todavia, tem o direito, conforme o meu entendimento, de dizer
  que o produto de suas horas privadas não está mais sujeito à
  publicação sem o seu consentimento porque a publicação lhe deva ser
  louvável ou vantajosa, do que o seria em circunstâncias opostas.''

  ``Eu penso, portanto, não apenas que o réu aqui presente esteja
  invadindo ilegalmente os direitos do autor da ação, mas também que a
  invasão é de um tal tipo, e afeta a propriedade de tal modo, que
  garante ao autor da ação a reparação preventiva de uma medida
  cautelar; e, se nada além disso, tampouco, certamente, nada menos,
  porque se trata de uma intromissão -- uma intromissão imprópria e
  indecente, uma intromissão que não apenas viola as regras
  convencionais, mas que é ofensiva ao sentimento de propriedade
  inerente a todo homem -- se o termo intromissão realmente descrever de
  um modo preciso a espionagem sórdida da privacidade da vida doméstica,
  do lar (palavra outrora sagrada entre nós), o lar de uma família cuja
  vida e conduta formam um reconhecido direito, embora não o seu único e
  inquestionável direito, ao mais distinto respeito deste país.'' Knight
  Bruce, V. C., in Príncipe Albert \emph{versus} Strange, 2 DeGex \& Sm.
  652, 696, 697.} De modo similar, uma coleção inédita de notícias que
não possua qualquer elemento de caráter literário é protegida contra a
pirataria.\footnote{Kiernan \emph{versus} Manhattan Quotation Co., 50
  How. Pr. 194 (1876).}

O fato dessa proteção não poder se apoiar sobre os direitos de
propriedade literária ou artística, no sentido exato do termo, aparece
mais claramente quando o objeto para o qual é invocada a proteção não
está nem sequer sob a forma de propriedade intelectual, mas tem os
atributos de uma propriedade comum tangível. Suponha-se que um homem
tenha uma coleção de pedras preciosas ou de curiosidades que mantém
privada: dificilmente se alegaria que qualquer pessoa poderia publicar
um catálogo desses itens; e ainda assim os artigos enumerados certamente
não são propriedade intelectual no sentido legal, não passam de uma
coleção de fogões ou de cadeiras.\footnote{``O advogado dos réus diz que
  um homem que toma conhecimento da propriedade de outro sem o seu
  consentimento não está, com base em qualquer regra ou princípio
  possivelmente aplicáveis por uma corte de justiça (independentemente
  de quão confidencial ele tenha mantido ou tentado manter isso)
  proibido de comunicar e publicar tal conhecimento sem o seu
  consentimento para o mundo, de informar o mundo do que se trata esta
  propriedade ou de descrevê-la publicamente, seja oralmente, por
  escrito ou pela palavra impressa.

  Eu ponho, no entanto, em dúvida se é seguro -- no que diz respeito a
  uma propriedade de natureza privada, cujo dono, sem infringir o
  direito de qualquer outra pessoa, pode e de fato mantém num estado de
  privacidade -- que uma pessoa que, sem o consentimento do dono,
  expresso ou implícito, adquiriu conhecimento dela possa legitimamente
  tirar vantagem do conhecimento assim adquirido para publicar sem o seu
  consentimento uma descrição da propriedade.

  ``É provavelmente verdade que uma tal publicação pode de certo modo
  ser ou remeter à propriedade de um tipo tal que tornaria a questão
  sobre a legalidade do ato demasiado trivial para merecer atenção.
  Todavia, eu posso imaginar casos nos quais um ato dessa natureza possa
  se dar em circunstâncias tais, ou remeter à propriedade de um tipo
  tal, que o assunto afetaria em peso os interesses ou sentimentos do
  dono, ou ambos. Por exemplo, a natureza e intenção de uma obra
  inacabada de um artista, prematuramente levada ao conhecimento do
  mundo, pode ser dolorosa e profundamente prejudicial para ele; nem
  seria difícil sugerir outros exemplos.

  ``Foi sugerido que publicar, por exemplo, um catálogo das pedras,
  moedas, antiguidades ou outras curiosidades do gênero de algum
  colecionador sem o seu consentimento seria fazer uso de sua
  propriedade sem o seu consentimento; e é verdade, certamente, que um
  procedimento desse tipo poderia tanto amargurar a vida de um
  colecionador quanto insuflar o orgulho de outro -- poderia não ser
  apenas uma calamidade total --, mas poderia causar, ao dono, danos no
  sentido mais vulgar do termo. Tais catálogos, mesmo quando não
  descritivos, são normalmente procurados e algumas vezes atingem preços
  bastante substanciais. Portanto, esses e outros casos similares não
  são necessariamente exemplos de mera dor infringida em matéria de
  sentimento ou imaginação; eles podem ser isso e algo mais.'' Knight
  Bruce, V. C., in Príncipe Albert \emph{versus} Strange, 2 DeGex \& Sm.
  652, 689, 690.}

A crença de que a ideia de propriedade em seu sentido estrito era a base
da proteção de manuscritos inéditos levou um hábil tribunal a recusar,
em vários casos, medidas contra a publicação de cartas particulares, sob
o fundamento de que ``cartas que não possuem os atributos de composições
literárias não são propriedades sujeitas à proteção''; e que era
``evidente que o demandante não poderia ter considerado as cartas a
partir de qualquer valor de produções literárias, pois uma carta não
pode ser considerada de valor para um autor se ele nunca consentiria em
publicá-la''.\footnote{Hoyt \emph{versus} Mackenzie, 3 Barb. Ch. 320,
  324 (1848); Wetmore \emph{versus} Scovell, 3 Edw. Ch. 515 (1842).
  Veja-se Sir Thomas Plumer in 2 Ves. \& B. 19 (1813).} Mas essas
decisões não foram seguidas,\footnote{Woolsey \emph{versus} Judd, 4
  Duer, 379, 404 (1855). ``Foi decidido, felizmente, para o bem-estar da
  sociedade, que o autor de cartas, ainda que escritas sem qualquer fim
  lucrativo ou qualquer ideia de propriedade literária, possui tal
  direito de propriedade sobre elas, que elas não podem ser publicadas
  sem o seu consentimento, a menos que propósitos de justiça civil ou
  criminal exijam a publicação.'' Sir Samuel Romilly, \emph{arg.,} in
  Gee \emph{versus} Pritchard, 2 Swanst. 402, 418 (1818). Mas veja-se
  High sobre medidas cautelares, 3ª ed., § 1012, \emph{contra.}} e não
pode ser considerado assentado que a proteção conferida pelo Direito
Comum para o autor de qualquer escrito é inteiramente independente de
seu valor pecuniário, de seus méritos intrínsecos ou de qualquer
intenção de publicá-lo e, claro, também totalmente independente do
material, se houver, sobre o qual ou no modo pelo qual o pensamento ou
sentimento foi expresso.

Embora os tribunais tenham afirmado que fundamentaram suas decisões nas
bases estritas da proteção à propriedade, ainda há reconhecimentos de
uma doutrina mais liberal. Assim, no caso já referido do Príncipe Albert
\emph{vs.} Estranho, as opiniões tanto do vice-chanceler quanto do
lorde-chanceler na apelação mostram uma percepção mais ou menos definida
de um princípio mais amplo do que aqueles que foram maiormente
discutidos e no qual ambos colocam sua confiança suprema. O
vice-chanceler Cavaleiro Bruce referiu-se à publicação do que um homem
havia ``escrito a pessoas específicas ou sobre assuntos específicos''
como um exemplo de divulgação possivelmente prejudicial sobre assuntos
particulares que os tribunais deveriam impedir no caso adequado; ainda
assim, é difícil perceber como, em tal caso, qualquer direito de
propriedade, em sentido restrito, poderia ser trazido para a questão, ou
por que ele não deveria ser igualmente aplicado se tais publicações
poderiam ser restringidas quando ameaçavam expor a vítima não apenas ao
sarcasmo, mas à ruína, se ameaçava amargar sua vida. Privar um homem dos
potenciais lucros a serem obtidos com a publicação de um catálogo de
suas joias não pode ser \emph{per se} algo errado para ele. A
possibilidade de lucros futuros não é um direito de propriedade que a
lei normalmente reconhece; isso deve, portanto, ser uma infração de
outros direitos que constituem o ato ilícito e essa infração é
igualmente injusta se seus resultados impedem os lucros que o indivíduo
poderia obter, seja por dar ao assunto uma publicidade desagradável para
ele, seja por oferecer uma vantagem às custas de sua dor mental e
sofrimento. Se a ficção de propriedade, em sentido estrito, deve ser
preservada, é ainda verdade que o resultado atingido pelo fofoqueiro é
alcançado pelo uso daquilo que é do outro, os fatos relativos a sua vida
privada, que ele decidiu manter privados. O lorde Cottenham declarou que
um homem ``é intitulado a ser protegido para o uso e gozo exclusivo
daquilo que é exclusivamente seu'' e citou com aprovação o parecer do
lorde Eldon, como relatado em um bilhete manuscrito sobre o caso Wyatt
\emph{vs.} Wilson em 1820, a respeito de uma gravura de George III
durante a sua doença, no sentido de que ``se um dos médicos do falecido
rei tivesse mantido um diário com o que ouviu e viu, o tribunal não
haveria, enquanto o rei estivesse vivo, permitido sua impressão e
publicação''; e o lorde Cottenham declarou, a respeito dos atos dos réus
no caso, diante dele, que ``a privacidade é o direito invadido''. Mas se
a privacidade é uma vez reconhecida como um direito suscetível de
proteção legal, a interposição dos tribunais não pode depender da
natureza particular das injúrias resultantes.

Essas considerações levam à conclusão de que a proteção conferida a
pensamentos, sentimentos e emoções expressos por meio da escrita ou das
artes, na medida em que consiste na prevenção da publicação, é apenas
uma instância da aplicação do direito mais geral do indivíduo de ser
deixado em paz. É como o direito de não ser agredido ou espancado, o
direito de não ser aprisionado, o direito de não ser processado
maliciosamente, o direito de não ser difamado. Em cada um desses
direitos, como de fato em todos os outros direitos reconhecidos pela
lei, existe a qualidade inerente de ser adquirido ou controlado --- e
(como isto é o atributo distintivo da propriedade) talvez haja alguma
propriedade em se falar de direitos como o de propriedade. Mas,
obviamente, carregam pouca semelhança com o que normalmente é
compreendido sob esse termo. O princípio que protege escritos e todas as
outras produções pessoais, não contra o roubo e a apropriação física,
mas contra a publicação de qualquer forma, na realidade não é o
princípio da propriedade privada, mas o de uma personalidade
inviolável.\footnote{``Mas uma questão foi sugerida sobre se cartas
  meramente privadas, sem a pretensão de composições literárias, estão
  sujeitas à proteção de uma medida cautelar do mesmo modo que as
  composições de caráter literário. Essa dúvida surgiu provavelmente do
  hábito de não se distinguir entre os diferentes direitos de
  propriedade pertencentes a um manuscrito inédito e os que pertencem a
  um livro publicado. Os últimos, conforme declarei em outra menção, se
  referem ao direito de se obter lucros da publicação. Os primeiros são
  o direito de controlar o ato de publicação e de decidir se haverá
  qualquer publicação. Chamaram-no de um direito de propriedade;
  expressão talvez não muito satisfatória, mas, por outro lado,
  suficientemente descritiva para um direito que, embora incorpóreo,
  envolve muitos dos elementos essenciais da propriedade e é, pelo
  menos, positiva e precisa. Essa expressão não nos deixa dúvidas quanto
  ao significado a ela dado pelos sábios juízes que a usaram quando a
  aplicaram aos casos de manuscritos inéditos. Eles obviamente não
  pretenderam usá-la em nenhum outro sentido, além do de contradição aos
  meros interesses de sentimento e o fizeram para descrever um direito
  substancial de interesse legal.'' Curtis sobre \emph{Copyright}, pp.
  93,94.

  A semelhança entre o direito de se evitar a publicação de um
  manuscrito inédito e os bem reconhecidos direitos de imunidade pessoal
  se encontra no seu tratamento em relação aos direitos dos credores. O
  direito de se evitar tal publicação e o direito de ação quando da
  infração - como a causa de ação por uma tentativa de agressão,
  difamação ou acusação maliciosa - não são bens disponíveis para os
  credores.

  ``Não há nenhuma lei que possa obrigar um autor a publicar. Ninguém
  pode determinar essa questão fundamental da publicação fora o autor.
  Seus manuscritos, por mais valiosos que sejam, não podem, sem o seu
  consentimento, ser confiscados pelos credores como propriedade.''
  McLean, J. in Bartlett \emph{versus} Crittenden, 5 McLean, 32, 37
  (1849).

  Também se defendeu que, mesmo quando os direitos do remetente não
  forem reivindicados, o destinatário de uma carta não possui nela uma
  tal propriedade que passe ao seu encarregado ou administrador como um
  bem vendável. Eyre \emph{versus} Higbee, 22 How. Pr. (N.Y.) 198
  (1861).

  ``O próprio significado da palavra 'propriedade' em seu sentido legal
  é 'aquilo que é peculiar ou próprio a qualquer pessoa; aquilo que
  pertence exclusivamente a alguém'. O primeiro significado da palavra
  da qual ela deriva -- \emph{proprius --} é 'próprio a alguém'.'' Drone
  sobre \emph{Copyright}, p. 6.

  É claro que uma coisa deve ser passível de identificação para poder
  ser objeto de posse exclusiva. Mas quando sua identidade pode ser
  determinada de modo que a posse individual possa ser reivindicada, é
  irrelevante a sua corporeidade ou incorporeidade.}

Se estivermos corretos nesta conclusão, a lei existente proporciona um
princípio que pode ser invocado para proteger a privacidade do indivíduo
da invasão tanto pela tão intrépida imprensa, quanto pelo fotógrafo ou o
detentor de qualquer outro dispositivo moderno para a gravação ou
reprodução de cenas ou sons. Pois a proteção garantida não está limitada
pelas autoridades àqueles casos em que qualquer meio ou forma de
expressão específica foram adotados, nem aos produtos intelectuais. A
mesma proteção é conferida a emoções e sensações expressas em uma
composição musical ou outra obra de arte, como uma composição literária;
e palavras faladas, uma pantomima encenada, uma sonata executada não são
menos autorizadas ao direito de proteção do que se cada uma delas
tivesse sido reduzida à escrita. A circunstância de que um pensamento ou
emoção hajam sido registrados de forma permanente torna mais fácil a sua
identificação, o que, consequentemente, pode ser importante do ponto de
vista das provas, mas não possui nenhum significado como uma questão do
direito substantivo. Se, então, as decisões indicam um direito geral de
privacidade para pensamentos, emoções e sensações, estes devem receber a
mesma proteção quando se expressam por escrito ou na conduta, na
conversa, em atitudes ou na expressão facial.

Deve-se incentivar que uma distinção seja feita entre a expressão
deliberada de pensamentos e emoções em composições literárias ou
artísticas e a expressão casual e muitas vezes involuntária dada a eles
na conduta corriqueira da vida. Em outras palavras, pode ser sustentado
que a proteção conferida é concedida a produtos conscientes do trabalho,
talvez como um incentivo pelo esforço.\footnote{``Tal sendo, acredito, a
  natureza e a fundamentação do Direito Comum para os manuscritos,
  independentemente de adições ou subtrações parlamentares, sua eficácia
  não pode ficar necessariamente confinada aos assuntos literários. Isso
  implicaria limitar a regra ao exemplo. Onde quer que o produto do
  trabalho esteja sujeito à invasão de maneira análoga, deve haver,
  suponho, um direito à proteção ou reparação análogo.'' Knight Bruce,
  V. C., in Prince Albert \emph{versus} Strange, 2 DeGex \& Sm. 652,
  696.} Essa discórdia, por mais plausível que seja, tem, de fato, pouco
para ser recomendada. Se a quantidade de trabalho envolvido for adotada
como avaliação, podemos muito bem descobrir que o esforço em se ter uma
conduta apropriada nos negócios e nas relações domésticas fora muito
maior do que aquele envolvido em pintar um quadro ou escrever um livro;
alguém poderia achar muito mais fácil expressar sentimentos elevados em
um diário do que na conduta de uma vida nobre. Se a avaliação das
intenções do ato for adotada, muitas correspondências casuais para as
quais é atualmente concedida proteção integral seriam excluídas da
operação benéfica das regras existentes. Depois das decisões, negando a
distinção que se tenta fazer entre as produções literárias que se
pretendiam publicar e aquelas que não, todas as considerações da
quantidade de trabalho envolvida, o grau de deliberação, o valor do
produto e a intenção da publicação devem ser abandonadas, e não se
discerne nenhuma base sobre a qual o direito de restringir a publicação
e a reprodução das chamadas obras literárias e artísticas possa se
fundamentar além do direito à privacidade, como parte do direito mais
geral à imunidade da pessoa --- o direito do indivíduo à própria
personalidade.

É importante ressaltar que, em alguns casos nos quais foi conferida
proteção contra uma publicação indevida, a tutela foi afirmada não com
base no princípio de propriedade, ou pelo menos não totalmente, mas no
princípio de uma alegada violação de um contrato implícito ou confiança
ou confidência.

Assim, em Abernethy \emph{vs.} Hutchinson, 3 L. J. ch. 209 (1825), em
que o autor, um distinto cirurgião, buscava conter a publicação na
\emph{Lancet} de palestras inéditas por ele apresentadas no Hospital de
São Bartolomeu, em Londres, o lorde Eldon duvidou se poderia haver
propriedade de palestras que não foram transcritas, mas concedeu a
medida cautelar em razão da violação de confiança, sustentando ``que
quando as pessoas foram admitidas como alunos ou de outra forma para
assistir a estas palestras, embora realizadas oralmente e embora as
partes pudessem chegar ao ponto, se fossem capazes, de anotar tudo
através da taquigrafia, poderiam fazê-lo apenas para efeitos de sua
própria informação e não para publicação com fins lucrativos de algo
para o qual não tinham obtido o direito de vender''.

Em Príncipe Albert \emph{vs. Estranho}, I McN. \& G. 25 (1849), o lorde
Cottenham, na apelação, reconhecendo o direito de propriedade nas
gravuras que, por si só, justificaria a emissão da medida cautelar,
afirmou, após discutir a evidência, que ele era obrigado a assumir que a
posse da gravura pelo réu tinha ``sua fundação em uma quebra de
confiança ou contrato'' e também sobre tal princípio o título do
requerente para a medida cautelar foi totalmente sustentada.

Em Tuck \emph{vs.} Priester, 19 Q.B.D. 639 (1887), os autores eram
proprietários de um quadro e empregaram o réu para elaborar certo número
de cópias. Ele assim o fez, mas fez também uma série de outras cópias
para si mesmo e as ofereceu à venda na Inglaterra por um preço menor.
Posteriormente, os queixosos registraram seus direitos autorais sobre o
quadro e então entraram com uma ação judicial por uma medida cautelar e
por perdas e danos. Os juízes divergiram quanto à aplicação das leis de
direito autoral para o caso, mas consideraram por unanimidade que,
independentemente desses atos, os requerentes tinham o direito à medida
cautelar e a uma reparação por perdas e danos por quebra de contrato.

Em Pollard \emph{vs.} Photographic Co., 40 Ch. Div. 345 (1888), um
fotógrafo que tirou uma foto de uma senhora em circunstâncias normais
foi proibido de exibi-la e também de vender suas cópias, seguindo o
princípio de que se tratava do descumprimento de um termo implícito no
contrato e também uma violação de confiança. O juiz North interrompeu a
argumentação do advogado do requerente com o questionamento: ``Você
contesta o fato de que se a imagem negativa foi tirada às escondidas, a
pessoa que a tirou pode exibir cópias?''. E o advogado respondeu:
``Nesse caso não haveria nenhuma confiança ou consideração que sustente
um contrato''. Mais tarde, o advogado do réu argumentou que ``uma pessoa
não tem nenhuma propriedade sobre suas próprias feições; na ausência de
uma ação difamatória ou ilegal, não há nenhuma restrição sobre o uso dos
negativos pelo fotógrafo''. Porém o tribunal, apesar de encontrar
expressamente uma violação do contrato e de confiança suficiente para
justificar a sua interposição, ainda parece ter sentido a necessidade de
apoiar sua decisão também em um direito de propriedade\footnote{``A
  questão, portanto, é se se justifica que um fotógrafo que tenha sido
  contratado por um ou uma cliente para tirar seu retrato imprima cópias
  de tal fotografia para seu uso próprio, venda e se disponha delas, ou
  as exiba publicamente através de propaganda ou outro meio, sem a
  autorização expressa ou implícita de tal cliente. Eu digo 'expressa ou
  implícita', porque frequentemente é permitido que um fotógrafo, por
  sua própria solicitação, tire uma fotografia de uma pessoa em
  circunstâncias tais que a venda subsequente estava contemplada por
  ambas partes ainda que não tenha sido efetivamente mencionada. Para a
  questão assim colocada, minha resposta é negativa: não se justifica
  que o fotógrafo o faça. Quando uma pessoa obtém informações durante um
  emprego confidencial, a lei não lhe permite fazer qualquer uso
  impróprio das informações assim obtidas e uma medida cautelar é
  garantida, se necessário, para impedir tal uso; como, por exemplo,
  impedir um empregado de divulgar as contas de seu chefe ou um advogado
  de tornar públicos assuntos de seu cliente dos quais tomaram
  conhecimento ao longo de tal emprego. Novamente, a lei é clara quando
  diz que uma quebra de contrato, expressa ou implícita, pode ser
  coibida por uma medida cautelar. A meu ver, o caso do fotógrafo se
  insere nos princípios dos quais ambas classes de casos dependem. A
  razão pela qual ele é contratado e pago é fornecer a seu cliente um
  número requisitado de fotografias impressas de um determinado objeto.
  Para esse fim o negativo é tirado pelo fotógrafo sobre o vidro; e a
  partir desse negativo cópias podem ser impressas em números muito
  maiores do que os normalmente requisitados pelo cliente. O cliente que
  posa para o negativo, assim, põe o poder de reprodução do objeto nas
  mãos do fotógrafo; e, na minha opinião, o fotógrafo que usa o negativo
  para produzir outras cópias para o seu próprio uso, sem autorização,
  está abusando do poder confidencialmente posto em suas mãos apenas
  para fins de fornecimento ao cliente; e além disso, eu defendo que o
  negócio entre o cliente e o fotógrafo inclui, por implicação, um
  acordo segundo o qual as impressões feitas a partir do negativo são
  destinadas unicamente ao uso do cliente.'' Ao se referir às opiniões
  emitidas em Tuck \emph{versus} Priester, 19 Q. B. D. 639, o sábio
  Ministro continua: ``Então o Sr. Ministro Lindley diz: `Eu primeiro
  lidarei com a medida cautelar, que se sustenta, ou poderia se
  sustentar, sobre uma base totalmente diferente tanto das penalidades
  quanto dos danos. Parece-me que a relação entre os autores da ação e o
  réu era tal que, independente do fato dos requerentes terem ou não
  direitos autorais, o réu atuou de maneira que o torna sujeito a uma
  medida cautelar. Ele fora empregado pelos autores da ação para fazer
  um certo número de cópias da imagem e esse emprego trazia consigo a
  implicação necessária de que o réu não deveria fazer cópias adicionais
  para si ou para vendê-las neste país, competindo com o seu empregador.
  Tal conduta de sua parte é uma flagrante quebra de contrato e uma
  flagrante quebra de confiança e, no meu julgamento, claramente
  intitula os autores da ação a uma medida cautelar, tenham eles os
  direitos autorais sobre a foto ou não.' Esse é o caso mais visível,
  uma vez que o contrato fora firmado por escrito; e ainda assim se
  considerou uma condição implícita a de que o réu não pudesse fazer
  nenhuma cópia para si mesmo. A frase `uma flagrante quebra de
  confiança' usada pelo Sr. Ministro Lindley nesse caso se aplica com
  igual força para o presente, quando os sentimentos de uma dama são
  feridos ao descobrir que o fotógrafo, que ela contratara para tirar um
  retrato seu para uso pessoal, estava exibindo e vendendo cópias dele
  publicamente.'' North, J., in Pollard \emph{versus} Photographic Co.,
  40 Ch. D. 345, 349-352 (1888).

  ``Também pode-se dizer que os casos aos quais me referi são todos
  casos nos quais havia alguma infração ao direito de propriedade
  baseado no reconhecimento pela lei da proteção devida aos produtos da
  habilidade particular de um homem ou de seu trabalho mental; ao passo
  que no presente caso a pessoa fotografada não fez nada para merecer
  tal proteção, que se destina à prevenção de ofensas legais e não de
  meras mágoas sentimentais. Mas uma pessoa cuja fotografia é tirada por
  um fotógrafo não é, dessa maneira, abandonada pela lei, pois o Ato de
  25 e 26 Vic., c. 68, S. I. prevê que quando o negativo de qualquer
  fotografia for feito ou executado para ou em nome de outra pessoa por
  um bem ou um valor, a pessoa encarregada de fazer ou executar o mesmo
  não deverá reter o seu direito autoral, a menos que este lhe seja
  expressamente reservado por acordo escrito, assinado pela pessoa para
  quem ou em nome de quem o mesmo for assim feito ou executado; mas o
  direito autoral deverá pertencer à pessoa para quem ou em nome de quem
  o mesmo for feito ou executado.

  O resultado é que, no presente caso, o direito autoral da fotografia
  está em um dos autores da ação. É verdade, sem dúvida, que a seção 4
  do mesmo ato prevê que nenhum proprietário de direitos autorais estará
  apto a beneficiar-se do ato até o seu registro e que nenhuma ação
  deverá ser movida sobre feitos anteriores ao registro; e , eu presumo,
  foi porque a fotografia da autora da ação não foi registrada que esse
  ato não foi mencionado pelo advogado no decorrer de sua argumentação.
  Mas, embora a proteção contra a sociedade em geral conferida pelo ato
  não possa ser efetivada até o registro, isso não destitui os autores
  de seu direito, garantido pelo Direito Comum, de ação contra o réu por
  quebra de contrato e quebra de confiança. Isso é bastante claro a
  julgar pelos casos de Morison \emph{versus} Moat {[}9 Hare, 241{]} e
  Tuck \emph{versus} Priester {[}19 Q. B. D. 629{]} já mencionados, no
  último dos quais o mesmo ato do Parlamento estava em questão.'' Per
  North, J., ibid. p. 352.

  Essa linguagem sugere que o direito de propriedade em fotografias ou
  retratos poderia ser aquele criado por estatuto, que não existiria na
  ausência de registro; mas admite-se que nesse caso se deve
  eventualmente sustentar, como tem sido feito em casos similares, que a
  provisão do estatuto se torna aplicável apenas quando há publicação e
  que antes do ato de registro existe propriedade da coisa sobre a qual
  o estatuto atuará.} com o intuito de alinhá-la com os casos que
serviram como precedentes.\footnote{Duke of Queensbery \emph{versus}
  Shebbeare, 2 Eden, 329; Murray \emph{versus} Heath, I B. \& Ad. 804;
  Tuck \emph{versus} Priester, 19 Q. B. D. 629.}

Esse processo de inferir um termo em um contrato ou de pressupor uma
relação de confiança (particularmente quando se tratar de um contrato
escrito e quando não existe uso ou costume estabelecido) não é nada além
de uma declaração judicial de que a moralidade pública, a justiça
privada e a conveniência geral exigem o reconhecimento de tal regra e de
que a publicação em circunstâncias semelhantes seria considerada um
abuso intolerável. Desde que essas circunstâncias apresentem um contrato
sobre o qual tal termo possa ser incorporado pela mente judicial ou
forneçam relações sobre as quais possam ser erguidas confiança e
confidência, não haverá qualquer objeção em se desenvolver a proteção
desejada por meio das doutrinas de contrato ou confiança. Mas o tribunal
dificilmente pode parar por aí. A doutrina mais estrita pode ter
satisfeito as demandas da sociedade em uma época em que o abuso contra o
qual deveria haver proteção raramente poderia ter surgido sem violar um
contrato ou uma confiança especial; porém, agora que os dispositivos
modernos oferecem oportunidades abundantes para a perpetração de tais
injustiças sem qualquer participação da parte lesada, a proteção
concedida pela lei deve ser posta sobre uma fundação mais ampla.
Enquanto, por exemplo, o estado da arte da fotografia era tal que a foto
de alguém raramente poderia ser feita sem que a pessoa conscientemente
estivesse ``sentada'' para tal propósito, a lei do contrato ou da
confiança conseguia garantir ao homem prudente proteção suficiente
contra a circulação indevida de seu retrato; mas desde que os últimos
avanços na arte fotográfica tornaram possível se fotografar
furtivamente, as doutrinas do contrato e da confiança não são
inadequadas para sustentar a proteção necessária, e o Direito Civil deve
ser invocado. O direito de propriedade, em seu sentido mais amplo -
incluindo todas as posses, todos os direitos e privilégios e, portanto,
abraçando o direito de uma personalidade inviolável -, sozinho
proporciona essa ampla base sobre a qual a proteção que o indivíduo pode
ser assegurada.

Assim, os tribunais, buscando algum princípio que pudesse ser imposto à
publicação de cartas particulares, chegaram naturalmente às ideias de
quebra de confiança e de um contrato implícito; mas foi necessário um
pouco de atenção para perceber que essa doutrina não poderia fornecer
toda a proteção exigida, uma vez que não permitiria que o tribunal
concedesse compensação contra um estranho; e então a teoria da
propriedade do conteúdo das cartas foi adotada.\footnote{Veja-se Sr.
  Ministro Story em Folsom \emph{versus} Marsh, 2 Story, 100, III
  (1841):

  ``Se ele {[}o destinatário de uma carta{]} tentar publicar tal carta
  ou cartas em outras ocasiões não justificáveis, um tribunal de
  equidade impedirá a publicação com uma medida cautelar, por se tratar
  de uma quebra de confiança privada ou de contrato ou dos direitos do
  autor; e \emph{a fortiori}, se ele tentar publicá-la visando ao lucro;
  pois aí se tem não apenas uma quebra de confiança ou contrato, mas uma
  violação dos direitos autorais exclusivos do autor. {[}...{]} A
  propriedade em geral e os direitos, em geral, ligados à propriedade
  pertencem ao autor, sejam as cartas composições literárias, ou cartas
  familiares, ou detalhes de fatos, ou cartas de negócios. A propriedade
  em geral nos manuscritos permanece no autor e em seus representantes,
  assim como os direitos autorais em geral. \emph{A fortiori},
  terceiros, não tendo relações privadas com nenhuma das partes, não
  estão intitulados do direito de publicá-las, promovendo os seus
  propósitos privados de interesse, ou curiosidade, ou paixão.''} Na
verdade, é difícil conceber em qual teoria do direito o destinatário
casual de uma carta, que procede com a publicação da mesma, seria
culpado de uma violação de contrato, expresso ou implícito, ou de
qualquer quebra de confiança, na acepção comum desse termo. Suponha-se
que uma carta tenha sido dirigida a ele sem sua solicitação. Ele a abre
e lê. Certamente, ele não fez qualquer contrato e não aceitou nenhuma
confiança. Ele não pode, por ter aberto e lido a carta, ter qualquer
obrigação além do que a lei declara; e, mesmo expressa, essa obrigação é
simplesmente a de observar o direito legal do remetente, qualquer que
seja, e se ele é chamado de seu direito de propriedade em relação ao
conteúdo da carta ou de seu direito à privacidade.\footnote{``O
  destinatário de uma carta não é um depositário, tampouco exerce um
  papel análogo ao de um depositário. Não há direito à posse, presente
  ou futura, no autor. O único direito a ser garantido contra o detentor
  é o direito de impedir a publicação, não o de exigir o manuscrito do
  detentor para uma publicação de si mesmo.'' Per. Hon. Joel Parker,
  citado em Grigsby \emph{versus} Breckenridge, 2 Bush. 480, 489 (1857).}

Um tratamento semelhante acerca do princípio sobre o qual uma publicação
injusta pode ser proibida encontra-se na lei de segredos comerciais. No
caso, geralmente foram emitidas medidas cautelares baseadas na teoria de
quebra de contrato ou de abuso de confiança.\footnote{Em Morison
  \emph{versus} Moat, 9 Hare, 241, 255 (1851), uma ação judicial
  requerendo medida cautelar para impedir o uso de um composto químico
  secreto, Sir George James Turner, V. C., disse: ``Que o tribunal tenha
  exercido tutela em casos dessa natureza não admite, creio, nenhuma
  dúvida. Diferentes fundamentos foram, de fato, atribuídos ao exercício
  dessa tutela. Em alguns casos fez-se referência à propriedade, em
  outros, ao contrato e em outros, novamente, tratou-se o caso como
  fundamentado na boa-fé e na confiança -- o que quer dizer, conforme eu
  vejo, que a corte associa a obrigação à consciência da parte e a
  aplica da mesma maneira que a aplica contra uma parte a quem um
  benefício é dado a obrigação de cumprir uma promessa pela confiança
  com a qual o benefício fora conferido; mas quaisquer que sejam os
  fundamentos de tutela encontrados, as autoridades não deixam dúvida a
  respeito de seu exercício.''} Obviamente, raramente aconteceria de
alguém ter a posse de um segredo sem que qualquer confiança tivesse sido
colocada nele. Mas seria possível supor que o tribunal hesitaria em
conceder qualquer medida contra aquele que obteve seu conhecimento
através de uma transgressão ordinária, por exemplo, olhando
indevidamente dentro de um livro no qual o segredo foi registrado ou
ouvindo escondido? Com efeito, em Yovatt \emph{vs.} Winyard, I J.\&W 394
(1820), em que foi emitida medida cautelar contra qualquer uso ou
comunicação de certas receitas de medicina veterinária, identificou-se
que o réu, enquanto empregado do requerente, teve clandestinamente
acesso ao seu livro de receitas e o copiou. O lorde Eldon ``concedeu a
medida cautelar com base no princípio de ter havido uma quebra de
confiança e confidência'', mas pareceria difícil estabelecer qualquer
distinção legal sólida entre esse caso e outro no qual um mero estranho
tenha obtido indevidamente acesso ao livro.\footnote{Uma expansão
  similar do direito, mostrando o desenvolvimento de direitos
  contratuais em direitos de propriedade, se encontra na lei de
  patrimônios de marca. Há indicações, tão antigas quanto os \emph{Year
  Books} de negociantes tentando garantir para si mesmos, por meio de
  contratos, as vantagens hoje denominadas pelo termo ``patrimônio de
  marca'', mas foi apenas em 1743 que os patrimônios de marca receberam
  reconhecimento legal como uma propriedade separada dos acordos
  pessoais dos negociantes. Veja-se Allan sobre \emph{Goodwill}, pp. 2,
  3.}

Devemos, portanto, concluir que o direito, desse modo protegidos,
qualquer que seja sua exata natureza, não é decorrente de um contrato ou
de confiança especial, mas contra o mundo; e, como afirmado acima, o
princípio que tem sido aplicado para proteger esse direito não é na
realidade o princípio da propriedade privada, a menos que essa palavra
seja utilizada em um sentido ampliado e incomum. O princípio que protege
escritos pessoais e quaisquer outras produções do intelecto ou das
emoções é o direito à privacidade e a lei não tem nenhum princípio novo
para formular quando estende essa proteção à aparência pessoal, falas,
atos e relações pessoais, domésticas ou de outro tipo.\footnote{A
  aplicação de um princípio existente a um novo estado de coisas não é
  legislação judicial. Chamá-la assim é afirmar que um marco legal
  existente consiste praticamente de estatutos e casos decididos e negar
  que os princípios (dos quais tais casos são normalmente considerados
  evidências) existam de todo. Não se trata da aplicação de um princípio
  existente a novos casos, mas a introdução de um novo princípio, que é
  propriamente denominada legislação judicial.

  Mas mesmo o fato de certa decisão poder envolver legislação judicial
  não deveria ser tomado como conclusivo contra a propriedade de
  fazê-lo. Tal poder tem sido constantemente exercido por nossos juízes
  quando aplicam a um novo objeto princípios de justiça privada,
  adequação moral e conveniência pública. De fato, a elasticidade do
  nosso direito, sua adaptabilidade a novas condições, sua capacidade de
  crescimento --- que lhe permitiu ir ao encontro das necessidades de
  uma sociedade em constante transformação, e aplicar reparações
  imediatas para todo mal reconhecido ---, têm sido seu principal motivo
  de orgulho.

  ``Não consigo entender como alguém, tendo refletido sobre o assunto,
  possa supor que a sociedade poderia ter possivelmente avançado caso os
  juízes não tivessem legislado, ou que haja algum perigo de qualquer
  natureza em lhes conceder esse poder que eles, de fato, têm exercido
  para compensar a negligência ou a incapacidade do legislador assim
  declarado. A parte das leis de todo país que foi feita por juízes tem
  sido muito melhor do que a parte composta pelos estatutos aprovadas
  pela legislatura.'' \emph{I Austin's Jurisprudence}, p. 224.

  Os casos mencionados acima mostram que o Direito Comum vem protegendo
  há um século e meio a privacidade em certos casos e que garantir a
  proteção futura agora sugerida seria meramente uma outra aplicação de
  uma regra existente.}

Se a invasão de privacidade constitui uma \emph{injuria} legal, os
elementos para se exigir reparação existem, uma vez que o valor do
sofrimento mental, causado por um ato abusivo em si mesmo, é reconhecido
como uma base para compensação.

O direito de alguém, que se mantém como indivíduo privado, de evitar a
publicação de seu retrato representa o caso mais simples para tal
extensão; o direito de se proteger de um retrato ou de um debate, pela
imprensa, de seus assuntos privados seria mais importante e de maior
alcance. Se até declarações casuais e sem importância em uma carta, se
um trabalho, mesmo sem natureza artística e sem valor, se as posses de
todos os tipos são protegidas não apenas contra a reprodução, mas também
contra a descrição e a enumeração, quanto mais os atos e falas de um
homem em suas relações sociais e domésticas deveriam ser protegidos de
uma publicidade implacável. Se não se pode reproduzir o rosto de uma
mulher fotograficamente sem seu consentimento, também não deveria ser
tolerada a reprodução de seu rosto, sua forma e as suas ações por
descrições gráficas coloridas para servir à imaginação nojenta e
depravada.

O direito à privacidade, limitado como tal direito deve necessariamente
ser, já encontrou expressão na lei da França.\footnote{Loi Relative a la
  Presse. II Mai I868.

  "II. Toute publication dans un écrit periodique relative à un fait de
  la vie priveé constitue une contravention punie d'un amende de cinq
  cent francs.

  La poursuite ne pourra être exercée que sur la plainte de la partie
  interessée."

  Riviére, Codes Francais et Lois Usuelles. App. Code Pen., p. 20.}

Resta considerar quais são as limitações desse direito à privacidade e
que reparações podem ser garantidas para a aplicação do direito.
Determinar, antecedendo-se à experiência, a linha exata a partir da qual
a dignidade e a conveniência do indivíduo devem ceder às exigências do
bem-estar público ou da justiça privada seria uma tarefa difícil; mas as
regras mais gerais são fornecidas pelas analogias legais já
desenvolvidas na lei de calúnia e difamação e na lei da propriedade
literária e artística.

\section{1. O direito à privacidade não proíbe qualquer publicação de
  questões de interesse público ou geral.}

Para determinar o âmbito de aplicação dessa regra, ter-se-ia a ajuda da
analogia, na lei de calúnia e difamação, com casos que tratam o
privilégio qualificado de comentário e crítica sobre as questões de
interesse público e geral.\footnote{Veja-se Campbell \emph{versus}
  Spottiswoode, 3 B. \& S. 769, 776; Henwood \emph{versus} Harrison, L.
  R. C. P. 606; Gott \emph{versus} Pulsifer, 122 Mass. 235.} Há,
naturalmente, dificuldades em se aplicar tal regra; mas elas são
inerentes a esse assunto e certamente não são maiores do que as que
existem em muitos outros ramos do direito --- por exemplo, em um grande
número de casos em que a razoabilidade ou a irracionalidade de um ato é
usada como teste de responsabilidade. O desenho da lei deve proteger
aquelas pessoas sobre as quais seus assuntos não têm nenhum interesse
legítimo para a comunidade de serem arrastadas para uma publicidade
indesejada e indesejável e proteger todas as pessoas, de qualquer tipo;
em sua posição de ter assuntos que elas podem devidamente preferir
manter privados, tornados públicos contra a sua vontade. É a invasão
injustificada da privacidade de um indivíduo que é repreensível e deve
ser evitada tanto quanto possível. No entanto, a distinção observada na
declaração acima é óbvia e fundamental. Existem pessoas que
razoavelmente podem reivindicar como direito a proteção contra a
notoriedade gerada por terem sido vítimas da imprensa. Há outras que, em
graus variados, renunciaram ao direito de viver suas vidas protegidas da
observação pública. Questões que as pessoas do primeiro tipo podem
justamente afirmar que só dizem respeito a elas mesmas podem para as do
segundo, estarem sujeitas a um interesse legítimo de seus concidadãos.
Peculiaridades da pessoa e de seu modo de ser que, no indivíduo normal,
devem estar livres de comentários, podem adquirir importância pública no
caso de um candidato a um cargo político. É necessária uma discriminação
melhor, portanto, do que classificar fatos ou atos como públicos ou
privados de acordo com um padrão a ser aplicado ao fato ou à ação
\emph{per se}. Publicar algo sobre um indivíduo modesto e recatado que
sofre de algum tipo de deficiência na fala ou que não sabe pronunciar
corretamente é uma violação injustificada e sem precedentes dos seus
direitos, enquanto afirmar e comentar sobre as mesmas características
encontradas em um potencial congressista não poderia ser considerado
como algo fora dos limites.

O objeto geral em perspectiva é proteger a privacidade da vida privada e
qualquer que seja o grau e a conexão em que a vida de um homem tenha
deixado de ser privada antes de que a publicação sob consideração tenha
sido feita, é nessa medida que a proteção está suscetível de ser
revogada.\footnote{``Nos moeurs n'admettent pas la prétention d'enlever
  aux investigations de la publicité les actes qui relèvent de la vie
  publique, et ce dernier mot ne doit pas être restreint à la vie
  officielle ou à celle du fonctionnaire. Tout homme qui appelle sur lui
  l'attention ou les regards du publique, soit par une mission qu'il a
  reçue ou qu'il se donne, soit par le rôle qu'il s'attribue dans
  l'industrie, les arts, le theâtre, etc., ne peut plus invoquer contre
  la critique ou l'exposé de sa conduite d'autre protection que leslois
  qui repriment la diffamation et l'injure. " Circ. Mins. Just., 4 Juin,
  I868. Riviére Codes Français et Lois Usuelles, App. Code Pen. 20 n
  (b).} Visto que, então, a propriedade de se publicar os mesmos fatos
pode depender inteiramente da pessoa sobre quem eles são publicados,
nenhuma fórmula fixa pode ser usada para proibir publicações
detestáveis. Qualquer regra de responsabilidade adotada deve ter em si
uma elasticidade que lhe permitirá considerar as diferentes
circunstâncias de cada caso --- uma necessidade que infelizmente torna
tal doutrina não só de mais difícil aplicação, mas também, em certa
medida, incerta em sua operação e facilmente frustrada. Além disso, são
apenas as violações mais flagrantes de decência e decoro que podem na
prática ser atingidas e talvez não seja desejável nem sequer tentar
reprimir tudo o que condenariam o melhor dos gostos e o mais agudo senso
do respeito devido à vida privada.

Em geral, então, as matérias pelas quais a publicação deveria ser
reprimida podem ser descritas como aquelas que dizem respeito à vida
privada, aos hábitos, atos e relações de um indivíduo, não tendo nenhuma
conexão legítima com sua adequação a um cargo público, uma posição
pública ou quase pública, que ele almeje ou para o qual ele fora
indicado, e nenhuma relação legítima ou nenhum efeito sobre qualquer ato
realizado por ele em uma função pública ou quase pública. O que precede
não é concebido como uma definição inteiramente exata ou aprofundada,
uma vez que algo que deve, ao fim e ao cabo, em um grande número de
casos tornar-se uma questão de julgamento e opinião individual é incapaz
de tal definição; porém, é uma tentativa de indicar amplamente a classe
das questões referidas. Algumas coisas todos os homens igualmente têm o
direito de preservar da curiosidade popular, seja na vida pública ou
não, enquanto outras são privadas apenas porque as pessoas em questão
não assumiram uma posição que faça dos seus atos objeto de investigação
pública.\footnote{``Celui-la seul a droit au silence absolu qui n'a pas
  expressément ou indirectment provoqué ou authorisé l'attention,
  l'approbation ou le blâme.'' Circ. Mins. Just., 4 de junho, 1868.
  Rivière, \emph{Codes Français et Lois Usuelles,} App. Code Pen. 20
  n(b).

  O princípio assim expresso evidentemente se destina à exclusão de toda
  e qualquer investigação acerca do passado de proeminentes homens
  públicos com os quais o público americano está bastante familiarizado
  e também, infelizmente, demasiadamente contente; não estando aptos ao
  ``silêncio \emph{absolu'',} que homens menos proeminentes podem
  reivindicar como prerrogativa sua, eles ainda podem exigir que todos
  os detalhes de sua vida privada, em seu sentido mais limitado, não
  sejam expostos numa inspeção.}

\section{2. O direito à privacidade não proíbe a comunicação de qualquer
assunto, a não ser em sua natureza privada, quando a publicação é feita
em circunstâncias que a tornam uma comunicação privilegiada de acordo
com a lei de calúnia e difamação.}

Sob essa regra, o direito à privacidade não é invadido por qualquer
publicação feita em um tribunal de justiça, nos órgãos legislativos ou
em comissões desses organismos; em assembleias municipais, ou comissões
de tais assembleias, ou praticamente qualquer comunicação em qualquer
outra entidade pública, municipal ou paroquial, ou em qualquer corpo
quase público, como em grande parte das associações voluntárias formadas
para quase todos os fins de benevolência, negócios ou outros de
interesse geral; e (pelo menos em muitas jurisdições) relatórios de
quaisquer desses procedimentos teriam de, em alguma medida, ser
atribuídos um privilégio similar.\footnote{Wason \emph{versus} Walters,
  L. R. 4 Q. B. 73; Smith \emph{versus} Higgins, 16 Gray, 251; Barrows
  \emph{versus} Bell, 7 Gray, 331.} Tampouco a regra proíbe qualquer
publicação feita por alguém afastado de um cargo público ou privado,
seja legal ou moral, ou na condução dos seus próprios assuntos, em
matéria na qual seu próprio interesse é envolvido.\footnote{Essa
  limitação ao direito de se impedir a publicação de cartas privadas já
  foi reconhecida anteriormente:

  ``Mas, de maneira consistente com esse direito {[}do autor de
  cartas{]}, as pessoas a quem elas se destinam podem ter, ou melhor,
  devem, por implicação, possuir o direito de publicar qualquer carta ou
  cartas enviadas a elas diante de situações que requeiram ou
  justifiquem a sua publicação ou uso público; mas esse direito se
  encontra estritamente limitado a tais ocasiões. Desse modo, uma pessoa
  pode justificadamente usar e publicar, num processo legal ou no
  tribual de equidade, tal carta ou tais cartas como for necessário ou
  apropriado, de modo a estabelecer o seu direito de manter a ação ou de
  modo a se defender. Então, se ele tiver sido caluniado ou difamado
  pelo autor, ou acusado de conduta imprópria, de maneira pública, ele
  pode publicar tais partes da carta ou das cartas, mas nada além do
  necessário para defender o seu caráter e reputação ou se livrar do
  descrédito e reprovação injustos.'' Story, J., em Folsom \emph{versus}
  Marsh, 2 Story, 100, 110, 111 (1841).

  A existência de qualquer direito, no destinatário de cartas, de
  publicar as mesmas foi vigorosamente negada pelo Sr. Drone; mas o
  raciocínio sobre o qual sua negação se sustenta não parece
  satisfatório. Drone sobre \emph{Copyrights}, pp. 136-139.}

\section{3. A lei provavelmente não concederá qualquer reparação para a
invasão da privacidade por meio de publicação oral na ausência de dano
especial.}

As mesmas razões existem para distinguir publicações orais e escritas de
assuntos privados, conforme garantido na lei de difamação pela
responsabilização restrita por difamação em comparação com a
responsabilização por calúnia.\footnote{Townshend sobre \emph{Slander
  and Libel}, 4ª ed., § 18; Odgers sobre \emph{Libel and Slander}, 2ª
  ed., p. 3.} O prejuízo resultante de tais comunicações orais seria
normalmente tão insignificante que poderia muito bem a lei, no interesse
da liberdade de expressão, ignorá-la completamente.\footnote{``Mas
  enquanto a fofoca fosse oral, ela se espalhava, no que diz respeito a
  qualquer pessoa individual, por uma área bastante pequena e estava
  confinada ao círculo imediato de suas relações. Ela não alcançava, ou
  raramente alcançava, pessoas que nada sabiam a seu respeito. Ela não
  tornava seu nome, ou seu andar, ou suas conversas familiares a
  estranhos. E, o que vem ainda mais a calhar, ela o poupava da dor e da
  mortificação de se saber alvo de fofocas. Um homem raramente ouvia
  falar de uma fofoca oral a seu respeito que simplesmente o tornava
  ridículo ou violava a sua privacidade legítima, mas não fazia qualquer
  ataque positivo à sua reputação. Sua paz e seu conforto eram,
  portanto, só levemente afetados por ela.'' E. L. Godkin, ``\emph{The
  Rights of the Citizen: To his Reputation.''} Scribner's Magazine,
  julho, 1890, p. 66.

  O vice-chanceler Cavaleiro Bruce sugeriu em Príncipe Albert
  \emph{versus Estranho}, 2 DeGex \& Sm. 652, 694, que uma distinção
  fosse feita, no que diz respeito ao direito à privacidade em obras de
  arte, entre descrições ou catálogos orais e escritos.}

4. \section{O direito à privacidade cessa após a publicação dos fatos
pelo indivíduo, ou com o seu consentimento.}

Isso não é nada além de outra aplicação da regra que se tornou familiar
com a lei da propriedade literária e artística. Os casos assim decididos
estabeleceram também o que deve ser considerado uma publicação --- sendo
o princípio importante nessa conexão o de que uma comunicação privada de
circulação com finalidade restrita não é uma publicação na acepção da
lei.\footnote{Veja-se Drone sobre \emph{Copyrights}, pp. 121, 289, 290.}

5. \section{A verdade sobre o assunto publicado não fornece defesa}

Obviamente, esse ramo do direito não deve ter nenhuma preocupação com a
verdade ou a falsidade das matérias publicadas. Não é por injúria ao
caráter do indivíduo que é pedida a reparação ou prevenção, mas pelo
prejuízo ao direito à privacidade. Para a primeira, a lei de calúnia e
difamação talvez forneça garantia suficiente. A última implica o direito
de não apenas evitar um retrato impreciso da vida privada, mas evitar
que ela seja retratada de todo.\footnote{Compare-se com o direito
  francês:

  ``En prohibant l'envahissement de la vie privée, sans qu'il soit
  nêcessaire d'établir 'intention criminelle, la loi a entendue
  interdire toute discussion de la part de la défense sur la vêrité des
  faits. Le remède eut été pire que le mal, si un débat avait pu
  s'engager sur ce terrain'' Circ. Mins. Just., 4 de junho, 1868.
  Rivière, \emph{Codes Français et Lois Usuelles,} App. Code Penn. 20
  n(a).}

6. \section{A ausência de ``malícia'' no divulgador não permite defesa.}

Má vontade pessoal não é um ingrediente da ofensa, mais do que em um
caso comum de transgressão contra a pessoa ou a propriedade. Nunca é
necessário que tal malícia seja demonstrada em uma ação por difamação ou
calúnia no Direito Comum, exceto na refutação de uma defesa, por
exemplo, de que a ocasião tornou a comunicação privilegiada, ou,
conforme os estatutos deste Estado e alhures, de que a declaração em
questão era verdadeira. A invasão da privacidade que deve ser protegida
é igualmente completa e igualmente prejudicial, sejam os motivos pelos
quais o que falou ou escreveu atuou considerados por si mesmos culpáveis
ou não; assim como o dano à índole e, em certa medida, a tendência para
provocar uma violação da paz são igualmente resultado de difamação sem
se considerarem os motivos que levaram a sua publicação. Vistos como um
mal ao indivíduo, essa regra é a mesma que permeia toda a lei de
delitos, pela qual alguém é responsabilizado por seus atos intencionais,
mesmo que eles tenham sido cometidos sem intenção sinistra; e vistas
como um mal para a sociedade, esse é o mesmo princípio adotado em uma
grande categoria de infrações legais.

Os reparos para uma invasão do direito à privacidade também são
sugeridos por aqueles administrados na lei de difamação e na lei da
propriedade literária e artística, nomeadamente:

a) Uma ação de responsabilidade por danos em todos os casos.\footnote{Comparar
  Drone sobre \emph{Copyrights}, p. 107.} Mesmo na ausência de danos
especiais, uma compensação substancial pode ser permitida pela lesão aos
sentimentos como na ação de calúnia e difamação.

b) Uma medida cautelar, talvez em uma categoria muito limitada de
casos.\footnote{Comparar High sobre \emph{Injunctions}, 3ª ed., §1015;
  Townshend sobre \emph{Libel and Slander}, 4ª ed., §§ 417a -- 417d.}

Sem dúvida seria desejável que a privacidade do indivíduo recebesse
proteção adicional do Direito Penal, mas para isso seria necessária uma
legislação.\footnote{O seguinte anteprojeto de lei foi preparado por
  William H. Dunbar, Esq., da \emph{Boston Bar}, como sugestão para uma
  possível legislação:

  SEÇÃO 1. Quem quer que publique, em qualquer diário, jornal, revista
  ou outra publicação periódica, qualquer afirmação concernente à vida
  ou assuntos privados de outra pessa, tendo sido solicitado por escrito
  por tal pessoa que não publicasse tal afirmação ou qualquer afirmação
  a seu respeito, será punido com detenção de não mais do que cinco anos
  em prisão estadual, ou em cadeia por no máximo dois anos, ou com multa
  não superior a mil dólares; tendo-se em vista que nenhuma afirmação
  referente à conduta de qualquer pessoa que ocupe, ou às qualificações
  de qualquer pessoa que postule, a função ou posição pública que tal
  pessoa possua, tenha possuído ou busque obter, ou à qual tal pessoa
  seja, quando da publicação, candidata, ou para a qual ela tenha sido
  então sugerida como candidata, e nenhuma afirmação acerca de ou
  referente aos atos de qualquer pessoa em seu negócio, profissão, ou
  vocação, e nenhuma afirmação que se refira a qualquer pessoa no que
  concerne a sua posição, profissão, negócio ou vocação, que traz tal
  pessoa de modo proeminente à público, ou em relação às qualificações
  para tal posição, negócio, profissão, ou vocação de qualquer pessoa
  proeminente ou buscando proeminência diante do público, e nenhuma
  afirmação que seja relativa a qualquer ato realizado por qualquer
  pessoa em local público, tampouco qualquer outra afirmação sobre
  assunto de interesse público e geral, será considerada uma afirmação
  concernente à vida ou assuntos privados de tal pessoa dentro dos
  termos dessa lei.

  ``SEÇÃO 2. Não constituirá defesa para nenhuma ação criminal
  decorrente da Seção 1 desta lei que a afirmação sob queixa seja
  verdadeira ou que tal afirmação tenha sido publicada sem intenção
  maliciosa; mas ninguém estará sujeito a punição por qualquer afirmação
  publicada sob circunstâncias tais que a publicação daquilo viesse a
  ser privilegiada caso fosse de fato difamatória.} Talvez fosse tido
como adequado trazer a responsabilidade penal para tal publicação dentro
de limites mais estreitos; mas não podem haver dúvidas de que a
comunidade tem interesse em impedir tais invasões de privacidade, um
interesse suficientemente forte para justificar a introdução de tal
solução. Ainda assim, a proteção da sociedade deve se dar principalmente
através de um reconhecimento dos direitos do indivíduo. Cada homem é
responsável apenas por seus próprios atos e omissões. Se ele tolera o
que ele reprova, com uma arma equivalente a sua defesa, ele é
responsável pelos resultados. Se ele resistir, a opinião pública se
juntará em seu apoio. Tem ele, portanto, tal arma? Acredita-se que o
Direito Comum fornece-lhe uma, forjada no fogo lento dos séculos e hoje
bem ajustada e adequada a sua mão. O Direito Comum sempre reconheceu a
casa de um homem como o seu castelo, insuperável, muitas vezes, até
mesmo para seus próprios oficiais envolvidos na execução de seu comando.
Devem os tribunais assim fechar a entrada da frente às autoridades
constituídas e abrir a porta traseira para a curiosidade ociosa ou
lasciva?

Boston, dezembro de 1890

\chapter{O Desenvolvimento geracional da proteção de dados na
Europa\footnote{Texto originalmente publicado em
  AGRE, P. E., ROTENBERG, M. (org). \emph{Technology and Privacy: The
  NewLandscape.} Cambridge, MA:The MIT Press, 1997, pp. 219-242.}}\label{o-desenvolvimento-geracional-da-proteuxe7uxe3o-de-dados-na-europa}

Viktor Mayer-Schönberger

Na Europa, desde a década de 1970, ``proteção de dados'' tornou-se uma
expressão comum utilizada para descrever o direito das pessoas
controlarem seus próprios dados. Atualmente, as normas de proteção de
dados são questões estabelecidas e bem aceitas nos ordenamentos
jurídicos das nações europeias. No entanto, as conotações associadas ao
termo ``proteção de dados'' sofreram mudanças repetidas e substanciais,
de modo que definir o termo acabou por ser uma questão inútil, senão
tautológica.\footnote{SASSE, C. \emph{Sinn und Unsinn des
  Datenschutzes}. Leipzig: Karlsruhe, Müller, 1976, p. 78.} Além disso,
o próprio termo é, como muitos notaram, mal escolhido. Não são os
``dados'' que necessitam de proteção, mas sim o indivíduo a quem os
dados dizem respeito.\footnote{O primeiro comissário de dados na
  Alemanha, Spiros Simitis, observou esse fato sobretudo na primeira
  edição de seu comentário sobre a Lei Federal de Proteção de Dados
  alemã: SIMITIS, S. \emph{Kommentar zum BDSG}. Baden-Baden: Nomos,
  1979, p. 53 e os correspondentes anexos.}

Por outro lado, abandonar completamente o termo já tão bem difundido da
``proteção de dados'' também não é a melhor solução. Muita coisa seria
perdida: uma imagem venerada, uma metáfora já tão bem estabelecida que
tem funcionado no discurso público. Por isso, a Europa tem aderido a tal
terminologia, apesar de reconhecer que o que este termo está lidando é e
continuará a ser um alvo em constante movimentação. Ainda assim, para
compreender a concepção europeia de ``proteção de dados'' é aconselhável
deixar de lado as definições teóricas e usar o termo com um espírito
mais dinâmico e evolutivo.

Nesse sentido, leis de proteção de dados foram promulgadas em grande
parte dos países europeus desde 1970.\footnote{O estado alemão de Hesse
  promulgou a primeira lei de proteção de dados do mundo em 1970. A
  Suécia o seguiu com o primeiro estatuto nacional de proteção de dados
  em 1973. A Alemanha promulgou uma lei de proteção de dados, a
  \emph{Bundesdatenschutzgesetz} (BDSG), em 1977. Em 1978, França,
  Áustria, Noruega e Dinamarca aprovaram suas normas de proteção de
  dados. O Reino Unido seguiu o exemplo em 1984, a Finlândia em 1987 e a
  Suíça em 1992. Após a queda da Cortina de Ferro, as leis de proteção
  de dados constavam entre as primeiras normas a serem promulgadas em
  países do Leste Europeu.} Isso não significa apenas a conscientização
tanto dos políticos quanto do público para o problema da privacidade
informacional; elas também evidenciam as dramáticas mudanças
tecnológicas no processamento de informações.

Muita tinta tem sido gasta sobre o problema da proteção de dados e como
ele deve ser encarado.\footnote{Ver, por exemplo, FLAHERTY.
  \emph{Privacy and Data Protection}: An International Bibliography.
  Boston: G. K. Hall, 1984, e as informações bibliográficas do projeto
  de lei de proteção de dados (impressas em Mayer-Schönberger,
  \emph{Recht der Information,} Böhlau, 1997).} Na Europa, a maior parte
do trabalho feito tem focado nas leis de proteção de dados e em suas
implicações específicas em cada um dos países europeus. Estudos
comparativos são usados principalmente para oferecer subsídios às
reformas legislativas dessas normas nacionais de proteção de dados. A
relativa pequena quantidade de trabalhos internacionais provêm da OECD
\emph{---} Organization for Economic Cooperation and Development
(Organização para a Cooperação Econômica e Desenvolvimento) ---, que
inovou internacionalmente em 1981 ao publicar suas \emph{Guidelines on
the Protection of Privacy and Transborder Flows of Personal Data}
{[}Diretrizes sobre a proteção da privacidade e de fluxos transnacionais
de dados pessoais{]},\footnote{OCDE. \emph{Guidelines on the Protection
  of Privacy and Transborder Flows of Personal Data}. 1981.} do Conselho
da Europa e, recentemente, da União Europeia. Na Europa, quase todas as
normas nacionais promulgadas após 1981 refletem o espírito, senão o
próprio texto, das diretrizes da OECD.\footnote{As orientações e as suas
  temidas repercussões econômicas tiveram forte influência, ao forçar a
  Grã-Bretanha a adotar uma lei de proteção de dados em 1984.} É
difícil, no entanto, calcular com precisão a sua influência real e
genuína . Afinal, os princípios nela incorporados foram destilados a
partir de normas de proteção de dados promulgadas na Europa.

No início de 1980, a \emph{European Convention on Data Protection}
{[}Convenção Europeia sobre a Proteção de Dados{]} foi assinada pela
grande maioria dos países-membros do Conselho da Europa, mas teve pouco
impacto prático nas discussões.\footnote{CONSELHO DA EUROPA. Convenção
  para a proteção das pessoas em relação ao tratamento automatizado de
  dados pessoais. E.T.S. n. 108, 1981, 19 I.L.M. S71, 1981. Essa
  convenção entrou em vigor em 1985, depois de França, Alemanha,
  Noruega, Espanha e Suécia a ratificarem.} Tardiamente, em 1995, a
União Europeia, após quase uma década de discussão, aprovou a legislação
que obriga seus países-membros a internalizarem o texto da diretiva e a
legislarem suas próprias leis de proteção de dados.\footnote{Diretriz
  95/46/EG relativa à proteção das pessoas em relação ao tratamento de
  dados pessoais e à livre circulação de tal dados, OJ, 23 nov. 1995, L
  281/31.}

Assim, a proteção de dados tem sido vista em grande parte como uma
questão nacional que incorpora características locais únicas de
privacidade e de autodeterminação individual. Até recentemente, a
maioria dos estudos comparativos das normas de proteção de dados
pessoais enfatizou dessemelhanças transnacionais e aventurou-se a
explicar essas variações como evidências de diferentes percepções de
países específicos sobre a questão da proteção de dados. Mais
recentemente, essa perspectiva nacional está sendo cada vez mais
contestada e se encontra sob uma crescente pressão de uma grande
quantidade de demandas internacionais pelo livre fluxo informacional
transfronteiriço, e do resultante desejo por um regime europeu de
proteção de dados mais homogêneo.

Além disso, um recente estudo preliminar, focando em vários aspectos da
privacidade e da proteção de dados e da interface entre esses direitos,
falha em confirmar supostas diferenças entre nações tão diversas como os
Estados Unidos, a Tailândia, a Dinamarca, o Reino Unido e a
França.\footnote{MILBERG; BURKE; SMITH; KALLMAN. \emph{Values, Privacy,
  Personal Information and Regulatory Approaches}. Communications of the
  ACM 58, n. 12, 1995, p. 65-74.} Se diferenças significativas foram
mesmo detectadas, elas tinham mais a ver com aspectos singulares de
privacidade e proteção de dados dentro de próprio país do que com
aspectos similares através dos países.\footnote{A pesquisa analisou
  quatro aspectos da proteção de dados: o problema da coleção (``muito é
  armazenado''), uso secundário, dados errôneos e acesso indevido aos
  dados. Em uma escala de sete pontos, essas cinco nações variaram entre
  0,3 e 0,8 em seu julgamento sobre a importância de qualquer um desses
  quatro aspectos, com as três nações europeias mostrando uma
  homogeneidade ainda maior (os desvios máximos foram 0.4, 0.6, 0.1 e
  0.3, respectivamente). Do mesmo modo, as diferenças através dos quatro
  aspectos dentro de uma nação específica foram substancialmente mais
  elevadas (entre 0,7 e 1,3). Ibid., p. 70.}

Os resultados empíricos do estudo acima mencionado só reforçam a
importante demonstração de Colin Bennett\footnote{BENNETT.
  \emph{Regulating Privacy}: Data Protection and Public Policy in Europe
  and the United States. 1992.} de que a proteção de dados, acima e além
de idiossincrasias locais, pode ser vista como um processo internacional
coordenado informalmente. Nele, as nações podem estar em diferentes
fases do desenvolvimento legislativo, mas não podem resistir a uma
tendência evolutiva geral comum das normas de proteção de dados
(especialmente na Europa).

Se de fato for o caso das leisde proteção de dados estarem se
desenvolvendo em um contexto internacional informal, ainda que vagamente
coordenado, nós deveríamos parar de examinar as diferenças entre as suas
diversas estruturas. Pode ser mais útil, particularmente na Europa, um
olhar sobre as normas de proteção de dados, não analisando as distinções
locais, mas agrupando semelhanças entre os diversos regimes de proteção
de dados. O desenvolvimento das leis de proteção de dados entre os
países europeus pode se tornar mais visível e compreensível, se visto
como um processo contínuo em que, ao longo do tempo, alguns modelos
superficiais de proteção de dados surgem, prevalecem e, por fim,
minguam.

Este capítulo tentará descrever o desenvolvimento europeu nas normas de
proteção de dados em termos geracionais. Através desse modelo
geracional, temas recorrentes no debate sobre a proteção de dados podem,
espera-se, ser melhor compreendidos. Com certeza, uma abordagem
geracional não pode processar resultados exatos. Comparar as normas dos
diferentes sistemas jurídicos sempre requer alguma simplificação. Mas a
exatidão absoluta não é necessária. Qualquer tensão causada pela
categorização de normas em grupos distintos tende a promover apenas uma
análise holística.

\section{A primeira geração de normas de proteção de dados}

As primeiras leis de proteção de dados foram promulgadas em resposta ao
surgimento do processamento eletrônico de dados pelo governo e pelas
grandes corporações. Eles representam tentativas de se antecipar às
visões sombrias de uma inevitável aproximação ao romance \emph{Admirável
mundo novo,} corporificado pelos planos, discutidos na década de 1960 e
início da década de 1970, de centralizar todos as bases de dados
pessoais em gigantescos bancos de dados nacionais.\footnote{BENNETT.
  \emph{Regulating Privacy}, p. 45-53.}

A Lei de Proteção de Dados do estado alemão de Hesse (1970),\footnote{Hessisches
  Datenschutzgesetz vom 7.10.1970, GVBl, 1970 I, p. 625.} a sueca
(1973),\footnote{Lei de Dados sueca, datada de 11 de maio de 1973. Essa
  lei teve sua última emenda em 1994. Para um texto completo com a
  tradução do sueco e outras leis europeias de proteção de dados,
  consulte-se SIMITIS et al. (Ed.). \emph{Data Protection in the
  European Community} , 1992, com atualizações.} a do estado alemão de
Rheinland-Pfalz (1974),\footnote{Lei contra o Uso Indevido de Dados de
  24 de janeiro de 1974, GVBl. 31.} várias propostas para uma Lei alemã
Federal de Proteção de Dados,\footnote{Proposta de uma Lei para Proteger
  Dados Pessoais contra Uso Indevido Durante o Processamento de Dados,
  21 de setembro de 1973, BT-DRS 7/1027; compare-se a proposta de
  Adalbert Podlech para um estatuto com muito mais alcance em seu livro
  \emph{Datenschutz im Bereich der öffentlichen Verwaltung}. 1973. E o
  parecer sobre a proposta pela STEINMÜLLER et al. \emph{Grundfragen des
  Datenschutzes}. BT-DRS 6/3826, p. 5.} as propostas austríacas
(1974)\footnote{Proposta do Governo de uma Lei Federal para Proteger os
  Dados Pessoais, nº 1423 d. Sten.Prot. 13. GP NR.} e, finalmente, a
promulgada Lei alemã Federal de Proteção de Dados (1977)\footnote{O nome
  completo do \emph{Bundesdatenschutzgesetz} pode ser traduzido como
  ``Estatuto para proteger os dados pessoais contra uso indevido durante
  o processamento de dados.'' A BDSG foi promulgada em 27 de janeiro de
  1977. Ver BGBl. I, 1978, p. 201.} podem ser vistas como reações
diretas aos planejados e referidos bancos de dados nacionais
centralizados. Na estrutura, linguagem e abordagem, elas representam a
primeira geração de normas de proteção de dados.

Para analisar com precisão essas normas deve-se entender as intrusivas
mudanças organizacionais que estavam ocorrendo naquela época. O
computador, originalmente concebido para controlar mísseis e quebrar
códigos secretos, apareceu na hora certa para a burocracia estatal. As
nações europeias tinham acabado de iniciar reformas sociais enormes e
ampliar seu estado do bem estar-social. Riscos e deveres do cidadão
individual foram assegurados pelo Estado e, por conseguinte, cada vez
mais socializados. Essa mudança de papeis e atribuições exigiu um
sofisticado sistema de planejamento por parte do governo, e planejamento
requer dados. Assim, a burocracia governamental tinha que constantemente
coletar cada vez mais informação sobre os cidadãos para cumprir
adequadamente suas tarefas e planejar eficientemente o futuro. No
entanto, a coleta de dados por si não é suficiente. Dados devem ser
processados e correlacionados entre si para criar os necessários
instrumentos de planejamento, de modo que essa complexa engenharia
social possa ser aplicada às necessidades individuais dos cidadãos. Por
isso, nos Estados modernos de bem-estar social, o processamento de dados
mostra-se necessário em duas frentes: de baixo para cima: para criar as
informações agregadas de planejamento, a partir de milhões de dados
pessoais e; de cima para baixo: para transformar os complexos
regulamentos sociais em direitos individuais concretos.

Sem a computação, a feição moderna do estado bem estar social não
poderia funcionar. Isso explica o porquê da euforia velada da burocracia
pela nova tecnologia. Mas, não só os governos imediatamente
compreenderam os benefícios do computador. Grandes corporações também
poderiam melhor planejar, administrar e gerir suas empresas. Isso criou
um solo fértil para propostas de diversos países de centralizar todas as
informações em único banco de dados. Na Suécia, registros e dados de
recenseamento já haviam sido agregados, e dados fiscais foram
armazenados em bancos de dados centrais de impostos . Na década de 1960
o legislativo sueco propôs que se reunissem todas essas fontes de
informações em um único banco nacional de informações.\footnote{BENNETT.
  \emph{Regulating Privacy}, p. 47; FLAHERTY. \emph{Privacy and
  Government Data Banks}, p. 105.}

Planos semelhantes existiram na Alemanha. O estado da Bavaria queria
usar os computadores instalados para os Jogos Olímpicos em 1972 para
formar um sistema centralizado de informações.\footnote{STEINMÜLLER.
  \emph{Rechtsfragen der Verwaltungsautomation} \emph{in Bayern}.
  Relatório de dados 6/1971, p. 24.} Em 1970, o estado hessiano sugeriu
a utilização do uso do processamento de informações centralizadas para a
administração estatal.\footnote{Hessische Zentrale für
  Datenverarbeitung, Entwicklungsprogramm für den Ausbau der
  Datenverarbeitung. In: HESSEN. \emph{Grosser Hessenplan} (Grande plano
  de Hesse), 1970.} No nível federal, a Alemanha criou um comitê
especial de coordenação para vincular os então planejados bancos de
dados municipais, estaduais e federais em um único sistema
abrangente.\footnote{WALTER et al. \emph{Informationssysteme in
  Wirtschaft und Verwaltung}, 1971.}

Houve resistência contra propostas tão monstruosas. O medo dos cidadãos
de uma burocracia automatizada e em grande parte desumanizada produziu
uma força catalizadora, unificando fronteiras e movimentos de proteção
de dados. A tecnologia estava no centro dessas críticas, e o medo da
vigilância total de um \emph{Big Brother} eletrônico a alimentava. Essa
movimentação em direção à formação de bancos de dados centralizados
repercutiu na estrutura, na linguagem e no enfoque da primeira geração
das leis de proteção de dados.

A maioria das normas de proteção de dados da primeira geração não foca
na proteção direta da privacidade individual. Em vez disso, concentra-se
em qual é a função do processamento de dados para a sociedade. De acordo
com essa análise, o uso de computadores por si só põe em perigo a
proteção dos dados pessoais,\footnote{STEINMÜLLER. \emph{Grundfragen}.
  In: STADLER et al. \emph{Datenschutz}. 1975; SIMITIS.
  \emph{Kommentar}, p. 51. Ver também 1º parágrafo 2 BDSG, proposta do
  governo austríaco, § 1, 1974.} sendo, por isso, uma ferramenta
projetada especificamente para combater esse tipo de perigo. O
computador parece ser o problema, de modo que a sua utilização deve ser
regulamentada e controlada.

Consequentemente, as normas de proteção de dados da primeira geração têm
um olhar funcional sobre o fenômeno do processamento de dados. Se o ato
de processar é o verdadeiro problema, então a legislação deve mirar no
funcionamento do computador. Normas de proteção de dados eram vistas
como parte de uma tentativa maior de domar a tecnologia. Se a tecnologia
é uma ferramenta poderosa, ela deve ser usada para potencializar as
mudanças políticas e sociais. O processamento dos dados, de acordo com
essa perspectiva, deve ser regulado para assegurar a sua conformidade
com os objetivos da sociedade em geral. Procedimentos sociais e
políticos especiais foram concebidos para garantir o uso ``correto'' do
processamento de informações. No início da década de 1970, os
legisladores inclinavam-se em promulgar normas funcionais de proteção de
dados, isto é, com ênfase em regular a atividade em si de processamento
dos dados por meio procedimentos de autorização e registro. Visava-se,
em última análise, o controle \emph{ex ante} do uso do computador. A lei
de proteção de dados de Hesse em 1970, a primeira lei de tal gênero no
mundo, encarna essa abordagem funcional.\footnote{Hessisches
  Datenschutzgesetz, de 7 de outubro de 1970, GVBl. I, p. 625.} Ela
lidava principalmente com o fato da informação se autoprocessar
(parágrafo 1º) e regulava, portanto, sob quais condições o processamento
de dados ocorreria de forma legal. Além disso, medidas de segurança e
sigilo (parágrafo 3º) foram prescritas. A lei sueca de dados de 1973 é
ainda mais funcional em sua configuração. Questões sobre a segurança dos
dados, sigilo e precisão dominam substantivamente as suas regras que
regulam o processamento de dados (parágrafo 5º e parágrafo 6º). A
proposta de Adalbert Podlech, de 1973, de uma Lei Federal de Proteção de
Dados alemã, contém um capítulo inteiro prescrevendo regras
organizacionais para o processamento de dados,\footnote{PODLECH.
  \emph{Datenschutz im Bereich der öffentlichen Verwaltung}. 1973,
  §36--§44.} segurança\footnote{Ibid., §45--§50.} e integridade do
código-fonte.\footnote{Ibid., §26--§35.} A proposta de 1974 do governo
austríaco previa normas legais específicas, a serem internalizadas por
cada um dos banco de dados estruturados.\footnote{§9 der RV zum DSG, Nr.
  1423 d. Beilagen zu den Sten.Prot. des NR 13. GP.}

Consequentemente, a primeira geração de normas de proteção de dados não
confiou aos cidadãos-indivíduos o cumprimento das leis de proteção de
dados. Pelo contrário, criou-se instituições especiais para
supervisionar se havia tal cumprimento da lei. O comissário de proteção
de dados estabelecido pela Lei de Proteção de Dados de Hesse,\footnote{Hessisches
  Datenschutzgesetz §7.} a autoridade criada pelas leis do estado alemão
de Rheinland-Pfalz,\footnote{§ 6 Rheinland-Pfälzisches
  Datenschutzgesetz, 1974.} o comissário de proteção de dados federal da
Alemanha,\footnote{§17, BDSG.} e o (mais proeminente) Conselho Suíço de
Fiscalização de Proteção de Dados,\footnote{§15, Swedish Data Act.} a
todos eles foi delegado o poder para investigar o cumprimento das normas
de proteção de dados.

A visão funcional da primeira geração das leis de proteção de dados é
também visível em razão de outro aspecto dessas normas: algumas das
primeiras leis de proteção de dados enquadram explicitamente a atividade
de processamento de dados como algo relevante para o equilíbrio de poder
dentro do próprio governo. Em razão dos órgãos governamentais coletarem
uma quantidade massiva de dados sobre os indivíduos, o governo tinha um
instrumento de planejamento e controle, mas, também, de enorme poder em
suas mãos. Ao Poder legislativo, que supostamente deveria promulgar leis
baseadas nesses dados relevantes para o planejamento estatal, faltava o
acesso direto a tais informações. Por conseguinte, algumas das primeiras
normas de proteção de dados estabeleceram o direito do Poder legislativo
acessar os dados coletados e armazenados pela esfera
executiva.\footnote{Por exemplo, parágrafo 6º, Hessisches
  Datenschutzgesetz, 1970; parágrafo 5º, Rheinland-Pfälzisches
  Datenschutzgesetz, 1974; art. III da proposta de uma Lei
  Constitucional austríaca de Proteção de Dados, 11 jun. 1974, Nr. II -
  3586 der Beilagen zu den Sten. Protocolo do NR 13. GP.}

As leis dessa primeira geração evitavam usar palavras bem conhecidas
como ``privacidade'', ``informações'' e ``proteção da intimidade''; em
vez disso, elas empregavam jargões bastante técnicos: ``dados'', ``banco
de dados'', ``registro de dados'', ``base de dados'', ``arquivo de
dados''. Somente os ``dados'' em ``bancos de dados'' eram objeto da
regulação.\footnote{§ 6 e § 2 da proposta austríaca e §1 da BDSG.} E
exigia-se que ``bancos de dados'' fossem registrados\footnote{O §39 da
  BDSG afirma o dever de registrar. Funções de registo similares podem
  ser encontradas na proposta austríaca de 1974 e no § 10 da proposta
  alternativa de Podlech. A esse respeito o Ato britânico de Proteção de
  Dados de 1984 é um tanto anacrônico, uma vez que seus procedimentos de
  registro e sua configuração estão muito mais em sintonia com as normas
  de proteção de dados da primeira geração. A razão para isto pode
  residir menos no desejo dos legisladores britânicos de reviver as
  normas de proteção de dados da primeira geração do que em sua
  incapacidade de vislumbrar a inovação legislativa. A Grã-Bretanha
  passou seu Ato de Proteção de Dados não por causa de um desejo de
  dominar a tecnologia ou salvaguardar o indivíduo, mas porque
  interesses empresariais temiam bloqueios no fluxo de dados através das
  fronteiras.} e, algumas vezes, até autorizados.\footnote{Ver o 2º
  parágrafo da Lei sueca de Dados.}

A estrutura das legislações de proteção de dados foi, portanto, moldada
para regular os bancos de dados centrais que haviam sido idealizados.
Estabeleceram-se procedimentos complicados de registro e autorização.
Porque alguns gigantescos bancos de dados já haviam sido cogitados,
vincular as normas de proteção de dados ao contexto técnico em questão
era visto como algo aceitável e útil. Um número tão limitado de bancos
de dados, pensava-se, poderia ser controlado e regulado por meio de
procedimentos específicos. Como esses bancos de dados só seriam criados
depois de uma deliberação substancial, os procedimentos de autorização
de proteção de dados poderiam ser iniciados mesmo quando um banco de
dados estivesse, por sua vez, ainda em fase de planejamento. Além disso,
a segurança dessas bases centralizadas de dados poderia ser mantida por
simples controle do seu acesso físico.\footnote{Por exemplo, o parágrafo
  6º e o parágrafo 2º da Lei sueca de Dados, parágrafos 1º, 2º e 5º do
  apêndice até o § 6/1/1 da BDSG, §9 (1) do Ato de Registro das
  Autoridades Públicas Dinamarquesas, §9/2 da proposta austríaca e §45
  da proposta do Podlech.} De acordo com cenário, mesmo a exigência da
existência uma pessoa responsável pela proteção dos dados, o denominado
encarregado, em cada um dos bancos de dados parecia ser algo
sensato.\footnote{Consulte a \emph{Datenschutzbeauftragter}, o alemão
  BDSG de 1977 e as propostas para a Lei de Proteção de Dados austríaca.}

Procedimentos de controle complexos e dos mais variados, bem como a
estratégia em se regular o uso em si da tecnologia tiveram precedência
sobre a proteção dos direitos de privacidade do indivíduo. Isso criou um
ambiente para que leis de proteção de dados tivessem o seu enfoque,
linguagem e estrutura voltados para os aspectos organizacionais da
atividade de processamento eletrônico dos dados.

No entanto, os planos ambiciosos de bancos de dados centralizados não se
concretizaram. Não apenas os cidadãos se opuseram a eles; além disso, a
tecnologia desenvolveu-se em uma direção oposta ao longo da década de
1970. Os ``minicomputadores'' emergiram e, pela primeira vez, permitiram
que pequenas unidades organizacionais, no governo e nas empresas,
usassem um processamento eletrônico de dados descentralizado. O que
havia sido originalmente previsto como um número modesto de atores
detentores de bancos de dados e potenciais violadores da proteção de
dados subiu para milhares de unidades organizacionais individuais que
processavam eletronicamente dados. A imagem de um \emph{Big Brother}
monolítico\emph{,} que poderia ser muito facilmente regulado através de
procedimentos rigorosos baseados no estado da arte tecnologia, deu lugar
a uma imagem ampla e borrada de uma constelação de potenciais infratores
da proteção de dados, distintos e inéditos.

Isso acarretou uma mudança na discussão sobre a proteção de dados. A
proteção de dados, como foi alegado, deveria ser estendida para o
processamento de dados por pequenas empresas privadas, não devendo se
limitar aos entes governamentais e aos seus respectivos bancos de dados
centralizados.\footnote{Compare-se a proposta do governo austríaco para
  uma Lei de Proteção de Dados em 1974, que carecia de proteção de dados
  para o processamento no setor privado, exceto o fraco parágrafo 24,
  com a versão austríaca dos estatutos de proteção de dados finalmente
  promulgada em 1978 com uma seção inteira (§17-§31) sobre a proteção de
  dados para o setor privado.} Isto porque, os processos de autorização
e registro das atividades de processamento de dados se revelaram
demasiadamente complexos e demorados em um mundo não com dezenas, mas
com milhares de unidades de processamento. Como resultado, as empresas e
a administração pública começaram a desconsiderar abertamente tais
procedimentos jurídicos de proteção de dados.

Além disso, alguns dos termos técnicos e conceitos utilizados nas normas
da primeira geração (por exemplo, ``banco de dados'' e ``arquivo de
dados'') tinham perdido muito a sua validade. Consequentemente,
sugeriu-se a reforma das leis de proteção de dados.\footnote{SIMITIS.
  \emph{Kommentar}, p. 48.}

Por fim, muitos cidadãos tinham experimentado de maneira muito clara e
direta os potenciais perigos da coleta e do processamento desenfreado
dos seus dados pessoais. Como decorrência dessas experiências pessoais,
cada vez mais cidadãos protestavam por direito à privacidade e à
proteção dos dados pessoais, o que ia muito além das tentativas do Poder
Legislativo em controlar e regular uma determinada tecnologia para o
processamento de dados.

\section{A segunda geração: precavendo-se contra mais violações}

A proteção de dados da segunda geração focou nos direitos de privacidade
individual dos cidadãos. Fontes de privacidade bem conhecidas, como o
direito de ser deixado em paz e o direito de delimitar o próprio espaço
íntimo, foram trazidas para a discussão. A proteção de dados agora era
explicitamente vinculada ao direito à privacidade, e foi vista como o
direito do indivíduo de afastar a sociedade de assuntos
pessoais.\footnote{BULL. \emph{Datenschutz als Informationsrecht und
  Gefahrenabwehr}, NJW 1979, p. 1177-1182.} A privacidade informacional
tornou-se um direito garantido pelas Constituições da Áustria, da
Espanha e de Portugal.\footnote{Artigo 35 da Constituição Portuguesa de
  1976; § 1º (dispositivo constitucional) da Lei de Proteção de Dados
  austríaca (DSG), 1978; Art. 18, parágrafo 4º. Constituição espanhola
  de 1978.} As raízes da proteção de dados foram vistas nas liberdades
negativas, na liberdade individual das pessoas e em conceitos
decorrentes do Iluminismo, da Revolução Francesa e da declaração de
independência dos Estados Unidos.

O perigo não era o \emph{Big Brother}. Ele não se manifestava mais em um
punhado de bancos de dados nacionais centralizados, que tinham de ser
regulamentados, registrados e autorizados desde o início. O perigo agora
parecia estar no processamento de dados disperso por milhares de
computadores em todo o país, e acreditava-se que o melhor remédio era
que os cidadãos lutassem, eles mesmos, pela privacidade, com a ajuda dos
fortes, e até mesmo constitucionalmente protegidos, direitos
individuais.

As normas de proteção de dados criadas e adaptadas nesse período não
são, à primeira vista, totalmente distintas dos estatutos da primeira
geração. Mas um olhar mais atento revela alterações. O jargão técnico
foi eliminado, e as definições se tornaram mais abstratas e menos
ligadas a um estágio específico da tecnologia. Direitos individuais
existentes foram reforçados, vinculados às disposições constitucionais,
ampliados e estendidos. Procedimentos regulamentares enfatizaram o
registro em detrimento da autorização, e alguns dados de processamento
padrão ficaram isentos.

Os estatutos de proteção de dados francês,\footnote{Ato nº 78-17 em
  \emph{Data Processing, Data Files and Individual Liberties}, de 6 de
  janeiro de 1978} austríaco,\footnote{Lei de 18 de outubro de 1978
  Relativa à Proteção dos Dados Pessoais (DSG), BGBl. 565/78.} e em
certa medida o dinamarquês\footnote{O estatuto dinamarquês é ainda
  fortemente influenciado pelo modelo sueco de primeira geração. A
  Dinamarca tem na verdade dois estatutos de proteção de dados: um para
  o setor público, o Ato dinamarquês de Registro para Autoridades
  Públicas, nº 621 de 2 de outubro de 1987; e outro para o setor
  privado, o Ato dinamarquês para Registros Privados etc., nº 293 de 8
  de junho de 1978. A tradução utilizada para os fins do presente artigo
  foi publicada pelo Ministério da Justiça dinamarquês em outubro de
  1987 (nº 622).} e o norueguês\footnote{Lei de Proteção de Dados
  norueguesa, nº 48, de 9 de junho de 1978; Tradução fornecida pelo
  Ministério Real Norueguês para Negócios Estrangeiros, maio de 1993.}
se encontraram no começo e na vanguarda desta segunda geração. É claro
que todas as normas de proteção de dados sempre incluíram os direitos do
indivíduo de acessar e corrigir os seus dados pessoais. Mas durante a
primeira geração de normas esses direitos individuais foram
interpretados funcionalmente. Eles eram vistos como suportes para a
precisão dos dados pessoais armazenados e processados. Indivíduos não
podiam em nenhuma hipótese decidir se os seus dados seriam processados;
eles poderiam apenas retificar informações enganosas ou imprecisas sobre
eles mesmos.\footnote{Conforme estabelecido no parágrafo 4º do
  Hessisches LDSG (1970), na seção 10 da Lei sueca de Dados (1973), nos
  §11-§14 do LDSG Rheinland-Pfalz (1974), no §16 da proposta de Podlech,
  nos §10 e §11 da proposta do governo austríaco de 1974.}

Na segunda geração dos direitos de proteção de dados, indivíduos
obtiveram voz no processo. Seu consentimento era às vezes uma condição
prévia para o processamento de dados; em outros casos, o consentimento
individual podia sobrescrever uma inferência legal que proibia o
processamento. Esses direitos se diferiam substancialmente dos direitos
de acessar, modificar, e sob certas condições excluir seus próprios
dados pessoais. Os direitos individuais recém-criados delegavam um poder
explícito de decisão para que o indivíduo pudesse escolher quais dos
seus dados pessoais seriam usados para quais fins. Por exemplo, o
parágrafo 7º e o parágrafo 33 da Lei de Proteção de Dados norueguesa
delegavam aos indivíduos o direito de recusar o processamento de seus
dados para fins de marketing direto e estudos de mercado. As normas
dinamarquesas sobre proteção de dados confiavam aos indivíduos os
direitos de decidirem sobre a transferência de dados pessoais de bancos
de dados públicos para privados, sobre o armazenamento de informações
sensíveis e antigas em bancos de dados privados e sobre a transferência
de dados de consumidor.\footnote{§16, Ato dinamarquês de Registro para
  Autoridades Públicas; § 4º, Ato Dinamarquês para Registros Privados
  etc.}

Além disso, o enorme crescimento dos fluxos transnacionais de
informações acrescentou um novo desafio dimensão às normas nacionais de
proteção de dados. Todavia, a regulação continuou a se dar à nível
local, sendo as transferências internacionais operacionalizadas com base
em regras de reciprocidade.

Assim, criaram-se normas de proteção de dados bastante peculiares. Ainda
que inspiradas por suas antecessoras de primeira geração, as leis de
segunda geração procuraram seguir um caminho diferente. Abandonou-se a
ideia de que haveria uma grande solução. As legislações agora esperavam
--- sem fornecerem uma razão teórica ou empírica para o seu entusiasmo
--- que o cidadão-indivíduo era o agente mais capacitado para garantir a
proteção dos seus dados. Da tentativa em regular a própria tecnologia, a
proteção dos dados pessoais passou a ser enquadrada como uma liberdade
individual dos cidadãos.

Na Alemanha, em nível federal, essa mudança de estratégia --- mais bem
expressada pelo primeiro comissário federal de dados, Hans-Peter
Bull\footnote{Ver, por exemplo, BULL. \emph{Datenschutz als
  Informationsrecht und Gefahrenabwehr}. NJW 1979, p. 1177-1182.} ---
permaneceu sendo em grande parte apenas retórica. No entanto, as leis
dos estados alemães internalizaram o tema. E, nesse sentido, as leis de
proteção de dados austríaca, dinamarquesa e norueguesa, todas decretadas
em 1978, seguiram essa nova orientação.

A mudança temática é perceptível também no domínio institucional da
aplicação e fiscalização da proteção de dados aos órgãos fiscalizadores.
As instituições já enxergavam mudanças (ou melhor novas atribuições) em
sua lista de tarefas. Essas mudanças eram respostas diretas ao novo
enfoque para a aplicação e fiscalização dos direitos do titulares dos
dados pessoais recentemente ampliados e garantidos.

Em primeiro lugar, alguns órgãos fiscalizadores em meio à segunda
geração das leis de proteção de dados pessoais iam além da investigação
das violações desses direitos, transformando-se em algo como o ombudsman
da proteção de dados pessoais. Os direitos dos titulares dos dados
pessoais foram reforçados, os cidadãos precisavam de uma instituição
para a qual relatassem a violação dos seus direitos --- um ator que de
alguma forma os ajudassem a fazer valer os direitos prevista nas leis de
proteção de dados.

Em segundo lugar, algumas instituições de proteção de dados foram
transformadas, bem como aparelhadas para se tornarem efetivamente órgãos
fiscalizadores e de aplicação da lei. Elas passar a emitir opiniões
acerca de como regras de proteção de dados deveriam ser interpretadas.
Esse foi o caso das agências de proteção de dados francesa e austríaca.
A autoridade francesa e sueca não são só responsáveis pelos
procedimentos de registro para o processamento de dados, mas, também,
funcionam como um órgão judicante em disputas sobre proteção de dados
pessoais (artigo 35). A autoridade de proteção de dados austríaca
decide, quase como tribunal, sobre controvérsias em torno da proteção de
dados entre cidadãos e autoridades públicas (parágrafo 14). A autoridade
dinamarquesa (DSA) também tem poder decisório sobre demandas individuais
de violações de direitos,\footnote{§15, Ato dinamarquês de Registro para
  Autoridades Públicas; §15, Ato dinamarquês para registros privados
  etc.} tal como sucede com a autoridade da Noruega.\footnote{§8, Lei de
  Proteção de Dados norueguesa.}

Essa reorientação da proteção de dados, da regulaçao da tecnologia para
a sua configuração como liberdade individual, foi associada
retoricamente às velhas categorias jurídicas-legal do direito à
privacidade. No entanto, os nobres ideais de liberdade negativa e de
liberdade individual permaneceram sendo, em larga medida, um exercício
político de pensamento idealizado. Sua internalização em lei estava
fadada ao fracasso. É impossível alcançar a privacidade e a liberdade
individual informacional sem comprometer o complexo funcionamento dos
Estados de bem-estar social. Na vida real, o indivíduo raramente teve a
chance de decidir entre fazer parte ou não da sociedade.

O direito do cidadão em aderir à serviços sociais e ser beneficiário de
programas de transferências de renda governo exigem um fluxo contínuo de
informações pessoais suas para a burocracia governamental. Os cidadãos e
a sociedade são tão intensamente e subliminarmente interligados, que uma
tentativa deliberada por parte do indivíduo de resistir a tais pedidos
de informações, se é que isso é possível, acarretaria um custo social
extraordinário. Da mesma forma, questões bancárias, monetárias, vistos
para viagens e o voto, em todas essas atividades a divulgação de
informações pessoais é, na maioria das vezes, uma pré-condição para a
participação do indivíduo.

A proteção de dados como parcela da autonomia privada pode proteger a
liberdade do cidadão. Pode a ele oferecer a possibilidade de
simplesmente recusar os pedidos de informação que sejam excessivos. Mas
sob qual preço? É aceitável que tais direitos de proteção de dados
possam ser apenas exercidas por eremitas? Chegamos a um plano ideal de
proteção de dados que garante que o exercício do direito à privacidade
tem como efeito colateral a exclusão social?

\section{A terceira geração: o direito à autodeterminação informacional}

Essas ideias e outras semelhantes conduziram a uma terceira grande
reforma nas leis de proteção de dados. A liberdade individual e o
direito de se colocar a salvo de violações aos dados pessoais foram
transformados em um direito mais participativo como indica a
terminologia autodeterminação informacional. O indivíduo agora passa a
ser capaz de determinar como seus dados o representarão na sociedade. A
pergunta não é mais se alguém quer participar desse processos social,
mas como.\footnote{Ver SIMITIS. \emph{Kommentar}, p. 57; cf. WESTIN.
  \emph{Privacy and Freedom}, p. 7.}

Esta foi a linha de argumentação da famosa decisão sobre o censo alemão
da Corte Constitucional daquele país\footnote{Decisão do Senado, 15 de
  dezembro de 1983, 1 BvR 209/83-NJW 1984, p. 419.} de 1983, que
popularizou o termo ``autodeterminação informacional''.\footnote{SIMITIS.
  Reviewing Privacy in an Information Society. \emph{University of
  Pennsylvania Law Review}, 135, 1987, p. 734; SIMITIS\emph{. Die
  informationelle Selbstbestimmung, Grundbedingung einer
  verfassungskonformen Informationsordnung}, NJW, 1984, p. 398ff;
  VOGELSANG. \emph{Grundrecht auf informationelle Selbstbestimmung}.
  1987.} Não é por acaso que essa abordagem participativa apareceu em um
momento em que as virtudes e as tradições cívicas desfrutavam de uma
súbita retomada, enfatizando-se a participação ativa e deliberada em
detrimento das liberdades negativas.\footnote{Para a Alemanha, ver
  KRIELE; FREIHEIT; GLEICHHEIT. \emph{Handbuch des Verfassungsrechts}.
  In: BENDA et al. (Ed.). 1984; para os EUA, ver KARST. Equal
  Citizenship Under The Fourteenth Amendment. \emph{Harvard Law Review},
  91, 1977, p. 1; KARST. \emph{Belonging to America}. 1989. Sobre o
  renascimento cívico republicano em geral, ver MICHELMAN. Law's
  Republic. \emph{Yale Law Journal}, 97, 1988, p. 1493; POWELL. Reviving
  Republicanism. \emph{Yale Law Journal}, 97, 1988, p. 1703; SANDEL.
  \emph{Liberalism and the Limits of Justice}. 1982; POCOCK. \emph{The
  Machiavellian Moment}. 1975.}

Essa reinterpretação da proteção de dados como um direito à
autodeterminação informacional pode ser facilmente encontrada na decisão
do Tribunal Constitucional alemão: ``os direitos fundamentais garantem a
capacidade do indivíduo de decidir em geral, por si próprio, a
disponibilização e o uso de seus dados pessoais''.\footnote{Ibid.}

A decisão do Tribunal não apenas vinculou a proteção de dados
explicitamente a uma disposição constitucional, ao extrair do texto
constitucional o direito da autodeterminação informacional; ela também
teve profundas consequências na estrutura inteira da Lei Federal de
Proteção de Dados alemã, originalmente orientada de maneira mais
funcional restritiva. O tribunal declarou que todas as fases do
processamento dos dados, da coleta à transmissão, estão sujeitas às
limitações constitucionais. Consequentemente, os direitos de
participação do indivíduo precisam ser estendidos a todas as fases de
processamento da informação. O indivíduo não pode apenas, como ocorria
com as normas da segunda geração, decidir de uma vez por todas, através
de uma escolha binária do tipo ``ou tudo ou nada'', se quer ter seus
dados pessoais processados. Ele tem que estar --- pelo menos em
princípio --- continuamente envolvido ao longo de todas as etapas do
tratamento dos seus dados. Além disso, o tribunal deixou claro que o
Estado deve, sempre que solicitar os dados pessoais dos cidadãos,
explicar porque os dados solicitados são necessários, bem como as
consequências específicas que por ele acarretar caso se negue a
consentir para o processamento dos seus dados.

O tribunal também considerou insuficiente a linguagem dos direitos de
participação existentes. Sob o outrora regime funcional mais restritivo
da proteção de dados, quem processava as informações precisava apenas
levar em conta os interesses individuais ``dignos de proteção''. Na sua
decisão, a Corte Alemã substituiu essa norma a ser construída pelo
próprio julgador, por um critério normativo que levasse em conta mais o
indivíduo, o que seria preponderante para se analisar a legalidade do
processamento de dados pessoais. A proteção de dados vislumbrada pelo
tribunal é o arquetípico da leis da terceira geração. O indivíduo não
está sitiado em sua casa sob pressão para abrir as comportas dos seus
dados pessoais e ser destituído do direito de controla-los dali em
diante. Pelo contrário, pelo princípio da autodeterminação informacional
o ser humano individual deve estar presente durante todo o ciclo do
tratamento dos seus dados pessoais.

As revisões e substituições dos termos técnicos da primeira geração
faziam parte da segunda geração, em decorrência do progresso
tecnológico. Ao longo da terceira geração de proteção de dados, a
tecnologia da informação se desenvolveu em um sentido ainda mais
distante dos modelos centralizados de processamento de informação.

As tecnologias de rede e de telecomunicações tornaram possível que
estações de trabalho e computadores pessoais fossem conectados entre si
por meio de redes eletrônicas rápidas, eficientes e baratas. Os dados já
não são mais facilmente rastreáveis fisicamente, nem armazenados em uma
unidade de processamento central específica e bem definida. Em vez
disso, são armazenados em redes e podem ser transferidos em segundos.
Essa mobilidade torna inúteis as medidas técnicas de segurança de dados
antes imaginadas, as quais já haviam sido comprovadas anteriormente
ineficientes.

Essa mudança de cenário criou novos desafios legais. Ainda que, muito
antes, com a segunda geração de normas de proteção de dados, as
legislações tivessem abandonada a ideia de regulamentar a tecnologia. Ao
invés de trilharem esse difícil caminho de adaptar continuamente a
legislação ao estado da arte da tecnologia, os legisladores optaram por
conceitos mais abstratos de liberdade individual e direitos de
participação. As tendências tecnológicas de descentralização das
atividades de tratamento dos dados só fizeram acelerar esse voo
legislativo em direção a uma regulação que fosse mais substantiva.

Assim, as normas de proteção de dados da terceira geração
caracterizam-se pela concentração --- para não chamar de recuo --- no
direito da autodeterminação informacional e, por conseguinte, na crença
de que os cidadãos o exercerão. Eles incluem uma grande dose de
transigência pragmática necessária para trazer resultados normativos, na
busca frequentemente longa por um meio termo entre a promoção e o
controle eficiente do processamento das informações.

Algumas reformas legislativas se enquadram nessa categoria: as leis de
proteção de dados dos Estados alemães que vieram logo após a decisão do
Tribunal Constitucional; a tardia reforma deral à Lei Federal de
Proteção de Dados alemã (1990); a reforma de 1986 da Lei austríaca de
Proteção de Dados, na qual os procedimentos de autorização e registro
foram substancialmente simplificados e se tornaram coadjuvantes, em
comparação aos direitos da autodeterminação informacional que foram
ampliados; o mesmo ocorreu na Lei de Proteção de Dados norueguesa; a
emenda constitucional holandesa para garantir a proteção dos dados
pessoais; e, por fim, reformas pontuais da da Lei finlandesa de Registro
Civil de 1987.

A terceira geração enfatizou a participação do titular dos dados, tal
como prescreve o direito da autodeterminação informacional. O
renascimento cívico republicano forneceu apenas um alicerce teórico. O
outro foi o entendimento mais pragmático por parte do legislativo de que
havia uma sociedade cada vez mais interdependente e interligada, na qual
o indivíduo deveria exercer controle sobre as suas informações pessoais.
A ele deveria ser dada a oportunidade de moldar a forma com que os seus
dados o identifica perante a sociedade. Os legisladores ainda
acreditavam que os cidadãos estavam se protegendo em pé de igualdade com
as poderosas práticas de coleta e processamento das suas informações.
Por isso, as legislações tinham enfoque na responsabilidade dos cidadãos
exercerem seus direitos, ainda que houvesse a percepção de que para o
indivíduo controlar sua própria imagem informacional deveria haver a
observância dos direitos individuais de maneira rigorosa e clara.

Os direitos de participação do titular dos dados pessoais foram
ampliados novamente. As diversas e novas leis alemães de proteção de
dados dão voz aos indivíduos não só na fase de processamento, mas também
em todas as outras fases de coleta, armazenamento, tratamento e
transmissão de informações. Os cidadãos gozavam de prerrogativas
especiais no campo de pesquisa de mercado e na supressão de informações
defasadas.\footnote{Por exemplo, o parágrafo 28, (nova) Lei Federal
  alemã de Proteção de Dados de 1990; e o parágrafo 17 da Lei de
  Proteção de Dados do estado alemão de Berlim.} Da mesma forma, a Lei
finlandesa de Registro Civil previa o direito do cidadão em consentir ou
recusar o compartilhamento e a vinculação de suas informações
pessoais.\footnote{§18-§20, Lei finlandesa para Registro de Pessoas.}

Mas a realidade acabou sendo novamente diferente. Mesmo quando
habilitadas com novos e ampliados direitos de participação no fluxo
informacional, as pessoas não estavam dispostas a pagar o elevado custo
monetário e social que teriam que suportar ao exercerem rigorosamente o
seu direito de autodeterminação informacional. A esmagadora maioria
temia o que poderia acontecer durante a contratação de serviços
bancários e financeiros, se a recusa em fornece seu dados pessoal
poderia resultar na abertura de ações judiciais. Com isso, o cidadão
abdicava de maneira rotineira e inconsciente ao seu direito à
autodeterminação informacional que nem mesmo poderia ser considerada uma
``moeda de troca'' durante as negociações. Ainda assim, uma vez que o
consentimento era uma das base legais para legitimar o processamento dos
dados, houve a sua desvalorização contratual para dar legalidade a tais
atividades de tratamento de dados pessoais. O direito da
autodeterminação informacional se tornou, de repente, um tigre banguela
de papel.

Portanto, apesar de tentativas deliberadas de se ampliar o acesso e
simplificar o exercício do direito à proteção de dados pessoais, esse
permaneceu em larga medida um privilégio de minorias. Somente aqueles
que poderiam bancar econômica e socialmente o exercício desse direitos.
A prometida capacidade dos cidadãos exercerem em grande escala o
controle da sua própria imagem informacional permaneceu sendo uma
retórica política.

\section{A quarta geração: perspectivas holísticas e setoriais}

As respostas para os problemas acima apontados podem ser amalgamadas em
uma quarta geração de normas de proteção de dados. Os legisladores
perceberam que a posição do indivíduo na mesa de negociação era
geralmente fraca. A quarta geração de leis de proteção de dados tentam
corrigir isso, por meio de duas abordagens um tanto distintas.

Por um lado, elas tentam equalizar a assimetria de poder através do
reforço do empoderamento do indivíduo frente aos atores com interesses
em seus dados pessoais, geralmente mais poderosos. Em sua essência, tais
tentativas mantêm a crença na capacidade do indivíduo de efetivar a
proteção de seus dados através da autodeterminação individual, caso o
poder de barganha fosse simétrico.

Por outro lado, as leis de quarta geração retiraram parte da autonomia
privada do indivíduo constantes das normas de proteção de segunda e
terceira gerações. Tal abordagem reflete o entendimento de que algumas
áreas de privacidade informacional devem ser absolutamente protegidas e
não podem ser objeto de negociação.

Cada uma dessas abordagens encontrou um caminho nas normas de proteção
de dados de quarta geração. Por exemplo, as recentes reformas às leis de
proteção de dados dos estados alemães e a lei federal de proteção de
dados alemã de 1990 introduziram o sistema de responsabilidade objetiva
na proteção de dados pessoais.\footnote{Parágrafo 7º, (nova) Lei Federal
  alemã de Proteção de Dados de 1990. Para obter um exemplo de uma regra
  independente de responsabilidade em nível estadual, veja o § 20 da Lei
  de Proteção de Dados do estado alemão de Brandemburgo.} O mesmo fez o
modelo norueguês com relação aos bureau de créditos.\footnote{§40, Lei
  de Proteção de Dados norueguesa.}

Em consonância com a segunda abordagem, certos dados pessoais ficaram
fora do escopo da autonomia do indivíduo. O tratamento de dados pessoais
sensíveis via de regra proibido. Esse conceito está previsto no §6º da
Lei de Proteção de Dados norueguesa, no §6º da Lei finlandesa para
Registro Civil, nas leis de proteção de dados dinamarquesas,\footnote{Mas
  apenas para o setor público (§9, secção 2). No setor privado,
  indivíduos devem explicitamente consentir no armazenamento de dados
  sensíveis (parágrafo 4º, seção 1)} no artigo 6º da lei de proteção de
dados belga, na seção 31 da lei de proteção de dados francesa e na lei
britânica 62 de proteção de dados.\footnote{O Ato britânico de Proteção
  de Dados autoriza o governo a regulamentar ainda mais a proteção de
  dados sensíveis, nos termos da seção 2.-(3). Além disso, o tratamento
  de dados sensíveis é restringido pelos estatutos nacionais de proteção
  de dados da Irlanda, Luxemburgo, Países Baixos, Portugal e Espanha.
  Para uma comparação útil classificada por temas, consulte
  KUITENBROUWER; PIPE. Compendium of European Data Protection
  Legislation. In: \emph{Data Protection in the European Community},
  SIMITIS et al (Ed.).} As leis de proteção de dados suíça e alemã,
particularmente dos novos estados alemães, não proíbem o tratamento de
determinados dados sensíveis, mas, em vez disso, restringem a autonomia
privada dos indivíduos com relação à negociação dos seus direitos
básicos de proteção de dados pessoais, como de acesso, correção e
exclusão pelo indivíduo.\footnote{Ver artigo 8, seção 6, Lei suíça de
  Proteção de Dados (1992); Parágrafo 6º, Lei Federal de Proteção de
  Dados alemã (1990); parágrafo 5º, seção (1), último parágrafo, Lei de
  Proteção de Dados do estado alemão de Brandemburgo (B-LDSG).} Da mesma
forma, as Diretrizes da União Europeia para a Proteção de Dados de 1995
proíbem o tratamento de dados sensíveis (raça, religião, opiniões
políticas etc.), exceto em alguns casos enumerados
taxativamente.\footnote{Artigo 8 ° das Diretrizes da União Europeia.}

Além disso, a quarta geração das leis gerais de proteção de dados são
complementadas por leis setoriais e específicas. Isso significa a
propagação e aceitação por toda a Europa de uma abordagem que surgiu
originalmente nos países nórdicos. Os estatutos norueguês, dinamarquês e
finlandês incorporaram, por algum tempo, leis setoriais específicas de
proteção de dados.\footnote{§1--§5 da Lei finlandesa de Registro de
  Pessoas regulamentam o uso de arquivos para pesquisa, estatística,
  pesquisa de mercado, marketing direto e emissão de relatórios de
  crédito. O §13 da Lei de Proteção de Dados norueguesa regulamenta a
  emissão de relatórios de crédito, §25, corretagem de endereço e §31,
  pesquisa de mercado. O 8º parágrafo da Lei dinamarquesa para Registros
  Privados etc. restringe relatórios de crédito, §17, de marketing
  direto. Essa abordagem setorial também pode ser rastreada até a
  abordagem muito específica prevista pela original Lei de Proteção de
  Dados sueca, que incumbiu o Conselho Sueco de Inspeção de Dados de
  emitir regulamentos específicos para cada arquivo registrado.} Agora,
mesmo os países em que as leis gerais de proteção de dados têm uma longa
tradição --- incluindo a Alemanha e a Áustria --- há uma tendência desse
quadro geral ser complementado setorialmente.\footnote{Por exemplo, a
  proteção de dados setoriais relacionados ao marketing direto e à
  corretagem de listas representados no §268 austríaco. Abs. 3 GewO.
  1994, BGBl. 1994/194.} E., nesse linha, as diretrizes da união
europeia de 1995 exigem explicitamente ``códigos de conduta'' setoriais
para o tratamento de dados.\footnote{Artigo 27 das Diretrizes da União
  Europeia.}

Com relação à aplicação e fiscalização das leis, houve a criação da
figura do ombudsman de proteção de dados em alguns países, bem como de
instituições com poder decisório para a aplicação e fiscalização de
forma imparcial das leis.\footnote{Ver BURKERT. Institutions of data
  protection. \emph{Computer Law Journal}, 3, 1982, p. 167-188.
  Wippermann, (\emph{Zur Frage der Unabhängigkeit der
  Datenschutzbeauftragten}, 929 DÖV, 1994) demonstrou que o comissário
  de proteção de dados alemão, originalmente advocatício, tornou-se cada
  vez mais parajudicial, e tem rastreado eloquentemente o início desta
  mudança para a Decisão de Censo da Corte Constitucional Alemã.}
Enquanto o primeiro tinha a atribuição de investigar violações e
auxiliava os cidadãos em suas demandas sobre proteção de dados, os
últimos assumiam o papel decisório de deliberar acerca de dessas
reivindicações. Esse foi um passo importante e pioneiro em diversos
sentidos. Primeiramente, tais leis reconheceram a necessidade tanto de
instituições de assistência, como, também, de fiscalização e aplicação
dos direitos de proteção dos dados pessoais. Como resultado, em segundo
lugar, houve a implementação de vários mecanismos que encorajavam os
cidadãos a exercerem os seus direitos, inclusive a ingressarem com
medidas judiciais-administrativas. Ao final, resolveu-se o problema
constitucional da confusão entre tarefas de defensoria e de
aplicação-fiscalização na proteção de dados. Agora o defensor da
proteção de dados podia fazer pressão pública por melhores aplicações
das normas de proteção de dados, enquanto o corpo fiscalizatória e de
aplicação da lei podia concentrar-se apenas nas tomadas de decisão. As
leis de proteção de dados da Finlândia (1987)\footnote{Lei finlandesa de
  Registro de Pessoas, 30 de abril de 1987, 471-HE 49/86.} e da Suíça
(1992)\footnote{Artigos 26 33, Lei suíça de Proteção de Dados; §29 e 35,
  Lei finlandesa de Registro de Pessoas.} implementaram tal arranjo
institucional bifurcado.

Com tal quadro institucional, a figura do defensor de proteção de dados
representa um passo além da ideia dos cidadãos garantirem por si mesmos
as as regras de proteção de dados. Isso produz uma simbiose
interessante, entre a atuação do próprio cidadão e o envolvimento direto
do Estado para a proteção dos dados. Essas legislações tentaram resgatar
a ideia do direito à autodeterminação informacional como cerne de todo o
modelo de proteção de dados, mas que é agora posto em prática de forma
detalhada, completa e que fornece suporte ao cidadão. A intervenção
direta do Estado, modo predominante nas leis de primeira geração e desde
então obsoleto, é parcialmente ressuscitada, mas condicionada a esse
papel complementar e de apoio à autonomia do cidadão no arranjo jurídico
da proteção dos dados pessoais.

A Diretiva Europeia de Proteção de Dados Pessoais de 1995 reflete essa
evolução geracional. Os direitos básicos de proteção de dados pessoais
(ARCO) são dela protagonistas. Lista-se o consentimento do titular dos
dados como uma das bases legais para legitimar o tratamento de dados
pessoais, bem como para a transferência de dados pessoais para países
com regimes inadequados de proteção de dados.\footnote{Artigo 7 (a),
  artigos 25 e 26.} Exige-se que o consentimento seja informado e
expresso para o tratamento de dados sensíveis.\footnote{Artigo 8, seção
  1.} Os cidadãos têm o direito de vetar o tratamento de seus dados para
fins de marketing-publicitários.\footnote{Artigo 14 (b).} Outras leis
preveem medidas efetivas para a aplicação da lei, bem como o direito à
indenização por violação dos direitos previstos em lei.\footnote{Artigos
  22 e 23.} De forma totalmente antagônica com as normas de proteção de
dados da primeira geração, idealizadas para domar a tecnologia, a
diretiva europeia e as leis da quarta geração abrangem não apenas o
processamento de dados em computadores, mas, também, o processamento
manual de dados em arquivos manuais, desde que esses arquivos estejam
estruturados e classificados (por exemplo, em ordem
alfabética).\footnote{Artigo 2 (c) em conjunto com o Artigo 3 (1) das
  Diretrizes da União Europeia.} Assim, essa nova geração de leis
ampliam substancialmente a aplicação da proteção de dados na esfera
administrativa. Além do mais, incluem algumas regras para a proteção de
dados setoriais específicas.\footnote{§14 (b), que regulamenta o
  marketing direto; §15 regulamenta sistemas automatizados de tomada de
  decisão; mas veja-se a chamada para ``códigos de conduta'' no artigo
  27.}

\section{Conclusão}

Desde 1970, as normas de proteção de dados europeias têm evoluído para
um quadro jurídico dinâmico e em constante alteração. Enquanto a
proteção de dados se transformou em um conceito aceito, seu conteúdo foi
deslocado e ajustado para abordar os desafios e mudanças tecnológicas,
bem como para levar em consideração as transformações filosóficas e
ideológicas. A proteção de dados já não é vista como uma ferramenta
puramente funcional, a ser usada para moldar e influenciar diretamente o
uso da tecnologia para o processamento de informações. Pelo contrário, o
foco passou a ser o próprio cidadão. Os direitos dele constam de maneira
destacada em todos o sistema europeu de proteção de dados. Afastou-se de
versões simplistas da privacidade informacional como uma liberdade
negativa, aproximando-se dos direitos que promovessem a participação do
indivíduo (liberdade positiva). Diante disso, a autonomia do titular dos
dados pessoais continuará a ser a peça central dos regimes de proteção
de dados ao longo das próximas décadas.

A evolução da proteção de dados não acabou. É um processo contínuo.
Daqui a uns anos, pode-se vir a encontrar uma quinta ou sexta geração de
normas de proteção de dados ocupando o território jurídico europeu.
Embora seja difícil prever o caminho que a proteção de dados pode
percorrer, algumas observações gerais podem ser feitas.

• As futuras normas nacionais de proteção de dados na Europa serão muito
mais coesas do que as existentes. A Diretiva Europeia de Proteção de
Dados Pessoais acarretará uma certa uniformidade das legislações
nacionais , garantindo-se o livre fluxo de dados e informações em toda a
União Europeia e, ao mesmo tempo, um elevado nível de proteção. A
tecnologia apontará para o mesmo lugar. O crescimento da internet e de
outras infraestruturas transnacionais de informação faz o fluxo
informacional transfronteiriço ser um assunto do cotidiano, pressionando
economicamente para que os ordenamentos jurídicos nacionais se atualizem
para que haja um quadro jurídico internacional uniforme.

• Os direitos de participação do titular dos pessoais continuarão a ser
expandidos. Novas tecnologias de informação e comunicação agora tornaram
tecnologicamente possível trazer o indivíduo para todas as etapas de
decisão do tratamento dos seus dados. Os custos de transação, para obter
do cidadão o seu consentimento para os vários estágios dessa cadeia, já
diminuíram e vão diminuir ainda mais.

• A perda da soberania nacional sobre os fluxos de informação,
resultante de infraestruturas de informação interligadas
internacionalmente, apresentará um duro desafio para a aplicação
nacional da proteção de dados. No entanto, a flexibilização de direitos
de participação dos titulares dos dados, característica marcante da
terceira e da quarta gerações de normas de proteção de dados, permitirão
que sobrevenham estruturas contratuais para a resolução desses litígios.

• Questões técnicas e de segurança da informação continuarão a ser
``descentralizadas'' e enfrentarão o mesmo destino que a tecnologia
experimentou ao tempo das leis de primeira geração. Enquanto os direitos
básicos de proteção de dados permanecerá condensado em leis gerais,
questões mais técnicas tendem a ser internalizadas por normas setoriais
e mais específicas. A Diretiva Europeia de Proteção de Dados Pessoais
aponta para essa direção, bem como as projetadas leis setoriais
europeias sobre a proteção de dados para ISDN\footnote{\emph{ISDN:
  Integrated Services Digital Network} (Rede Digital de Serviços
  Integrados). Diretriz para a Proteção de Dados Pessoais e Privacidade
  no Contexto das Redes Digitais de Telecomunicações, a ser aprovado em
  1997.} e para correspondência.\footnote{Proposta de Diretriz sobre a
  Proteção dos Consumidores em Relação a Contratos Negociados à
  Distância (Venda à Distância).}

• Todos esses desenvolvimentos jurídicos e técnicos aumentarão a pressão
econômica e política geral para que os Estados Unidos aprove leis de
proteção de dados pessoais semelhantes.

Espera-se que a visão geracional tenha ajudado a lançar luz sobre o
processo evolutivo que vem ocorrendo desde 1970 e, por conseguinte,
sobre os dois temas subjacentes sempre presentes em todas as leis
europeias de proteção de dados acima descritas: primeiro, o desejo de
disciplinar, moldar e instrumentalizar a tecnologia; e, segundo, o
objetivo de associar a proteção de dados a algum valor individual mais
profundo. A legislação sobre a proteção de dados nos últimos vinte e
cinco anos tem mostrado aos legisladores, nacionais e europeus, o que
pode ser feito com êxito e com que consequências. A experiência europeia
não é um monstro excessivamente complexo e desprovido de praticidade,
eficiência e pragmatismo. Pelo contrário, ela reflete profundas
convicções e uma tentativa digna de se buscar a proteção efetiva da
única entidade que verdadeiramente importa na nossa sociedade: as
pessoas.

\section{Agradecimentos}

Agradeço ao Professor Herbert Hausmaninger por sua orientação e a Dean
Teree E. Foster por toda a sua ajuda, apoio e incentivo.

\chapter{Uma Abordagem Contextual da Privacidade
On-Line}\label{uma-abordagem-contextual-da-privacidade-on-line}

Helen Nissenbaum

\emph{Dædalus,} periódico da Academia Americana de Artes e Ciências

\emph{Sinopse: Recentes revelações de mídia têm demonstrado a extensão
da terceirização do rastreamento e do monitoramento on-line, muito do
qual impulsionado pela agregação de dados, criação de perfis e seleção
de alvos. Como proteger a privacidade online é uma pergunta frequente no
discurso público e que reacendeu o interesse dos agentes reguladores.
Nos Estados Unidos, o modelo de ``notificação e consentimento''
permanece sendo a abordagem contingencial protagonista nas políticas de
privacidade on-line, apesar de suas fraquezas. Este ensaio apresenta uma
abordagem alternativa, enraizada na teoria da integridade contextual.
Propostas para melhorar e fortalecer o modelo de notificação e
consentimento, tais como políticas de privacidade mais claras e informar
de forma mais adequada, não superam uma falha fundamental deste modelo,
qual seja, a suposição de que os indivíduos são capazes de compreender
todos os fatos relevantes para um processo de tomada de decisão
verdadeiro como se no momento da contratação indivíduos e quem processa
seus dados estivessem em pé de igualdade. Em vez disso, nós devemos
articular uma abordagem contextual especificada com base em normas
substantivas que restringem a informação que pode ser coletada por
sites, com quem eles podem compartilhá-la e sob quais condições. Ao
desenvolver esse tipo de abordagem, este artigo alerta que o atual viés
de concepção da Internet, como uma empreitada predominantemente
comercial, limita seriamente a agenda de privacidade.}

HELEN NISSENBAUM é Professora de Mídia, Cultura e Comunicações e Membro
Sênior no Instituto de Direito de Informação na Universidade de Nova
York. Seus livros incluem \textbf{Privacy in Context: Technology, Policy
and the Integrity of Social Life} - Privacidade no Contexto: Tecnologia,
Política e a Integridade da Vida Social (2010), \textbf{Academy \& the
Internet} - Academia \& Internet (editado com Monroe E. Price, 2004) e
\textbf{Computers, Ethics \& Social Values} - Computadores, Ética e
Valores Sociais (editado com Deborah G. Johnson, 1995).

O ano de 2010 foi marcante para a ``privacidade
online''.\textsuperscript{{1}} Relatos sobre ``gafes da privacidade'',
tais como as inconsistências das políticas de privacidade do Google Buzz
e do Facebook, tomaram as primeiras páginas dos mais importantes
veículos da mídia. Na série "Sobre o que eles sabem", o \emph{The Wall
Street Journal} apontou o holofote para o monitoramento desenfreado das
pessoas para fins de publicidade comportamental e por outras
razões.\textsuperscript{{2}} O governo dos Estados Unidos, através da
Comissão Federal de Comércio - \emph{Federal Trade Commission --
(FTC)}\textsuperscript{{3}} e do Departamento de Comércio,
\textsuperscript{{4}} lançou dois relatórios em dezembro de 2010,
descrevendo a Internet como um lugar onde cada passo é vigiado e cada
clique é registrado por entidades, governamentais e privadas, sedentas
por dados. E, por fim, onde cada resposta é cobiçada por formadores de
opinião e ``buscadores de atenção''.\textsuperscript{{5}}

Este artigo explora as atuais preocupações sobre a ``privacidade
online'' com o intuito de compreender e explicar a realidade , as
atividades a e agitações deste campo, a fim de identificar os
princípios, as forças e os valores em jogo. Considera-se porque a
``privacidade online'' tem sido marginalizada, mas para além das
preocupações gerais sobre privacidade; porque as abordagens correntes
têm persistido, apesar de seus resultados limitados; e por que elas
deveriam ser desafiadas. Finalmente, o artigo estabelece uma abordagem
alternativa para endereçar o problema da ``privacidade online'' com base
na teoria da privacidade contextual, como algo cuja integridade depende
do seu contexto. Essa abordagem leva em consideração os ideais
formativos da Internet como um bem público. \textsuperscript{{6}}

Deixando de lado os fatores econômicos e institucionais, os desafios à
privacidade associados à internet são semelhantes àqueles causados
outrora por outros sistemas de informação e pela mídia digital, devido
às suas vastas capacidades de coleta, armazenamento, restauração,
análise, distribuição, exibição e disseminação de informações. Em uma
ecologia online florescente, onde indivíduos, comunidades, instituições
e corporações geram conteúdo, experiências, interações e serviços, a
moeda em jogo é obter informações, notadamente aquelas sobre pessoas.
Com o aumento do uso da Internet e da Web, elas se tornaram as
principais fontes de informação e mídia para transações, interações e
comunicações, particularmente entre pessoas e sociedades mais bem
alfabetizadas digitalmente. Desde então, nós temos assistido radicais
perturbações no fluxo das informações pessoais. Em meio a crescente
curiosidade e preocupação sobre o fluxo dessas informações, os
legisladores, advogados, defensores de direitos humanos e a mídia têm
respondido com denúncias e críticas ao rastreamento intrusivo e
clandestino, à publicidade comportamental manipuladora e os compromissos
evasivos com relação à privacidade dos principais atores corporativos.

No livro \emph{Privacidade em Contexto: Tecnologia, Política e a
Integridade Da Vida Social}, \textsuperscript{{6}} apresento uma análise
da privacidade com base em qual é o fluxo esperado das nossas
informações pessoais, a partir da construção de normas informacionais
contextuais. Os principais parâmetros dessas normas informacionais são
os atores (assunto, remetente, destinatário), os atributos (tipos de
informações), e os princípios de transmissão (restrições quanto ao fluxo
das informações). Geralmente, quando o fluxo de informações adere aos
costumes arraigados da sociedade não há maiores problemas. Mesmo assim,
violações de tais normas são frequentes e muitas vezes resultam em
protesto. No contexto de assistência médica, por exemplo, pacientes
esperam que os médicos mantenham as informações sobre o seu estado de
saúde confidenciais, ainda que aceitem que essas informações possam ser
compartilhadas com outros especialistas quando necessário. As
expectativas dos pacientes poderiam ser rompidas, e eles provavelmente
ficariam chocados e consternados, se descobrissem que seus médicos
tivessem vendido as suas informações para uma empresa de marketing.
Neste caso, nós diríamos que as normas informacionais foram violados em
razão do contexto em questão -- assistência médica.

Tecnologias de informação e mídia digital há muito tempo têm sido vistas
como uma ameaça à privacidade, porque elas têm rompido radicalmente os
fluxos de informações pessoais. Isso vai desde bancos de dados
corporativos e governamentais da década de 1960 até as câmeras de
vigilância e as redes sociais da atualidade. Em particular, a Internet
tem sido disruptiva e em uma escala e variedade sem precedentes. Para
aqueles que imaginavam que as ``ações online'' seriam sigilosas, eles
estão desiludidos. Por mais difícil que tenha sido circunscrever o
direito à privacidade em geral, é ainda mais complexo enquadrá-lo no
ambiente online em razão da alteração dos destinatários da informação,
dos tipos de informação e das restrições sob a quais essas informações
fluem. Já compreendemos que mesmo quando interagimos com familiares e
conhecidos, terceiros podem estar nos espionando e estarem envolvidos em
parcerias comerciais com nossos grupos de conhecidos. Nossas informações
pessoais que em algum momento permaneceram em arquivos esquecidos, agora
são localizadas em um instante através de consultas de pesquisa por
qualquer pessoa em qualquer lugar. Nestes ambientes altamente
informatizados (ou seja, ricos em informações), novos tipos de
informações são extraídos de cada uma de nossas ações e relacionamentos.

Nós estamos desnorteados com os novos e diferentes tipos de informação
gerados online. Alguns dos subprodutos de nossas atividades incluem
cookies, latências, cliques, endereços de IP, gráficos sociais
reificados e históricos de navegação. Novos e diferentes princípios
governam o fluxo informacional: informações, que compartilhamos como
condição de recebimento de bens e serviços, são vendidas para terceiros;
amigos, que não violariam nossas confidências, repassam nossas
fotografias pela web; as pessoas ao redor do mundo, com quem partilhamos
relacionamentos sem reciprocidade, podem ver nossas casas e carros;
provedores, de quem compramos o serviço de acesso à Internet, vendem os
nossos fluxos de comunicação para os anunciantes de publicidade.
Restrições por padrão do fluxo das nossas informações parecem não
corresponder à lógica social, ética e política, mas à lógica da
possibilidade técnica: ou seja, tudo o que a Internet permite.
Fotografias, curtidas e descurtidas, listas de amigos passam através dos
servidores do \emph{Facebook}, mas não se sabe se essas informações
serão abandonadas em algum momento; se um imperceptível (para o usuário
comum, pelo menos) código \emph{JavaScript} ou \emph{beacon} for
colocado por um site para rastrear a navegação do usuário, assim
acontecerá; se os \emph{Flash cookies} puderem habilmente contornar a
exclusão dos \emph{cookies http}, assim será feito.

A abordagem dominante para enfrentar estas preocupações e alcançar a
``privacidade online'' é uma combinação de transparência e de escolha.
Geralmente chamado como o modelo de ``consentimento informado'', a
essência desta abordagem é informar os visitantes do site e usuários de
serviços e produtos online sobre as respectivas práticas do fluxo da
informação e, em seguida, fornecer a escolha binária entre contratar ou
não contratar. Duas considerações substanciais explicam o apelo a este
tipo de abordagem para os reguladores e os atores regulados. Uma é a
definição comum do direito à privacidade como o direito de controlar
informação sobre si mesmo (autodeterminação informacional). Essa
abordagem de ``transparência e escolha'' parece ser um modelo de
controle, porque permite que os indivíduos avaliem deliberadamente as
opções e, em seguida, decidam livremente consentir ou recusar. O quão
eficientes são na verdade esses modelos não é uma questão que
continuarei a elaborar aqui, porque, qualquer que seja a resposta, ainda
há um problema mais profundo na definição do direito à privacidade como
um direito de controle das informações sobre si mesmo, conforme
discutido longamente em \emph{Privacy in Context}. \textsuperscript{{8}}

Uma segunda consideração é a compatibilidade do modelo ``consentimento
informado'' frente ao paradigma de um mercado competitivo, o qual
permitiria que vendedores e compradores comercializassem bens nos preços
que o mercado determina. Em teoria, os compradores têm acesso às
informações necessárias para tomarem decisões livres e racionais. Como a
informação pessoal é concebida como parte do preço da troca por um
serviço e produto \emph{online}, tudo procede se os compradores forem
informados das práticas do vendedor no que diz respeito à coleta e
utilização das informações pessoais e, em última análise, eles
decidiriam livremente se o preço é justo. O mercado ideal pressupõe
agentes livres e racionais, que tomariam decisões sem interferência de
terceiros (e.g., os reguladores). Isso não só demonstraria respeito
pelos atores, mas também permitiria que o mercado funcionasse de maneira
eficaz e produzisse a maior utilidade geral.

No entanto, há um considerável consenso de que esse modelo de
``transparência e escolha'' falhou. \textsuperscript{{9}} Defensores e
advogados especializados em privacidade, indivíduos e mídias populares
protestaram, de modo mais barulhento e insistentemente, contra as
práticas clandestinas e desenfreadas de coleta de dados, bem como contra
a agregação, análise e criação de perfis. Até mesmo a indústria e
reguladores tradicionalmente pró-negócios admitem que os regimes
existentes não têm feito o suficiente para impedir tais práticas
indesejáveis, como o monitoramento e rastreamento para fins de
publicidade comportamental e a coleta predatória de informações
disponíveis nas redes sociais e por toda a web. A razão pela qual os
modelos da ``transparência e escolha'' ou o ''consentimento informado''
falharam -- e o que fazer sobre isso -- continua a ser uma controversa
acalorada.

Para muitos críticos, a quem chamamos de críticos adeptos, a culpa recai
sobre o regime ubíquo de enquadrar a privacidade como uma escolha de
"pegar ou largar". Uma série de reflexões sobre o assunto (incluindo
aquelas contidas nos relatórios da Comissão Federal de Comércio -
\emph{Federal Trade Commission -- (FTC)} e do Departamento de Comércio
mencionados acima) têm chamado a atenção para processos de tomada de
decisão falhos, bem como informações pouco
esclarecedoras.\textsuperscript{{10}} Uma vez que escolher significa
deliberar e decidir livremente, a prática quase universal de modelos de
escolha de autoexclusão ``\emph{opt out}" pode dificilmente ser
considerada um modelo ideal para que o consumidor tome decisões genuínas
nesse contexto de livre mercado.

Uma questão ética mais profunda é se os indivíduos realmente escolhem
livremente transacionar -- aceitar uma oferta, visitar um site, realizar
uma compra, participar de uma rede social -- dado como estas escolhas
são moldadas, bem como quais são os custos na escolha de não
fazê-lo.\textsuperscript{{11}} Embora possa parecer que indivíduos
livremente optam por pagar com as suas informações pessoais. o preço de
não se engajar socialmente, comercialmente e financeiramente na verdade
pode ser o suficiente para colocar em questão o quão livremente estas
escolhas são feitas.

Políticas de privacidade não são os melhores meios. Por serem quase
todas textos extensos, incompreensíveis e compostos por termos
jurídicos, aumenta-se o fardo -- já irrealista -- de compreender todos
os termos de uso dos sites que visitamos, dos serviços que usamos e dos
conteúdos que absorvemos. Para piorar a situação, uma entidade tem o
direito de mudar sua política unilateralmente, desde que informe tal
mudança ao usuário. Como resultado, exigi-se que as pessoas interessadas
realizem sua leitura não uma, mas várias vezes. Sem surpresa, várias
pesquisas revelam que as pessoas não leem as políticas de privacidade,
quando leem não as compreendem\textsuperscript{{12}} e, realisticamente,
não as conseguiriam ler mesmo se quisessem. \textsuperscript{{13}} Isto
não é uma mera questão de falta de vontade.

Para críticos adeptos do modelo de ``transparência e escolha'', estas
observações apontam para a necessidade de mudança, mas não de revolução.
Tais críticos sugeriram correções que incluem melhores mecanismos para
escolha, assim como a remodelação das políticas em termos de "\emph{opt
in}" (confirmar) ao invés de "\emph{opt out"} (desistir), bem como
identificar momentos de escolha em etapas para habilitar os usuários a
fazer uma pausa e pensar.Eles também defendem o aumento de
transparência: por exemplo, termos de uso mais curtos que são mais
fáceis de seguir, similares a rótulos nutricionais. Sugestões aplicam-se
também ao conteúdo das políticas de privacidade. Se no passado
demandava-se pela simplificação dos contratos, correções atuais
exigiriam os princípios de informação justa.\textsuperscript{{14}} Os
detalhes destas sugestões ultrapassam o escopo deste ensaio, assim como
perguntas sobre como as práticas e políticas de privacidade deveriam ser
monitoradas e aplicadas. Isto é porque (como defendo abaixo), o modelo
de consentimento para a ``privacidade online'' ser respeitada é
atormentado por problemas mais profundos do que quaisquer problemas
práticos observados até agora.

Eu não estou convencida que o modelo de ``consentimento informado'',
mesmo se refinado, resultará em um melhor cenário para a ``privacidade
online''. Pelo menos enquanto continuar a ser um mecanismo
procedimental, divorciado das particularidades e da dinâmica do ambiente
online. Tomemos como exemplo uma publicidade comportamental online que
rapidamente revela uma falha inerente na abordagem do modelo de
``consentimento informado''.\textsuperscript{{15}} Para começar, examine
como os usuários deveriam ser informados a respeito de como seus dados
serão capturados, para onde serão enviados e como serão usados. A
trajetória técnica e institucional é tão complicada que provavelmente
apenas um pequeno grupo de especialistas qualificados seria capaz de
reunir um relato completo. Eu arriscaria dizer que a maioria dos
proprietários não seriam capazes de fazê-lo, apesar de contratarem redes
de publicidade para anúncios direcionados. Mesmo se, por um determinado
momento, fosse possível fotografar o fluxo informacional, a realidade é
que ele está em constante movimentação, com novas empresas entrando em
cena, novas análises e novos contratos de \emph{backend}\footnote{Nota
  do Tradutor: serviço online em que o usuário não interage é
  normalmente acessível somente aos programadores ou administradores}
sendo forjados. Em outras palavras, estamos lidando com uma capacidade
recursiva cuja extensão é indefinível.\textsuperscript{{16}} Como
resultado deste panorama complexo e inconstante, os usuários ficam
propensos a associar (de forma conveniente, mas enganosa) rastreamento
com direcionamento. Tamanha complexidade torna não difícil, mas
impossível compreender quais ações são colocadas em prática e quais
restrições respeitadas. \textsuperscript{{17}}

Para críticos adeptos do modelo de ``consentimento informado'', estes
tipos de casos exemplificam a necessidade de políticas claras e breves
para que as pessoas comuns possam entender o uso que se faz dos seus
dados. Eu vejo isso como um esforço inútil causado pelo o que eu chamo
de \emph{paradoxo de transparência}. Alcançar a transparência significa
transmitir informações relevantes dos procedimentos em questão para que
os indivíduos possam fazer escolhas. Se essa `notificação'' (sob a forma
de uma política de privacidade) detalhar minuciosamente cada fluxo,
condição, qualificação e exceção, nós sabemos que ela será
improvavelmente compreendida e, muito menos, lida. Mas, resumir as
práticas no estilo de rótulos, digamos, nutricionais não será tão útil,
porque isso deixa de fora detalhes importantes, notadamente aqueles que
fazem a diferença: quem são os parceiros comerciais e as informações que
são com eles compartilhadas; quais são os seus compromissos; que medidas
são tomadas para tornar as informações anônimizadas; como as informações
serão mineradas e utilizadas. Uma política com linguagem acessível e
abreviada seria rápida e fácil de ler, mas são os mínimos detalhes que
carregam significados importantes.\textsuperscript{{18}} Daí surge o
paradoxo da transparência: transparência do significado textual e
transparência do conflito prático em termos de absorção da informação.
\textsuperscript{{19}} Parecemos incapazes de alcançar um sem desistir
do outro, no entanto, ambos são essenciais para o modelo de
``consentimento informado'' funcionar.

Os adeptos podem persistir e apontar para outras arenas, tais como
pesquisas sobre assistência médica e que envolvam seres humanos em que
um paradoxo de transparência semelhante parece ter sido superado. Na
assistência médica, protocolos de consentimento informados são comumente
aceitos para comunicar os riscos e benefícios para os pacientes
submetidos à cirurgia. Ou, ainda, assuntos que envolvam programas de
tratamento experimental, mesmo que com a improvável compreensão total
dos detalhes. Em minha opinião, estes protocolos funcionam não porque
eles encontraram a formulação certa de ``notificação'' e o mecanismo
autêntico de consentimento, mas porque eles existem dentro de um quadro
de garantias de apoio. A maioria das pessoas é péssima em avaliar
probabilidades e em compreender os riscos dos efeitos colaterais e as
falhas de procedimentos. Nós somos extremamente fracos na visualização
de órgãos internos do nosso corpo. Não é o formulário de consentimento
em si que recebe nossa assinatura que nos leva para a mesa de operações,
mas sim a nossa fé no sistema.\textsuperscript{{20}} Confiamos nos
longos anos de estudo e aprendizagem a que se submetem os médicos, nas
certificações profissionais e governamentais, no sistema de fiscalização
por pares, nos códigos profissionais e, acima de tudo, no interesse do
sistema (seja qual for a origem) em nosso bem-estar. Acreditamos na
benevolência das instituições de ensino superior e, em grande parte, na
sua missão de promover o bem-estar humano. Longe de serem perfeitos e
sujeitos a violações de alta visibilidade, os sistemas que constituem
essas redes de segurança têm evoluído ao longo dos séculos. Eles
sustentam e garantem os acordos de consentimento que pacientes e
indivíduos enfrentam todos os dias. Por outro lado, no ambiente online,
acordos de consentimento individual devem carregar todo o peso da
expectativa.

Encontrar as lacunas na abordagem problemática do modelo de
consentimento informado não é o ponto final do meu argumento. Do jeito
que está essa pode ser a melhor abordagem para este período provisório
enquanto garantias de apoio desenvolvem-se para sobrepô-la. Tais
garantias não são alcançadas rapidamente, mas podem exigir décadas para
que formas institucionais relevantes e procedimentos progridamem em um
processo de tentativa e erro até chegar a um ponto de equilíbrio. A
teoria da integridade contextual oferece um caminho mais curto e mais
sistemático para tanto, invocando a sabedoria aprendida de sistemas
maduros de normas informacionais que evoluíram para acomodar diversos
interesses legítimos, bem como os princípios morais e políticos gerais
e, por fim, propostas e valores de contextos específicos. A promessa
desse caminho não é meramente que o equilíbrio alcançado em contextos
familiares possa fornecer orientação analógica para os domínios online.
Pelo contrário, o caminho reconhece como o campo online está
inextricavelmente ligados com as estruturas existentes da vida social. A
atividade \emph{online} está profundamente integrada dentro da vida
social em geral e é \emph{radicalmente heterogênea} no sentido de
refletir a heterogeneidade da experiência \emph{offline}.

Agora, a história é familiar: com o advento da \emph{ARPANET} e, à
partir dela, da Internet, o e-mail como o imprevisto "\emph{killer
app"}\footnote{Nota do Tradutor: uma aplicação de computador que supera
  seus concorrentes} forçou a entrega de uma gestão governamental para o
setor privado e o surgimento da \emph{Web} como plataforma dominante
para a experiência das pessoas mais comuns na rede. Ao longo do caminho,
a Internet evoluiu de um utilitário esotérico para o compartilhamento de
recursos de computador e conjuntos de dados, destinado ao uso por poucos
especialistas, para um meio onipresente e multifuncional utilizado por
milhões em todo o mundo. \textsuperscript{{21}} Por progredir através
destes estágios, a Internet tem sido conceitualizada por uma série de
pensadores influentes:\textsuperscript{{22}} de \emph{super estradas da
informação}\textsuperscript{{23}}, permitindo rapidez no fluxos de
informação e do comércio;\textsuperscript{{24}} de ciberespaço, uma nova
fronteira imune das leis de qualquer nação; de Web 2.0, um ponto de
encontro que transborda serviços e conteúdo, muitos dos quais gerados
pelos próprios usuários.\textsuperscript{{25}}

Cada uma destas criações capturou aspectos salientes do vasto sistema
sócio-técnico, o qual tenho chamado de ``\emph{a rede}'' por ter se
desenvolvido através de fases progressivas e que continua a ocorrer até
hoje. De fato, \emph{a rede} é caracterizada pela enorme maleabilidade,
ao longo do tempo e em todas as suas aplicações. Embora o substrato
técnico bruto de mídia digital -- arquitetura, design, protocolo,
características -- possa restringir ou permitir certas atividades, ele
faz mais do que isso. É, digamos, uma força gravitacional que da mesma
forma que restringe e proporciona a atividade humana, também deixa muito
espaço para variação.Por exemplo, a \emph{rede} pode ter parecido
essencialmente ingovernável até que a China impôs o controle e
fronteiras territoriais ressurgiram silenciosamente. mesmo que essa
manobra seja incompleta, deixa-se bolsões de autonomia intactos e
emocionantes.\textsuperscript{{26}}

Uma fotografia da Internet hoje seria uma abstração de camadas técnicas
e sistemas sociais (econômicos e políticos), a qual opera como uma
infraestrutura e um espaço povoado e movimentado. Se "online," "no
ciberespaço", "na Internet" ou "na Web", os indivíduos se envolvem em
práticas banais que poderiam ter sido realizadas por telefone ou por
meio de uma visita em local físico, como, por exemplo, atividades
bancárias, reservas de viagens e compras . Outras atividades -- assistir
filmes, ouvir gravações, ler livros e jornais, falar em um
\emph{IPphone}, buscar informações, comunicação via e-mail, veneração e
algumas formas de compras -- são transformadas ao migrarem para a
\emph{rede}.

Em muitos casos, estas transformações não são meramente experienciais,
mas configuram inovações institucionais, tais como igrejas online e
serviços de namoro, universidades virtuais, ou, ainda, sites como
\emph{Amazon}, \emph{Netflix}, \emph{Mayo Clinic}, \emph{WebMD},
\emph{eBay} e \emph{E*TRADE}.E, por fim, programas e serviços tais como
o \emph{e-government}, \emph{e-zines}, \emph{e-vites}, \emph{e-readers},
\emph{iTunes} e \emph{iShares}.

Mesmo as transformações mais fundamentais e as grandes novidades
encontram-se nas atividades, práticas e formas institucionais e de
negócios que são construídas com base em meta-mecanismos, os quais
agregam, indexam, organizam e localizam sites, serviços, produtos,
notícias e informações. Exemplos incluem o \emph{kayak.com}, o
\emph{Google Search}, o \emph{Google News} e o \emph{Yelp.com}. A Web
2.0 tem forjado uma camada adicional de alterações, particularmente na
produção, criatividade e vida social. Estas alterações incluem interação
através das redes sociais, interconexão em plataformas, a produção por
pares\footnote{Nota do Tradutor: peer-production refere-se a qualquer
  produção coordenada, com base na internet, pela qual voluntários
  contribuem como componentes do trabalho e existe um processo para
  combiná-los e produzir uma obra intelectual unificada} e conteúdo
gerado pelo usuário por meio de inumeráveis blogs, \emph{wikis} e web
sites pessoais e, por fim, repositórios de escala global: a
\emph{Wikipedia}, \emph{IMDb, mmorgs} (gigantesco \emph{multiplayer}
online), \emph{Flickr} e \emph{YouTube}; a comunidade online de suporte
aos pacientes - \emph{PatientsLikeMe}; assim como \emph{mash-ups},
\emph{folksonomies}, \emph{crowdsourcing} e sistemas de reputação.

Perguntas sobre como proteger a privacidade, particularmente quando
formuladas com o adjetivo ``online'', sugerem que "online" é um local,
esfera, lugar ou espaço distinto que é definido pelas infraestruturas
tecnológicas e protocolos de rede. Por essa razão, um conjunto singular
de regras de privacidade poderia ou deveria ser elaborado. Eu resisti a
essa noção. Ainda que com a excitante visão do ciberespaço como uma nova
fronteira, a experiência revela que nenhum domínio isolado divorciou-se
da ``vida real'' e mereceu uma regulamentação distinta. A rede não
constitui um contexto diferente (valendo-se da terminologia da
integridade contextual). Não é um espaço social singular, mas a
totalidade da experiência conduzida via \emph{rede}. Isso vai desde
sites específicos para dispositivos de busca e plataformas até chegar à
"nuvem", cruzando vários campos. Atividades online mediadas pela rede
("dentro" da Web) estão profundamente integradas na vida social. Elas
podem ser contínuas como correlações tradicionais ou, pelo menos, têm o
poder de afetar as comunicações, transações, interações e atividades
nessas diferentes arenas (e vice-versa).Não se trata apenas da
integração da vida online à vida social e, por conseguinte, produto de
um contexto diferente..Ela é radicalmente heterogênea, composta por
múltiplos contextos sociais, que não apenas um e que certamente não é
apenas o contexto comercial pelo qual proteger privacidade equivaleria a
proteger a privacidade do consumidor e das informações
comerciais.\textsuperscript{{27}} Certamente, os contornos da tecnologia
(arquitetura, protocolo, \emph{design} e assim por diante) moldam o que
você pode fazer, dizer, ver e ouvir \emph{online.} Ao mesmo tempo,
contudo, alterações ou interrupções, devido às características
particulares da rede, impõem enigmas e apresentam desafios para
contextos sociais, o que não garantem regras \emph{sui generis},
uniformes e transversais determinadas pela mídia. Em vez disso, os
contextos nos quais as atividades são baseadas formam expectativas que,
quando não satisfeitas, causam ansiedade, medo e
resistência.\textsuperscript{{28}}

Responder a perguntas sobre ``privacidade online'', tais como aqueles
sobre privacidade em geral, nos obriga a prescrever restrições adequadas
ou apropriadas sobre o fluxo das informações pessoais. O desafio da
``privacidade online'' não acontece porque o local é distinto e
diferente ou porque os requisitos de privacidade são distintos e
diferentes, mas porque, enquanto agimos, interagimos e transacionamos
online, essa mediação pela rede leva à perturbações na captura, análise
e disseminação de informações. A decisão heurística derivada da teoria
de integridade contextual sugere que podemos localizar contextos,
explicar normas informativas arraigadas, identificar fluxos disruptivos
e avaliar esses fluxos face a normas com base em princípios éticos e
políticos, bem como em finalidades e valores de contextos específicos.

Certamente, localizar contextos online e explicitar as normas que os
presidem nem sempre é uma tarefa simples (da mesma forma que não é
simples quando se lida com espaços sociais sem essa intermediação pela
rede). No entanto, alguns dos casos mais familiares podem fornecer maior
clareza para a tarefa. Se você transacionar com o seu banco online, ao
telefone ou pessoalmente em uma filial, é razoável esperar que as regras
que regem a informação não mudaram de acordo com o meio. Nos Estados
Unidos, bancos e outras instituições financeiras são regulamentados
pelas regras de privacidade formuladas pela FTC, quem está embuída dessa
competência pela lei \emph{Gramm-Leach-Bliley
Act}.\textsuperscript{{29}} Informações auxiliares (por exemplo,
endereço do IP ou \emph{histórico de navegação}), os registros das
transações eletrônicas, não devem ser simplesmente concebidos para serem
"disputados" só porque essas informações não foram consideradas
explicitamente nas regras formuladas antes do banco online tornar-se
comum. Em vez disso, devem haver os mesmos padrões que orientaram a
privacidade financeira originalmente.

Da mesma forma, as expectativas dos visitantes de Bloomingdales.com,
NYTimes.com e MOMA.org que são afetadas pelas marcas correspondentes
preexistentes. Elas também são formatadas pelos respectivos contextos
sociais no quais tais entidades estão apoiadas, incluindo os tipos de
experiências e ofertas que eles prometem.

Nesse sentido, o caso da Amazon.com que, apesar de ter entrado em cena
como uma livraria online sem qualquer precursor tradicional, é parecido
com, digamos, a livraria Morávia, em
Bethlehem/Pensilvânia.\textsuperscript{{30}} Como a Amazon.com
expandiu-se para outros campos, vendendo e alugando DVDs, supõe-se que
os fluxos de informações pessoais gerados nas operações sejam regulados
por restrições expressadas no vídeo \emph{Privacy Protection Act} de
1988 \textsuperscript{{31}} da mesma forma que a White Coast Vide está
sujeita a tal lei. Se é incerto que a legislação aplicável às locadoras
de vídeo tradicionais deveria ser também aplicada aos provedores de
aluguel de vídeo online (e.g., \emph{iTunes} e a \emph{Amazon)}; os
requisitos de integridade contextual, que ancora a regras de privacidade
em contextos sociais e papéis sociais, sugere o contrário.

Esses exemplos apenas delineiam superficialmente a heterogeneidade
marcante da \emph{rede}. A gama de ofertas online vai desde provedores
de informações especializadas e distribuidores, como a
\emph{MayoClinic.com} e a \emph{Web-MD} até portais do governo federal,
estadual e local, fornecendo informações e serviços diretamente aos
cidadãos, e ainda repositórios estruturados de conteúdo gerados pelos
usuários, tais como a \emph{Wikipedia}, o \emph{YouTube}, o
\emph{Flickr}, o \emph{Craigslist} e as redes sociais, incluindo o
\emph{Facebook}. Denominações religiosas ao redor do globo têm presenças
online que vão desde a Santa Sé, alegando ser o site de "oficial" do
Vaticano,\textsuperscript{{32}} até igrejas online.
\textsuperscript{{33}} Eles oferecem compromisso religioso online que
substitui ou complementa a regular frequência à igreja. Esta lista não
captura a fluidez e a modularidade das ofertas existentes, o que incluem
combinações e permutações (\emph{mash-ups}) limitadas apenas pela
criatividade humana e os limites tecnológicos do momento. Muitos sites
populares, por exemplo, combinam módulos de conteúdo gerados pela
plataforma com feedbacks gerados pelos próprios usuários, ou, ainda, com
o conteúdo das redes sociais de blogs abertos e muito mais.

Na medida em que a rede está profundamente imbrincada na vida social,
normas informacionais de contextos específicos devem ser ampliadas para
as atividades online correspondentes. Assim, as regras de privacidade
que regem as instituições financeiras poderiam ser ampliadas para o
\emph{E*TRADE} , mesmo ele operando principalmente através de um portal
online. As ofertas e experiências online podem desafiar as normas
existentes, mas quando elas incorporam algumas das formas de abordagem
mencionadas acima. Em tais circunstâncias, a teoria da integridade
contextual direciona-nos das normas existentes para os padrões
subjacentes que derivam de considerações morais e políticas gerais, bem
como dos propósitos, finalidades e valores de seus respectivos
contextos. \textsuperscript{{34}} Novas atividades e práticas, que
implicam em diferentes tipos de informação, expandiram os grupos de
destinatários e restrições de fluxo alteradas devem ser avaliadas em
relação a estas normas.

Aplicar este raciocínio para formulários online de restituição de
imposto de renda é bastante simples. Nos Estados Unidos, requisitos
rigorosos de confidencialidade, desenvolvidos ao longo dos últimos 150
anos, regem os registros fiscais individuais que são intransponíveis até
para algumas necessidades de aplicação de lei. \textsuperscript{{35}}
Embora o atual código, formalizado na década de 1970, pode ter pouco a
dizer sobre especificamente o arquivamento eletrônico, nós não podemos
esperar que informações complementares de interações online estejam
"prontas para serem acessadas", bem como disponíveis livremente por
todos os cantos. Mesmo na ausência de regras explícitas, orientações
podem ser extraídas com base nos valores e propostas resultantes das
regras de confidencialidade existentes dos tradicionais registros
fiscais impressos. Com base na Lei de Divulgação e Privacidade, a
Receita Federal afirma que "deve haver confiança do público no que diz
respeito a confidencialidade das informações pessoais e financeiras, que
nos foram dadas para fins de administração do imposto... A natureza da
confidencialidade desses registros requer que cada pedido de informações
seja avaliado à luz de uma considerável instância de leis e regulamentos
que autorizam ou proíbem a publicação." \textsuperscript{{36}} Esta
conexão foi reconhecida já em 1925, quando o Secretário do Tesouro
Andrew Mellon comentou: "Embora o governo não tenha como saber todas as
fontes de renda de um contribuinte e precise contar com a sua boa fé nas
declarações de renda, ainda assim, essa dependência é inteiramente
justificável na grande maioria dos casos. Isto porque, o contribuinte
sabe que ao fazer uma declaração verdadeira de suas fontes de sua renda,
tais informações ficaram detidas com o governo. É como confiar em de um
advogado.'' \textsuperscript{{37}} Uma presunção de estrita
confidencialidade é derivada de valores e propósitos desse contexto -- o
cumprimentos das normas, a fiabilidade e a confiança no governo -- que
proíbem todos os compartilhamentos desses dados, exceto os permitidos,
caso a caso, através de leis e regulamentos explícitos.

Os casos mais desafiadores que nos confrontam incluem formas de
conteúdo, serviços e interação que são específicos para a rede ou que
não têm correspondentes óbvios em outro lugar. Dispositivos de busca
como o \emph{Google} e o \emph{Bing}, essenciais para a navegação na
rede, podem constituir uma importante categoria de casos. Enquanto sites
tais como o \emph{Mayo Clinic} e o \emph{WebMD} podem se assemelhar ao
do ambiente familiar; sites de informação de saúde são prodigiosos,
\textsuperscript{{38}} oferecendo de tudo, desde serviços personalizados
e interativos que permitem aos usuários entrar e fazer perguntas sobre
seus problemas específicos até repositórios de dados médicos pessoais
que fornecem espaço para armazenar registros de saúde (por exemplo, o
\emph{Microsoft Health Vault} ); e, por fim, sites de redes sociais
dedicadas às comunidades de companheiros de infortúnio (por exemplo, o
\emph{PatientsLikeMe}).

Sem negar que a rede tem produzido muito do que é novo e estranho,
incluindo novos tipos de informações e novos arranjos institucionais, as
atividades online são estranhamente familiares: conectar-se com um
amigo, colaborar em uma missão política, procurar trabalho, buscar ajuda
religiosa ou espiritual, procurar oportunidades educacionais, assistir
notícias locais e mundiais, escolher um livro, uma música para relaxar
ou filmes para assistir. Embora, realizar uma pesquisa no Google é
diferente de procurar material em um catálogo de biblioteca, em parte
porque o conteúdo da rede é bastante diferente do conteúdo de uma
biblioteca, há semelhança entre essas duas atividades: ambas têm como
matriz a busca por conhecimento e pelo enriquecimento intelectual. Em
todas essas atividades, as sociedades democráticas liberais asseguram
que essa liberdade não estará sujeita ao olhar atento ou à aprovação das
autoridades, o que não inibe os cidadãos em procurar por informações
políticas, religiosas ou de outros tipos de afiliação (e.g., sindical).
Esperamos que haja simetria entre o ambiente offline e online nesse
sentido, de modo que os passos online não sejam gravados e registrados a
fim de de minimizar o risco de interferências, por qualquer ser humano
ou máquina.

O interesse na ``privacidade online'' demonstrado por parte da Comissão
Federal de Comércio - \emph{Federal Trade Commission -- (FTC)} e do
Departamento de Comércio é positivo, particularmente porque reconhece
uma preocupação crescente sobre privacidade e amplifica a discussão
pública sobre a coleta descontrolada e desenfreada de informações
pessoais por agentes não governamentais. Porém, esse interesse tem
encarado a proteção da privacidade como uma questão meramente de
proteção do consumidor. Ou seja, que restringe a proteção de informações
pessoais somente no âmbito das transações comerciais
online.\textsuperscript{{39}} Nenhuma dessas agências reconheceu
explicitamente o vasto panorama de atividades situadas fora do domínio
das relações de consumo. Enquanto o discurso público sobre ``privacidade
online'' estiver tomado pelos valores do mercado e do comércio, qualquer
tipo de concepções a seu respeito será inadequado. Precisamos levar
plenamente em conta a radical heterogeneidade das atividades e práticas
online.

Alguém poderia argumentar que a \emph{rede} é quase completamente
comercial, apontando para a prevalência de transações econômicas como
forma de viabilizar os produtos e serviços online.. À margem das
presenças governamentais, a \emph{rede} é quase integralmente controlada
pela iniciativa privada, por entidades com fins lucrativos, desde a
infraestrutura física até à camada de aplicação (e.g., e-commerce e
demais provedores de conteúdo e serviços). Este modelo é o que
prevalece, de grandes sites corporativos à blogs pessoais, tais como
\emph{Noob Cook} que é um site com dezessete rastreadores; ou como o
\emph{Dictionary.com} que conta com nove rastreadores de empresas de
publicidade (e.g., \emph{Doubleclick}, o \emph{Media Math} e o
\emph{Microsoft Atlas} entre outros).\textsuperscript{{40}} Vale
registrar, que o \emph{Wikipedia}, apoiado pela fundação sem fins
lucrativos \emph{Wikimedia}, não apresenta
rastreadores.\textsuperscript{{41}}

Por essa lógica, o Departamento de Comércio e a Comissão Federal de
Comércio - \emph{Federal Trade Commission -- (FTC)} seriam precisamente
quem avocariam tal competência regulatória, além de fazer também sentido
que o mercado fosse livre para se autorregular. Ainda assim, tais
relações privadas e de consumo, não devem se render apenas aos desejos
do próprio mercado. De acordo com a filósofa política Elizabeth
Anderson, muitas funções na sociedade transpuseram os limites entre o
comercial e o não comercial. Exemplo disso é a necessidade de se
garantir competitividade no mercado e a existência de padrões de
qualidade ou excelência. Não há sujeição total às normas do mercado; em
vez disso, espera-se que serviços como educação, saúde, religião,
telecomunicações e transporte, se executadas pelo Estado ou pelo setor
privado, sejam satisfatórios. Como Anderson alerta, "Quando os
profissionais vendem seus serviços, ele se deparam com normas que
potencialmente entram em conflito com as suas próprias normas de
excelência."\textsuperscript{{42}} Nós esperamos mais deles -- de
médicos, advogados, atletas, artistas, religiosos e professores -- do
que somente a busca pelo lucro. Quando as pessoas investem em cuidados
médicos, em educação e em uma variedade de outras instituições, a
rendição completa às normas do mercado resultaria em uma prática
corrupta e empobrecida. Anderson defende um equilíbrio adequado das
normas do mercado e dos seus padrões internos de excelência.

Este ponto pode parecer óbvio, mas algumas ideias de livre mercado e do
capitalismo tornam fácil confundir a busca pelo lucro pela busca por
padrões internos de excelência.\textsuperscript{{43}} Quando Sergey Brin
e Larry Page lançaram o mecanismo de busca do Google, eles consideravam
que influências comerciais seriam contrárias à missão fundamental dessa
ferramenta, a qual deveria servir e ser orientada pelos interesses dos
indivíduos em localizar informações na rede. Contestando o modelo de
negócio baseado em publicidade, eles escreveram no apêndice de um artigo
de 2007, "os objetivos do modelo de negócios publicitários nem sempre
correspondem ao fornecimento de uma pesquisa de qualidade para os
usuários... Acreditamos que a questão publicitária já causa tantos
incentivos mistos, que é crucial ter uma ferramenta de busca competitiva
e também transparente na esfera acadêmica."\textsuperscript{{44}} Em
outros casos menos visíveis, preocupações semelhantes podem ser
levantadas: por exemplo, a compra pela \emph{Amazon} do \emph{IMDb} - um
site de informações sobre filmes desenvolvido e inicialmente mantido por
voluntários. Mesmo no familiar caso de uma biblioteca que empresta
livros, originalmente concebida por Benjamin Franklin e com
financiamento público nos Estados Unidos,, muitas de suas funções
acabaram sendo privatizadas para empresas lucrativas
online.\textsuperscript{{45}}

O ponto aqui não é enxergar Brin e Page ou outros desenvolvedores como
"vendidos". Confundir fontes de patrocínio com orientação de normas
obscurece nosso reconhecimento por padrões internos de excelência, pelos
quais podemos alçar empresas que procuram suporte comercial
independentemente de seu desempenho no mercado
interno.\textsuperscript{{46}} O mesmo argumento se aplica a
fornecedores de conteúdo e serviços de informação que disponibilizam no
setor privado muitos serviços também prestados por bibliotecas públicas.
Não estou sugerindo que exista um consenso ou que questões sobre normas
internas de excelência são facilmente resolvidas. As lutas intermináveis
sobre o que constitui um bom jornal, escola ou sistema de saúde atestam
isso. Mas, eles também revelam uma forte crença de que, além do lucro,
essas normas estão em jogo e são socialmente importantes.

A atenção recente dada ao desafio de proteger a ``privacidade online'' é
um desenvolvimento positivo. No entanto, o sucesso fica entravado pela
inércia daqueles cujo poder e lucro são deoendentes do fluxo irrestrito
de informações pessoais, o que também é limitado por concepções
subdesenvolvidas a respeito da privacidade e o papel que desempenham na
manutenção da rede como um bem público, capaz de servir a diversos
interesses. As descrições iniciais do ciberespaço como uma nova
fronteira, diferente, distinta e fora do alcance do direito tradicional,
foram abandonadas em grande parte. No entanto, nenhuma outra única visão
capturou a imaginação pública da mesma forma. Isto é surpreendente dado
o crescimento maciço da \emph{rede} como uma infraestrutura complexa, um
sistema de entrega de conteúdo e de espaço na mídia. A falta de uma
visão pública dominante significou que assuntos morais e políticos
controversos são resolvidos mais frequentemente por meios técnicos do
que por princípios morais e políticos claramente articulados. Para a
privacidade isso foi devastador, uma vez que a \emph{rede}, construída
por meio de uma amalgamação das ciências e tecnologias de informação,
computação e redes, permitiu interrupções radicais do fluxo de
informação. Com incentivos econômicos e sociais articulados contra a
restrição desse fluxo informacional, é improvável conseguir sucesso ao
se sobrecarregar os indivíduos com o todo o peso da proteção dos seus
dados através do modelo da ``consentimento informado''.

Minha alternativa preferencial constrói-se a partir da visão da vida
online como algo heterogêneo e densamente integrado com a vida social.
Apesar das qualidades singulares dos seus movimentos, transações,
atividades, relações e relacionamentos, quando abstraídas de tais
particularidades, elas se mantém fiéis aos princípios fundamentais de
organização da prática humana e da vida social. Baseando-se no trabalho
proveniente da filosofia e da teoria social, a estrutura da integridade
contextual concebe tais esferas como parcialmente constituídas por
normas de comportamento, entre elas normas que regem o fluxo das
informações pessoais (compartilhamento, distribuição). Não devemos
esperar que as normas sociais, incluindo as normas informacionais,
simplesmente desaparecerão com a mudança do meio físico para o ambiente
digital eletrônico, tal qual ondas sonoras para partículas de luz.
Embora o meio possa afetar quais ações e práticas são possíveis e,
provavelmente, a formulação de políticas sensatas com enfoque em ações e
nas próprias práticas sociais. Deve-se ter em mente a sua função dentro
das esferas sociais e a sua posição em relação às normas sociais
enraizadas.

Duas recomendações gerais seguem os argumentos formulados até agora. Em
primeiro lugar, em nossa atividade online devemos nos guiar pelos
contornos das atividades sociais e estruturas familiares. Esta não é uma
tarefa difícil para a maior parte do que fazemos online -- operações
bancárias, compras, comunicação e experiências culturais e de
entretenimento. Onde as semelhanças são menos óbvias, como procurar por
um determinado material nos mecanismos de busca para localizá-lo na
rede, nós deveríamos considerar analogias próximas com base não tanto na
similaridade de ação, mas na similaridade de função ou propósito. Neste
sentido, consultar um material nos buscadores é semelhante à realização
de pesquisas em um catálogo de biblioteca, buscando-se informações para
enriquecimento intelectual. O tempo gasto em redes sociais, como o
\emph{Facebook}, é um amálgama de envolvimento com a vida pessoal,
social, íntima e doméstica, bem como de associações políticas e
profissionais. Quando identificamos as semelhanças, nós trazemos a tona
normas relevantes que governam o nosso fluxo informacional. Se eu
estiver certa sobre como as ferramentas de busca são usadas e para quais
finalidades, então as normas em questão indicariam para a estrita
confidencialidade no que diz respeito aos históricos de busca, tal como
os históricos de buscas de muitas bibliotecas públicas. Como resultado,
haveria a pronta extinção de tais registros para minimizar os riscos de
vazamento, bem como a tentação de compartilhá-los futuramente para
ganhos financeiros. Até o momento, o Google, ao contrário de outros
provedores de pesquisa, manifestou o compromisso de manter uma barreira
entre registros identificáveis de pesquisa e outros registros associados
aos perfis dos usuários. Embora esta decisão alinhe-se com o espírito da
concepção do mecanismo de busca antes mencionado, algumas dúvidas
remanescem sobre a eficácia dessa abordagens de \emph{logs} de busca
deidentificados. Além disso, há o fato de que o compromisso pode ser
revogado a qualquer momento, tal como foi com o compromisso do Google em
rejeitar anúncios de publicidade comportamental. \textsuperscript{{47}}

Este ponto de vista sobre a privacidade online também implica que os
contextos, e não a economia política, devem determinar quais os tipo de
restrições sobre o fluxo de informações. Empresas se fundem e adquirem
outras empresas, por muitas razões diferentes. Por exemplo, para
fortalecer e expandir a sua gama de participações financeiras, para
ganhar o domínio do mercado em uma área particular ou para estabelecer
controle sobre a cadeia vertical dos recursos necessários. Entre os
ativos valiosos que motivam aquisições estão os bancos de dados de
informações pessoais, como demonstrado (presumivelmente) pela aquisição
do \emph{DoubleClick} pelo Google e pelas aquisições sistemáticas de
pequenas empresas com as suas respectivas bases de dados pela
\emph{ChoicePoint}.\textsuperscript{{48}} Contudo, os bancos de dados de
informações pessoais estruturados em um determinado contexto, sob as
restrições das normas informacionais a ele pertinentes, não devem ser
equiparados como apenas outro ativo (e.g., semelhante aos edifícios,
mobiliário e suprimentos). As políticas de privacidade de diversas
empresas, tais como Walt Disney, General Electric, Google, Citigroup,
Viacom e Microsoft revelam, no entanto, fronteiras porosas entre as suas
subsidiárias. Há pouco reconhecimento da necessidade de se explicar os
padrões de fluxo de informações dentro de uma única empresa. Com os
conglomerados online não é diferente, em vista dos seus esforços para
atingir a integração vertical através do controle das matérias-primas de
sua indústria, a saber, informações. Contra essas tendências, nós
devemos estabelecer o respeito pelos limites do contexto e as normas
informacionais correspondentes.

Há pouca dúvida de que, ao se comunicar com o público, corporações
compreendem a importância de reconhecer a integridade dos contextos.
Mesmo que para investidores corporativos uma empresa possa se vangloriar
de diversos ativos de informação, ao público notam-se unidades que são
socialmente significativas. Pode ser que nestes atos de
autoapresentação, empresas reconheçam a heterogeneidade contextual de
suas atividades e, por conseguinte, abram as portas às normas
correspondentes aos contextos específicos e respectivos de suas
atividades.Uma universidade, uma loja de sapatos, uma igreja, um centro
médico, uma rede de amizades ou um banco ao anunciar suas atividades
online oferece aos seus usuários uma maneira de compreendê-los e
avaliá-los de acordo com as respectivas normas informacionais de cada um
dos seus contextos, pouco importando se elas estão incorporadas em lei
ou decorrente das expectativas razoáveis do contexto dessa
relação.\textsuperscript{{49}} Permanecer fiel a estes auto-retratos
exige que as empresas se comprometam em segmentar ativos de informações
ao longo de contornos contextuais, ao invés de ao longo de toda a
extensão das linhas da propriedade corporativa.

Não podemos negar os efeitos transformadores das tecnologias digitais,
incluindo as valiosas e fervilhantes atividades online dela resultantes.
Correlacionar, por analogia, os contextos sociais online e offline e
fazer o mesmo, quando recomendável, com as ofertas e relações comerciais
não é um conflito. Em vez disso, o objetivo é revelar as normas
informacionais relevantes de excelência. Assim como regras, atividades,
funções e contextos sociais migram online, os respectivos valores, fins
e propósitos de contextos específicos servem como normas informacionais
contra as quais as práticas de compartilhamento de informações podem ser
avaliadas como legítimas ou problemáticas. É importante ter consciência
de que as normas de privacidade não protegem apenas indivíduos, elas
também desempenham um papel crucial na manutenção das instituições
sociais.\textsuperscript{{50}} Nesse sentido, as restrições sobre os
aplicativos de busca ou sobre as redes sociais não são somente sobre
sustentar valores sociais importantes ao crescimento intelectual,
criatividade e engajamento social e político animado, mas, também, sobre
como proteger os indivíduos contra danos. Benjamin Franklin sabia disso
quando insistiu na proteção de privacidade nos Correios dos Estados
Unidos. Ele objetivava não só proteger os indivíduos, mas, também,
promover um papel social significativo decorrente deste serviço. Não
devemos esperar menos da telefonia e dos correios eletrônicos.

Minha segunda recomendação aplica-se aos casos online sem precedentes
sociais claros. Como discutido anteriormente, as formas sociais online
às vezes permitem configurações de agentes, informações, atividades e
experiências que são desconhecidas, pelo menos à primeira vista. Nestes
casos, sugiro começar com os propósitos, finalidades e valores em jogo e
assim voltar às normas informacionais. O site de um político que permite
aos cidadãos com ele interagir configura-se como uma plataforma incomum
para a qual regras preexistentes no tocante as pegadas digitais deixadas
pelos visitantes nesse site não se aplicam, por exemplo. Aqui, a
abordagem correta é não haver uma caixinha de surpresas com informações
oportunistas para rastrear os cidadãos. Embora isto possa servir para
necessidades imediatas de uma iminente campanha política, não serve para
a proposta de encorajar a franca discussão política, o que
reconhecidamente floresce em ambientes de grande
liberdade.\textsuperscript{{51}} Se as pessoas esperam ser monitoradas e
anteveem que seus pontos de vista serão gravados e compartilhados com
terceiros por dinheiro ou favores, é bem provável que elas se tornem
mais vigilantes, circunspectas ou não cooperativas. No entanto, a
questão não é como tais práticas particulares afetam os indivíduos, mas
as implicações para metas e valores em geral. Circunspecção e cooperação
são úteis para certos fins, mas não para outros. Ao trabalhar pelo
retrocesso destes valores, nós desenvolvemos regras para situações nas
quais parece não haver quaisquer candidatos óbvios.

O ímpeto crescente para enfrentar o problema da privacidade online é
algo positivo. No entanto, seria um erro procurar soluções que tornariam
a privacidade online algo singular. Proteger a privacidade trata-se de
assegurar que haja um fluxo adequado das nossas informações pessoais,
tanto online quanto offline. Interrupções desse fluxo adequado pelas
tecnologias de informação e da mídia digital podem ser igualmente
perturbadoras, se online ou offline, porque muito do que acontece online
é amplamente integrado com a vida social offline (e vice-versa), de modo
que resolver o problema de privacidade online requer uma abordagem
totalmente integrada. Eu articulei um passo em direção a este objetivo,
resistindo à sugestão de que, no que diz respeito a privacidade, a
\emph{rede} seria um território virgem, onde cabe às partes construir os
termos de compromisso para cada transação. Tendo em conta como são
profundamente enraizadas as nossas expectativas do que é certo e errado
com relação ao compartilhamento das nossas informações pessoais e de
nossos pares, não é de admirar que, ao longo do tempo, intrincados
sistemas de normas informacionais desenvolver-se-ão para governar todos
os domínios da vida social.

Para adaptar esses sistemas às relações sociais e aos contextos
expandidos da mídia digital, nós temos de tornar explícito o que até
então operou implicitamente. Devemos rejeitar as normas enraizadas que
já não promovem os valores morais e políticos, bem como os fins
esperados de um contexto específico. Permitir que a proteção da
privacidade online fique limitada às negociações do modelo de
``consentimento informado'' não é apenas injusto, mas é, também deixar
passar uma oportunidade de formular uma política pública crítica que
terá implicações para a forma e o futuro da \emph{rede}. Se exercido de
forma consciente, o processo de articular regras baseadas em contextos e
expectativas, inserindo algumas delas nas leis e em outros códigos
especializados terá o efeito de produzir laços de segurança e confiança
que suportemo consentimento como em campos tais como da pesquisa e
assistência médica. Com estas precauções em mente, ainda existirá muito
espaço para expressar preferências pessoais e manter um papel mais
robusto e significativo para o consentimento informado.

\section{Notas finais}

1 Este ensaio beneficiou-se das oportunidades de apresentação no
\emph{Center for Law, Technology, and Society (}Centro para Direito,
Tecnologia e Sociedade), na Universidade de Ottawa; no \emph{Center for
the Study of Law and Society} (Centro de Estudos de Direito e
Sociedade), na Universidade da Califórnia, em Berkeley; e no
\emph{Center for Internet and Society} (Centro para Internet e
Sociedade), na Universidade de Stanford, onde perguntas e comentários
levaram a melhorias e refinamentos significativos ao argumento. Estou
grata pelo retorno valioso de David Clark e do grupo de pesquisas de
privacidade da NYU (New York University), pela orientação especializada
de Cathy Dwyer e Foster Provost e a excelente assistência na pesquisa
por Jacob Gaboury e Marianna Tishchenko. Este trabalho foi apoiado pelos
subsídios AFSOR: ONRBAA 07 / 03 (MURI) e NSF CT-M: Privacidade,
Conformidade \& Informação de Riscos, CNS-0831124.

2 Jennifer Valentino-Devries, ``\emph{What They Know About You}'' (O Que
Eles Sabem Sobre Você), The Wall Street Journal, 31 de julho de 2010,
\emph{http://online.wsj.com/article/SB10001424052748703999304575399041849931612.html}\emph{http://online.wsj.com/article/SB10001424052748703999304575399041849931612.html}.

3 U.S. Federal Trade Commission (Comissão Comercial Federal dos Estados
Unidos),

Relatório pessoal preliminar da FTC, ``Protecting Consumer Privacy in an
Era of Rapid Change: A Proposed Framework for Businesses and
Policymakers'' (Protegendo a privacidade do consumidor em uma Era de
rápidas mudanças: A Abordagem proposta para empresas e formuladores de
políticas), de dezembro de 2010,
\emph{http://www.ftc.gov/os/2010/12/101201privacyreport.pdf}.

4 U.S. Department of Commerce (Departamento do Comércio dos EUA),
``Commercial Data Privacy and Innovation in the Internet Economy: A
Dynamic Policy Framework'' ( Dados Comerciais de Privacidade e Inovação
na Economia da Internet: A Dinâmica Política Estrutural), de dezembro de
2010,
\emph{http://www.ntia.doc.gov/reports/2010/IPTF\_Privacy\_GreenPaper\_12162010.pdf}

5 Comparar essas representações com as contas anteriores da Internet
como uma nova fronteira da liberdade e da autonomia: por exemplo, David
R. Johnson e David G. Post, ``Law and Borders -- The Rise of Law in
Cyberspace'' (Leis e Limites -- O Surgimento das Leis no Espaço
Cibernético), \emph{Stanford Law Review} 48 (1996): 1367; John Perry
Barlow, ``Electronic Frontier: Coming into the \emph{Country}''
(Fronteiras Eletrônicas: Chegando ao País) \emph{Communications of the
ACM} 34 (3) (Março de 1991).

6 Neste ensaio, eu extrai a maioria dos meus exemplos da \emph{World
Wide Web} (Ampla Rede Mundial), porque quase todas as preocupações
controversas sobre privacidade que chamaram a atenção do público são
resusltantes da atividade baseada na rede; e, porque em experiências
online das pessoas comuns ocorrem principalmente na rede. Usarei o termo
rede quando as observações feitas sobre a Web parecerem pertinentes para
outros serviços e aplicações da internet.

7 Helen Nissenbaum, \emph{Privacy in Context: Technology, Policy, and
the Integrity of Social Life} (Privacidade no Contexto: Tecnologia,
Política e a Integridade da Vida Social) - (Stanford, Califórnia:
Imprensa da Universidade de Stanford, 2010).

8 Idem, capítulo 5.

9 \emph{Federal Trade Commission} (Comissão Federal de Comércio),
``\emph{Protecting Consumer Privacy in an Era of Rapid Change}''
(Proteger A Privacidade Do Consumidor Em Uma Era De Rápida Mudança) e

\emph{Department of Commerce} (Departamento De Comércio),
``\emph{Commercial Data Privacy and Innovation in the Internet
Economy}'' (A Privacidade dos dados comerciais e A Inovação na economia
da Internet).

10 Fred Cate, ``\emph{The Failure of Fair Information Practice
Principles}''(O fracasso dos princípios das práticas de informação
justa) in \emph{Consumer Protection in the Age of the ``Information
Economy'' (}Defesa do consumidor na era da "Economia da
informação)\emph{,} ed. Jane K. Winn (London: Ashgate Publishing, 2006).

11 Ian Kerr, ``\emph{The Legal Relationship Between Online Service
Providers and Users'' (}A Relação Jurídica Entre Provedores De Serviço
Online E Os Usuários), \emph{Canadian Business Law Journal} 35 (2001):
1--40.

12. Joseph Turow, Lauren Feldman e Kimberly Meltzer, ``\emph{Open to
Exploitation: American Shoppers}

\emph{Online and Offline'' (}Abertos à exploração: compradores
americanos Online e Offline) (Philadelphia: \emph{Annenberg Public
Policy Center} (centro de políticas públicas Annenberg), Universidade da
Pensilvânia, 1 de junho de 2005),
\emph{http://www.annenbergpublicpolicycenter.org/Downloads/Information\_And\_Society/Turow\_APPC\_Report\_WEB\_FINAL.pdf}

13 Lorrie Faith Cranor e Joel Reidenberg, ``Can User Agents Accurately
Represent Privacy

Notices?'' (Agentes de usuários podem fielmente representar avisos de
privacidade?) \emph{The 30th Research Conference on Communication,
Information, and Internet Policy} (30ª Conferência de pesquisas sobre
comunicação, informação e políticas de Internet) (TPRC2002), Alexandria,
Virgínia, 28 a 30 de setembro de 2002.

14 U.S. Department of Health, Education, and Welfare (Departamento de
Saúde, Educação e Bem-Estar dos Estados Unidos), \emph{Records,
Computers, and the Rights of Citizens (}Registros, Computadores e os
Direitos dos Cidadãos), relatório do Comité Consultivo da Secretaria
para a automatiçazão de sistemas de dados pessoais, de julho de 1973,
\emph{http://aspe.hhs.gov/datacncl/1973privacy/tocprefacemembers.htm}.

15 Solon Barocas e Helen Nissenbaum, \emph{``On Notice: The Trouble with
Notice and Consent''} (Aviso: O Problema com o ``Aviso e
Consentimento''), \emph{Proceedings of the Engaging Data Forum: The
First International Forum on the Application and Management of Personal
Electronic Information} (Fórum sobre Procedimentos dos Dados Engajados:
O Primeiro Fórum Internacional sobre a Aplicação e Gerenciamento de
Informações Eletrônicas Pessoais), Cambridge, Massachusetts, 12 e 13 de
outubro de 2009; Vincent Toubiana, Arvind Nunes, Dan Boneh, Helen
Nissenbaum e Solon Barocas, ``Adnostic: Privacy-Preserving Targeted
Advertising,'' (Adnostic: Preservação da Privacidade Dirigida a
Publicidade), \emph{Proceedings of the Network and Distributed System
Symposium} (Simpósio sobre Procedimentos da Rede e Sistema Distribuído),
San Diego, Califórnia, de 28 de fevereiro a 3 de Março, 2010.

16 Calculando os servidores de anúncios sozinhos, uma lista atualizada a
partir de abril de 2011 mostra 2.766 entradas exclusivas; veja em:
\emph{http://pgl.yoyo.org/adservers/formats.php}
(acessado em 13 de abril de 2011).

17 Barocas e Nissenbaum, \emph{"On Notice}'' (Aviso) e Toubiana, Nunes,
Boneh, Nissenbaum e Barocas, \emph{"Adnostic".}

18 Vincent Toubiana e Helen Nissenbaum, ``An Analysis of Google Log
Retention Policies,'' (Uma análise das política de retenção de logs do
Google), \emph{The Journal of Privacy and Confidentiality (}Jornal da
Privacidade e Confidencialidade) (em breve).

19 Por exemplo, a informação pessoal não é compartilhada com ninguém e é
destruída após cada sessão.

20 Deborah Franklin, \emph{``Uninformed Consent''} (Consentimento
desinformado), \emph{Scientific American},, de março de 2011, 24-25.

21 ver Katie Hafner e Matthew Lyon, \emph{Where Wizards Stay Up Late:
The Origins of the Internet} (Quando os assistentes ficam acordados até
tarde: as origens da Internet) (Nova York: Simon e Schuster, 1999); Tim
Berners-Lee, \emph{Weaving the Web: The Past, Present and Future of the
World Wide Web by Its Inventor} (Tecendo A Rede: O Passado, Presente e
Futuro da Grande Rede Mundial Pelo Seu Inventor (Londres: Texere
Publishing, 2000); e Janet Abbate, \emph{Inventing the Internet
(I}nventar a Internet) (Cambridge, Massachusetts: mit Press, 2000).

22 Sobre essa transformação da Internet através da "luta" das partes
interessadas, ver David Clark, John Wroclawski, Karen Sollins e Robert
Braden, ``Tussle in Cyberspace: Defining Tomorrow's Internet'' (A Briga
no Ciberespaço: Definindo o Amanhã da Internet) Anais da Conferência
ACMSigComm de 2002, Pittsburgh, Pensilvânia, 19 a 23 de agosto de 2002,
publicados no \emph{Computer Communications Review (Revisão nas
Comunicações em C}omputadores) 32 (4) (Outubro de 2002).

23 Al Gore, ``Infrastructure for the Global Village: Computers, Networks
and Public Policy,'' (Infra-estrutura para a aldeia Global:
computadores, redes e políticas públicas), edição especial,
"Comunicações, computadores e redes," Scientific American, setembro de
1991, 150-153.

24 John Perry Barlow, "A economia das ideias," Wired, março de 1994, 84.

25 Clay Shirky, \emph{Here Comes Everybody: The Power of Organizing
Without Organizations (}Aí Vem Todo Mundo: O Poder de Organizar sem
Organizações) (New York:Penguin, 2009).

26 Samantha Shapiro, ``Revolution, Facebook-Style'' (Revolução ao estilo
Facebook)\emph{The New York Times}, 22 deJaneiro, 2009,

\emph{http://www.nytimes.com/2009/01/25/magazine/25bloggers-t.html}.

27 Federal Trade Commission (Comissão de Comércio Federal), ``Protecting
Consumer Privacy in an Era of Rapid Change'' (Proteger a privacidade do
consumidor em uma Era de rápidas mudanças) e Department of Commerce
(Departamento De Comércio), ``Commercial Data Privacy and Innovation in
the Internet Economy'' (A privacidade dos dados comerciais e a inovação
da economia da Internet).

28 Compare esta concepção a alegação de Mark Zuckerberg que as normas
mudam devido aos contornos das políticas de privacidade do Facebook; Ver
Bianca Bosker, ``Facebook's Zuckerberg Says Privacy No Longer a `Social
Norm,''' (Zuckerberg do Facebook diz que a privacidade não é mais uma
'norma Social), The Huffington Post, 11 de janeiro de 2010,

\emph{http://www.huffingtonpost.com/2010/01/11/facebooks-zuckerberg-the\_n\_417969.html}

29 U.S. Federal Trade Commission (Comissão Federal de Comércio dos
Estados Unidos), Gramm-Leach-Bliley Ato 15 U.S.C., Subcapítulo I, secão
6801--6809, 12 de November de 1999,
\emph{http://www.ftc.gov/privacy/glbact/glbsub1.htm}
; Adam Barth, Anupam Datta, John Mitchell e Helen Nissenbaum, ``Privacy
and Contextual Integrity: Framework and Applications'' (Privacidade e
integridade Contextual: Estrutura e aplicativos) \emph{Proceedings of
the IEEE Symposium on Security and Privacy} (Procedimentos do Simpósio
sobre Segurança e Privacidade) Berkeley, California, 21 de Maio de 2006.

30 A livraria \emph{Moravian Book Shop} tem agora o seu próprio portal
online,

\emph{http://www.moravianbookshop.com}

31 \emph{The Video Privacy Protection} (A Proteção da Privacidade) Ato
18 U.S.C. artigo 2710, 1988,

\emph{http://www.law.cornell.edu/uscode/html/uscode18/usc\_sec\_18\_00002710-\/-\/-\/-000-.html}.

32 Veja:
\emph{http://www.vatican.va/phome\_en.htm}

33 Veja:
\emph{http://www.online-churches.net}

34 Nissenbaum, \emph{Privacy in Context} (Privacidade no contexto),
especificamente o capítulo 8.

35 David Kocieniewski, ``IRS Sits on Data Pointing to Missing Children''
(As páginas do IRS com Dados sobre Crianças Desaparecidas), \emph{The
New York Times}, 12 de novembro de 2010,

\emph{http://www.nytimes.com/2010/11/13/business/13missing.html}.

36 \emph{Internal Revenue Service} (Serviço Interno de Impostos),
\emph{``Disclosure and Privacy Law Reference Guide''} (Guia de
Referencias sobre as Leis de Privacidade e Divulgação) publicação 4639
(10-2007) número de catálogo 50891P, 1 -- 7.

37 \emph{Hearings on Revenue Revision} (Audiências sobre Revisão de
Impostos) 1925 \emph{Before the House Ways and Means Committee (Comitê
para maneiras e meios de propostas de resolução)}, 69º Congresso, 1ª
sessão 8--9 (1925).

38 Uma pesquisa no Google sobre "status de HIV," realizada 11 de janeiro
de 2011, produziu mais de 7,5 milhões de resultados.

39 \emph{Federal Trade Commission} (Comissão Federal de Comércio),
``\emph{Protecting Consumer Privacy in an Era of Rapid Change}''
(Proteger a privacidade do consumidor em uma Era de rápidas mudanças) e
Department of Commerce (Departamento De Comércio), ``Commercial Data
Privacy and Innovation in the Internet Economy.'' (A privacidade dos
dados comerciais e a inovação da economia da Internet).

40 Noob Cook é apresentado como sendo escrita por "uma garota que gosta
de cozinhar"; consulte:
\emph{http://www.noobcook.com/about/}
(acessado em 25 de fevereiro de 2011).

41 Estabelecer o número de seguidores em um site é altamente inexato.
Vários utilitários oferecem esse serviço, por exemplo, o
\emph{Ghostery}; os números variam de acordo as abordagens adotadas. Nem
todas as abordagens reconhecem todos os tipos de rastreadores. Além
disso, estes números variam também de vez em quando porque sites
frequentemente podem revisar suas políticas subjacentes e arranjos
comerciais. (Estou em dívida com Ashkan Soltani por esclarecer este
ponto).

42 Elizabeth Anderson, \emph{Value in Ethics and Economics (}Valorização
em ética e economia), (Cambridge, Mass.: Harvard University Press,
1995), 147.

43 Milton Friedman, ``Can Business Afford to be Ethical?: Profits before
Ethics''( Os Negócios Podem Ser Éticos?: Lucros Antes De Étical) in
\emph{Ethics for Modern Life (}Ética Para A Vida Moderna), editores:
Raziel Abelson e Marie-Louise Friquegnon, 4ª edição (New York: St.
Martin's Press, 1991), 313--318.

44 Sergey Brin and Lawrence Page, ``\emph{The Anatomy of a Large-Scale
Hypertextual Web Search Engine}'' (A anatomia de um mecanismo
hipertextual de busca na internet em grande escala),

\emph{WWW7/Computer Networks} 30 (1--7) (1998): 107--117; citação
retirada da versão na rede,
\emph{http://infolab.stanford.edu/~backrub/google.html}
(acessado em 26 de fevereiro de 2011).

Veja também Alex Diaz, ``\emph{Through} \emph{the Google Goggles:
Sociopolitical Bias in Search Engine Design},''

(Através Do \emph{Google Goggles}: Viés Sociopolítico Na Busca Pelo
Design De Um Mecanismo) in \emph{Web Searching: Interdisciplinary
Perspectives (}Buscas Da Rede: Perspectivas interdisciplinares),
editores Amanda Spink e Michael Zimmer (Dordrecht, Países Baixos:
Springer, 2008).

45 Robert Ellis Smith, ``\emph{Ben Franklin's Web Site: Privacy and
Curiosity from Plymouth Rock to}

\emph{the Internet}'' (Site de Ben Franklin: Privacidade e Curiosidades
do Plymouth Rock à Internet \emph{) Privacy Journal} (2000): 34, 51.

46 Lucas Introna e Helen Nissenbaum, ``Shaping the Web: Why the Politics
of Search Engines

Matters" (Moldando a Rede: Porque A Política Do Mecanismo De Pesquisa
Importa), \emph{The Information Society} 16 (3) (2000): 1 -- 17;
Pasquale de Frank e Oren Bracha, " \emph{Federal Search Commission?
Access, Fairness, and Accountability in the Law of Search''}

Comissão Federal de Pesquisa? Acesso, Equidade E Responsabilidade na Lei
Da Pesquisa) \emph{Cornell Law Review} 93 (2008): 1149; Toubiana,
Narayanan, Boneh, Nissenbaum e Barocas, ``\emph{Adnostic}.''

47 Toubiana e Nissenbaum, ``An Analysis of Google Log Retention
Policies'' (An Analysis of Google Log Retention Policies,'' (Uma análise
das política de retenção de logs do Google).

48 Veja \emph{``Choicepoint},'' (Ponto de Escolha) EPIC-Electronic
Privacy Information Center,

\emph{http://epic.org/privacy/choicepoint/}
(acessado em 13 de abril de 2011).

49 Como uma questão prática, padrões para contextos de comunicação serão
necessários através do design de interface. Ver, por exemplo, o trabalho
de Ryan Calo e Lorrie Cranor.

50 Nissenbaum, \emph{Privacy in Context} (Privacidade no contexto),
especialmente os capítulos 8--9.

51 Danielle Citron, ``\emph{Fulfilling Government 2.0's Promise with
Robust Privacy Protections}'' (Cumprindo a Promessa do Governo sobre
Robustas Proteções a Privacidade) \emph{George Washington Law Review} 78
(4) (Junho 2010): 822--845

\textbf{Vigilância como triagem social: códigos de computador e corpos
móveis}\footnote{A presente tradução foi feita a partir de LYON, David.
  \emph{Surveillance as Social Sorting: Privacy, Risk and Digital
  Discrimination.} London: Routledge, 2003, pp. 13-30 (nota do editor).}

David Lyon

Este primeiro capítulo explora alguns dos temas-chave discutidos em
``vigilância como triagem social''. Os quatro primeiros parágrafos
apresentam brevemente o argumento, antes de eu sugerir uma série de
maneiras em que a triagem social tornou-se central para a vigilância. Em
seguida observo algumas implicações da vigilância enquanto ocorrência
rotineira da vida cotidiana; foco na ``codificação'' e ``mobilidade''
como aspectos emergentes da vigilância; e concluo sugerindo algumas
novas direções para os estudos sobre vigilância no início do século XXI.
\footnote{Este texto foi originalmente entregue como artigo na School of
  Information da Universidade de Michigan e nas reuniões da Associação
  Sociológica Americana em Anaheim, Califórnia, agosto de 2001.}

A vigilância transbordou dos velhos contêineres do Estado nacional para
se tornar um atributo da vida cotidiana, no trabalho, em casa, no
entretenimento, em movimento. Bem longe do único olho que tudo vê do
\emph{Big Brother}, agora, inúmeras agências rastreiam e controlam
atividades mundanas para uma infinidade de propósitos. Dados abstratos,
agora incluindo vídeos, dados biométricos e genéticos, bem como arquivos
administrativos informatizados, são manipulados para produzir perfis e
categorias de risco em um sistema líquido em rede. O objetivo é
planejar, prever e prevenir classificando e avaliando esses perfis e
riscos.

O conceito de ``triagem social'' destaca a orientação classificadora da
vigilância contemporânea. Ela também relativiza alguns dos aspectos
supostamente mais sinistros dos processos de vigilância (não se trata de
uma conspiração de intenções malignas ou um processo implacável e
inexorável). A vigilância é sempre ambígua (LYON, 1994, p. 219; NEWBURN;
HAYMAN, 2002, p. 167-8). Ao mesmo tempo, a triagem social coloca
firmemente a questão no âmbito social e não apenas no individual ---
como as preocupações com a ``privacidade'' muitas vezes tendem a fazer.
A vida humana seria impensável sem categorizações sociais e pessoais,
porém a vigilância hoje em dia não só racionaliza, mas automatiza o
processo. Como isso é alcançado?

Códigos, normalmente processados por computadores, resolvem transações,
interações, visitas, telefonemas e outras atividades; eles são as portas
invisíveis que permitem ou impedem a participação em uma infinidade de
eventos, experiências e processos. As classificações resultantes são
projetadas para influenciar e gerenciar populações e pessoas direta e
indiretamente, afetando as escolhas e possibilidades dos titulares de
dados. Os portões e barreiras que contêm, conduzem e classificam
populações e pessoas tornaram-se virtuais.

Mas não é somente o fazer das coisas à distância que requer uma
vigilância cada vez maior. Além disso, o processo de triagem social
ocorre como se estivesse em movimento. A vigilância agora lida com a
velocidade e a mobilidade. Na corrida para se chegar primeiro, a
vigilância é simulada para anteceder o evento. No desejo de rastrear, a
vigilância flui e reflui através do espaço. Mas esse processo não é só
de ida. Os sistemas de vigilância sociotécnicos também são afetados por
pessoas que consentem, negociam ou resistem à vigilância. Agora me
permitam detalhar isso com um pouco mais de fôlego.

Uma tendência chave da vigilância atual é a utilização de bancos de
dados indexados para processar dados pessoais com fins diversos. Essa
chave não é ``tecnológica'', como se os bancos de dados indexados
pudessem ser pensados de forma separada das intenções sociais,
econômicas e políticas nas quais estão inseridos. Ao contrário, o uso de
banco de dados indexadosé visto como uma meta futura, mesmo que no
presente, \emph{hardware} e \emph{software} não estejam necessariamente
disponíveis ou suficientemente sofisticados para tanto. O ponto é que o
acesso a uma maior velocidade de manipulação e a fontes mais ricas de
informação sobre indivíduos e populações é compreendido como a melhor
maneira de se verificar e monitorar o comportamento, influenciar pessoas
e populações e de se prever e antecipar riscos.

Um dos exemplos mais óbvios da utilização de bancos de dados indexados
para fins de vigilância ocorre nas práticas atuais de publicidade. Nas
últimas duas décadas, uma enorme indústria tem crescido rapidamente,
agregando populações de acordo com características geodemográficas. O
Canadá, por exemplo, é organizado pela Compusearch em grupos --- de U1
para a elite urbana a R2 para a plebe rural --- que são posteriormente
divididos em subgrupos. O U1 inclui o grupo dos ``Abonados'': ``famílias
com executivos e profissionais muito abastados e de meia-idade. Casas
grandes, caras e pouco financiadas em regiões muito estáveis, antigas e
exclusivas das grandes cidades. Adolescentes e crianças mais velhas''
(TETRAD, 2001). O grupo U6, ``Pressão da Cidade Grande'', é um pouco
diferente: ``bairros urbanos no centro da cidade com a segunda menor
renda domiciliar média. Provavelmente as áreas mais desfavorecidas do
país\ldots{} Os tipos domiciliares incluem solteiros, casais e famílias
de pais solteiros. Uma presença `étnica' significativa, porém
miscigenada. Os níveis de desemprego são muito elevados'' (Id.).

Por meio do uso de tais agrupamentos em conjunto com os códigos postais
--- códigos ZIP nos Estados Unidos --- as agências de publicidade
peneiram e classificam populações conforme seu poder aquisitivo,
tratando os diferentes grupos de acordo com tais padrões. Grupos
possivelmente valiosos ganham atenção especial, promoções especiais e
serviços eficientes de pós-venda, enquanto os demais, que não estão nas
categorias privilegiadas, devem se virar com menos informação e serviços
inferiores. Ferramentas Web ampliaram esses processos para incluir
outros tipos de dados, relativos não somente aos índices
geodemográficos, mas também a outros indicadores de valor. Em processos
muitas vezes conhecidos como ``redlining'' (PERRI, 2001, p. 6) ou
``\emph{weblining}'' (STEPANEK, 2000), os consumidores são classificados
de acordo com seu valor relativo. É tanto para o consumidor soberano! O
vendedor pode agora saber não só onde você mora, mas detalhes como sua
origem étnica (Stepanek observa que, nos Estados Unidos, a empresa
Acxiom\footnote{Nota do Tradutor: A Acxiom Corporation é uma empresa de
  serviços e tecnologias de publicidade criada em 1969 que possui
  escritórios nos Estados Unidos, Europa, Ásia e América do Sul} combina
nomes com dados demográficos para gerar classificações com o ``B'' para
negros, ``J'' para judeus, ``N'' para nipônicos-japoneses, e assim por
diante).

Já se pode ver como dados coletados \emph{off-line} e \emph{on-line}
podem ser associados ou combinados. Conforme a internet se tornou mais
importante como dispositivo de marketing, aumentaram os esforços para
combinar o poder dos bancos de dados off-line (principalmente
geodemográficos) com o dos on-line (principalmente padrões de navegação
e rastros). Isso estava por trás da compra do Abacus (offline) pela
Doubleclick (online) em 1999, que resultou em uma longa investigação
iniciada após protestos de ativistas e organizações pró-privacidade.
Quando as agências de publicidade mesclam informações individualmente
identificáveis pertencentes às características de códigos postais com
evidências de hábitos de consumo ou interesses, adquiridos pelo
rastreamento do uso da internet, em um banco de dados indexado, elas
criam uma relação mais estreita com os consumidores relevantes. Em um
caso notável, foi recentemente oferecida a um médico americano uma lista
de todas as suas pacientes na perimenopausa que não estavam submetidas a
uma terapia de reposição de estrogênio (HAFNER, 2001). Dados online e
offline podem ser combinados para produzir vendas aperfeiçoadas.

Outro campo no qual os bancos de dados indexados se tornaram mais
importantes para a vigilância foi o policiamento. Ao longo de 2001, a
polícia de Toronto e Ontário, no Canadá, introduziu em seus veículos de
patrulha melhorias que ampliaram o escopo das atividades baseadas em
informações. O e-Cops --- Enterprise Case and Occurrence Processing
System --- foi adotado e utiliza comunicação de dados sem fio para se
conectar policiais usando laptops em seus veículos a ferramentas web
para a detecção e prevenção de crimes (MARRON, 2001). Agora, os oficiais
podem não apenas conectar-se diretamente aos arquivos do Serviço
Policial de Toronto, ter acesso aos registros de licenças de motoristas
do Ministério dos Transportes de Ontário e às listas de suspeitos
mantidas pelo Centro de Informações da Polícia Canadense, mas também a
um banco de dados e um software de inteligência corporativa da IBM.

Essa iniciativa, assim como os bancos de dados de marketing, faz uso de
informações geodemográficas. Nesse caso, são identificados padrões
geográficos de crimes visando não só a detecção, mas também a prevenção
da criminalidade, indicando onde um agressor determinado pode atacar em
seguida. Os novos sistemas automatizam tarefas que, anteriormente,
exigiam empregados de escritório como intermediários no processamento de
informações e ferramentas de conexão que costumavam ser usadas em
relativo isolamento. Assim, o sistema é integrado de maneira mais
completa e, argumenta-se, mais vantajosa. Informações sobre o histórico
dos suspeitos agora estão disponíveis instantaneamente e podem ser
recuperadas pelos policiais em seus laptops desde o assento de seus
carros. E o banco de dados indexado pode ser usado para indicar se o
suspeito provavelmente é um infrator em série numa compulsão criminosa
ou um novato.

Assim como os bancos de dados de publicidade, os sistemas de
policiamento são sintomáticos de tendências mais amplas. Nesse caso, a
tendência é de uma tentativa de previsão e prevenção de comportamentos e
de uma mudança para a chamada ``justiça atuarial'', na qual a
comunicação do conhecimento sobre probabilidades desempenha um papel
muito maior nas avaliações de risco (ERICSON; HAGGERTY, 1997). A maneira
como certos territórios são mapeados socialmente torna-se central para o
trabalho da polícia, que é dependente de infraestruturas de informação.
Mas tal mapeamento também depende de estereótipos no tocante ao
território --- \emph{hot spots} --- ou a características sociais como
raça, classe socioeconômica ou gênero. Como Ericson e Haggerty
observaram, essas categorias não podem ser imparciais porque são
produzidas por instituições de risco que já atribuem valores diferentes
a jovens e velhos, ricos e pobres, negros e brancos, homens e mulheres
(Ibid., p. 256).

Os dois exemplos, da publicidade e do policiamento, indicam claramente
como os bancos de dados indexados tornaram-se centrais para a
vigilância. Se a vigilância é entendida como uma atenção sistemática a
detalhes pessoais com a finalidade de gerenciar ou influenciar as
pessoas e grupos em questão, o banco de dados indexado pode ser visto
como ideal também em outras áreas emergentes. O gerenciamento de riscos
e a avalição de seguros, em particular, tendem a incentivar a busca por
maior precisão na identificação e comunicação mais rápida do risco, de
preferência antes que ele seja percebido. Novas tecnologias, como a
biometria, usando impressões digitais dos dedos ou das mãos,
digitalização da íris ou amostras de DNA, são exploradas pela exatidão
da identificação, assim como a interconexão entre as diversas técnicas,
visando a um aumento na velocidade da comunicação, fazem parte, assim,
de um padrão cada vez mais comum.

Dois desenvolvimentos adicionais também ilustram essas mudanças na
vigilância. Um refere-se à rápida proliferação dos circuitos fechados de
televisão (CCTV) ou ``vigilância por vídeo'', e o outro se refere a uma
gama crescente de dispositivos de localização que não apenas situam os
titulares de dados em um espaço fixo, mas também enquanto se movimentam.
Novamente, essas não são meramente inovações tecnológicas com impactos
sociais. São tecnologias ativamente procuradas e desenvolvidas porque
respondem a pressões político-econômicas particulares. Os fatores de
atração política têm a ver com governos neoconservadores que desejam
terceirizar serviços e cortar gastos, especialmente trabalhistas. Ao
fazê-lo, eles também buscam reduzir o medo público do crime e criar
espaços para o consumo ``seguro'' na cidade. Do ponto de vista
comercial, os fatores de atração incluem o estreitamento das margens de
lucro e o desejo de capturar mercados através de relacionamentos com os
consumidores. Os fatores de impulso, por outro lado, referem-se à
orientação para vendas (empresas) e a adoção de novas tecnologias
(agências, organizações, governos).

O Reino Unido é atualmente a incomparável capital do mundo em
videovigilância em locais públicos, mas outros países estão rapidamente
seguindo o exemplo britânico. Grandes cidades da América do Norte,
Europa e Ásia estão usando os CCTV como um meio de controlar o crime e
manter a ordem social. Sudbury, por exemplo, foi a primeira cidade em
Ontário que instalou câmeras de vídeo públicas em 1996, em um movimento
inspirado diretamente pelo exemplo de Glasgow, na Escócia. A polícia de
Sudbury obteve ajuda dos Rotary Clubs e da Canadian Pacific Rail para
colocar suas primeiras câmeras que, segundo reivindicam agora,
conduziram a reduções significativas nas taxas de criminalidade --- que
estão caindo mais rápido do que em outras cidades canadenses (TOMAS,
2000). Na maioria dos casos, bancos de dados indexados ainda não são
utilizados em conjunto com os CCTV, embora o objetivo de criar
categorias de suspeitos, em que se situem comportamentos desviantes ou
anormais, esteja firmemente presente.

Em alguns casos, no entanto, bancos de dados indexados já estão em uso
em situações públicas e privadas na tentativa de conectar imagens
faciais das pessoas captadas pelas câmeras com outras que tenham sido
armazenadas digitalmente. Em Newham, Londres, os CCTV são reforçados por
sistemas inteligentes capazes de reconhecimento facial (NORRIS;
ARMSTRONG, 1999). Em um caso célebre em 2000, as catracas para os
eventos anuais do Super Bowl na Flórida foram filmadas por um sistema de
CCTV que comparou mais de 100 mil imagens daqueles que entraram no
estádio com imagens armazenadas de rostos de criminosos conhecidos (19
identificações foram feitas). Tratou-se de um teste executado por uma
empresa de sistemas de câmera para demonstrar a capacidade das máquinas,
o que ao menos sugere a natureza dos fatores de impulso tecnológico
nesse caso (SLEVIN, 2001).

Muito mais comuns do que os sistemas de reconhecimento facial nas ruas
--- ao menos antes do 11 de setembro de 2001 --- são as tecnologias de
reconhecimento facial usadas em cassinos para identificar trapaceiros.
Tal como ocorre com as catracas, as entradas dos cassinos permitem
capturar imagens relativamente nítidas. Estas podem então ser combinadas
com imagens de bancos de dados e usadas para prender criminosos
conhecidos (CNN, 2001). A crescente utilização de câmeras de segurança
digital pode vir a incentivar essa tendência (BLACK, 2001). Desde 11 de
setembro de 2001, no entanto, muitas cidades manifestaram um amplo
interesse pelos sistemas CCTV de reconhecimento facial para reduzir o
risco de ataques ``terroristas''. A nova vontade política e a
disponibilidade pública para apoiar a disseminação de tais sistemas em
locais públicos têm sido mais do que correspondidas por confiantes
anúncios ``especializados'' sobre a tecnologia disponível, mesmo que a
sua capacidade real de executar o que se solicita não seja comprovada
(ROSEN, 2001).

Sofisticados sistemas de CCTV, como aqueles de Newham, Londres, podem
ser utilizados para seguir pessoas de rua em rua, se elas são do
interesse dos operadores. Assim, não apenas lugares fixos, mas alvos em
movimento também podem se tonar alvos de vigilância. As câmeras, no
entanto, não são os tipos de dispositivos mais comumente usados para
acompanhar pessoas em movimento. Outras tecnologias de localização que
usam satélites de posicionamento global (no inglês, GPS) e sistemas de
informação geográfica (SIG), em conjunto com a telefonia sem fio,
oferecem um potencial de vigilância muito mais poderoso. Já existe um
mercado popular para tais tecnologias de telefones móveis e satélites
entre pais que desejam monitorar seus filhos; todavia, verifica-se um
interesse comercial mais amplo por parte de empresas locadoras de
veículos, serviços de segurança e emergência.

Alguns carros novos selecionados da Ford oferecem o serviço On-Star, que
permite, por exemplo, que hotéis ou restaurantes alertem os usuários
quando estão por perto. Essa é uma extensão previsível dos negócios
eletrônicos. Por decorrência de uma decisão federal, os telefones
celulares nos Estados Unidos terão uma tecnologia de monitoramento sem
fio para permitir a identificação das pessoas que efetuem chamadas de
emergência (ROMERO, 2001). Isso também é previsível e o benefício para
as pessoas com problemas, evidente. No entanto, tais sistemas também
permitem que outras agências --- companhias de seguro, empregadores ---
possam descobrir o paradeiro das pessoas e é apenas uma questão de tempo
até que também se desenvolvam os meios para traçar seus perfis. No Reino
Unido, uma legislação recente, o Ato de Regulamentação dos Poderes
Investigatórios, confere à polícia um acesso inigualável aos celulares
de nova geração (telefones ``móveis'' no Reino Unido ou Austrália) para
rastrear a localização de quem realiza chamadas (BARNETT, 2000).
Juntamente com os Sistemas de Transporte Inteligente (STI), essas
tecnologias emergentes permitem que a vigilância se transfira
decisivamente para o território do movimento através do espaço.

\section{Explicar a vigilância cotidiana}

Existe uma ideia segundo a qual as novas tecnologias são empregadas para
compensar perdas advindas da implementação de outras tecnologias. Novas
tecnologias de informação, e especialmente comunicação, assim como um
aperfeiçoamento nos meios de transporte, permitiram que muitas coisas
fossem feitas à distância na última metade de século. Uma elasticidade
sem precedentes nas relações permite que as partes sejam envolvidas em
uma produção --- de tudo desde automóveis até música ---, administração,
comércio, comunicação interpessoal, entretenimento e, claro, guerra
dispersa. Os relacionamentos já não dependem de pessoas encarnadas
coexistindo umas com as outras. Imagens e dados abstratos substituem a
população viva em muitas trocas e comunicações hoje. Alguns desses dados
e imagens abstratas são deliberadamente interceptados ou capturados com
a finalidade de monitorar as pessoas, agora, invisíveis, que estão,
contudo, em uma imensa rede de conexões. Assim, para fins de
gerenciamento e administração, fazem ressurgir o corpo desaparecido
utilizando mais ou menos as mesmas tecnologias que ajudaram a, em
primeiro lugar, desaparecer com ele (LYON, 2001b).

Entendida assim, a vigilância aparece muito menos sinistra. As antigas
metáforas do \emph{Big Brother} ou do Panóptico, impregnadas de um duro
controle social, parecem de alguma forma menos relevantes em um mundo
cotidiano de transações telefônicas, navegação na internet, câmeras de
segurança nas ruas, monitoramento no trabalho e assim por diante. E, de
fato, parece oportuno pensar tal vigilância como positiva e benéfica em
alguns aspectos, permitindo novos níveis de eficiência, produtividade,
comodidade e conforto que muitos nas sociedades tecnologicamente
avançadas consideram naturais. Ao mesmo tempo, a aparentemente inocente
inserção da vigilância na vida cotidiana levanta algumas questões
importantes para a análise sociológica. As associações superficiais
entre vigilância e contenção de riscos podem, por vezes, obscurecer as
maneiras pelas quais a expansão da vigilância pode na verdade contribuir
para tal, assim como reduzir riscos.

Não é por acaso que o interesse pela privacidade cresceu, aos trancos e
barrancos, na década passada. Essa mudança é atribuída justamente ao
aumento do âmbito e da abrangência da vigilância em ambientes tanto
comerciais e governamentais, quanto de trabalho. Pela mesma razão, isso
está ligado também à maior vigilância sobre populações masculinas de
classe média. Os grupos socioeconômicos mais baixos e as mulheres há
muito se acostumaram a serem observados por vários vigilantes. Do mesmo
modo, o crescimento das preocupações com a privacidade tem de ser visto
no contexto de sociedades crescentemente individualizadas (BAUMAN, 2001)
e, acima de tudo, da individualização do risco, uma vez que as redes de
segurança social se deterioram uma a uma. A privacidade informacional,
baseadas em ``práticas de informações justas'' em quase todos os
lugares, refere-se ao controle comunicativo, ou seja, até que ponto os
titulares de dados têm algum poder de decisão sobre a maneira como seus
dados pessoais são coletados, processados e utilizados. Tais políticas
de privacidade agora estão consagradas na lei e na autorregulação
voluntária em muitos países e contextos.

Todavia, a privacidade é ao mesmo tempo contestada e confinada em seu
escopo. Em termos culturais e históricos, a privacidade tem relevância
limitada em alguns contextos. Como veremos em seguida, a vigilância
cotidiana está implicada nos modos contemporâneos de reprodução social
--- é um meio fundamental de classificação das populações para
tratamento discriminatório --- e, por conta disso, é incerto se invocar
mais privacidade seria adequado como possível solução. Com certeza as
práticas justas de informação caminham de certa forma para uma abordagem
das potenciais desigualdades geradas, ou pelo menos facilitadas, pela
vigilância enquanto triagem social. Mas esse último parece ser um
processo sócio-estrutural que, por mais extenuantes que sejam as
declarações sobre a privacidade como um bem público ou comum (ver REGAN,
1995), parece clamar por políticas públicas e iniciativas políticas
diferentes, ou ao menos adicionais.

Outra questão sociológica é que o mapeamento da vigilância já não é uma
questão meramente regional. Antigamente, a sociologia podia
confiantemente supor que as relações sociais eram de certa forma
isomórficas em relação aos territórios --- e é claro que, ironicamente,
essa suposição é precisamente o que está por trás das atividades de
agrupamento geodemográfico nos bancos de dados comerciais. Mas o
desenvolvimento de diferentes tipos de redes de relacionamentos desafia
essa suposição simples. As relações sociais tornaram-se mais fluidas,
mais líquidas (BAUMAN, 2000), e os dados de vigilância,
correspondentemente, estão mais interconectados e devem ser vistos em
termos de fluxos (URRY, 2000). Não é apenas \emph{onde} as pessoas estão
quando usam seus celulares, e-mails ou navegam na internet. \emph{Com
quem} elas estão conectadas e como essa interação pode ser registrada,
monitorada ou rastreada são fatores que também contam.

Uma terceira questão sociológica levantada pela vigilância cotidiana se
refere aos processos de vigiar e ser vigiado. Muitas vezes está
implícito que um se infere do outro --- que certo modo de vigilância,
digamos, um circuito fechado de CCTV de rua, funcione simplesmente para
reduzir a criminalidade, ou que os temores paranoicos que alguém sinta
de um departamento de bem-estar social do \emph{Big Brother} sejam
plenamente justificados. Na verdade, muitas situações de vigilância têm
recebido pouca atenção sociológica, tornando fácil o recurso às
perspectivas do determinismo tecnológico ou às respostas jurídicas
quando se tenta compreender o que está acontecendo. Na verdade, ao menos
três fases do processo devem ser isoladas para fins analíticos. A
criação de códigos pelos usuários de dados é uma delas, revelando tanto
a economia política da tecnologia quanto as implicações de certas
capacidades técnicas. A extensão do consentimento dos titulares de dados
é uma segunda etapa; isso explora as circunstâncias nas quais os
vigiados simplesmente aceitam sua vigilância, em que medida negociam com
a vigilância e quando de fato resistem a ela. Isso leva a uma terceira
pergunta: o que é necessário para que a oposição à vigilância se
mobilize politicamente de forma organizada --- seja \emph{ad hoc} ou em
longo prazo --- e quais as razões pragmáticas ou éticas para fazê-lo?

\section{Triagem social}

A vigilância diária depende cada vez mais de bancos de dados indexados.
Mesmo onde esse ainda não é ou não é totalmente o caso -- tal como nos
sistemas de CCTV, predominantemente operados por humanos ---, um
objetivo central é a triagem social. Os sistemas de vigilância obtêm
dados pessoais e coletivos com o fim de classificar pessoas e populações
de acordo com critérios variáveis, para determinar quem deve ser alvo de
tratamento especial, suspeita, elegibilidade, inclusão, acesso e assim
por diante. O que Oscar Gandy, referindo-se aos bancos de dados de
marketing, chama de ``triagem panóptica'' é, em suma, uma tecnologia
discriminatória, ainda que não seja totalmente automatizada em todos os
casos (GANDY, 1993; 1995; 1996). Ela filtra e classifica com propósitos
de avaliação, de julgamento. Com isso, afeta as escolhas de estilo de
vida das pessoas --- se você não aceitar o \emph{cookie} que relata seus
hábitos de navegação para a empresa matriz, não espere que a informação
ou acesso esteja disponível --- e suas oportunidades de vida ---
detalhes sobre o bairro em que você vive repercutem em questões como
seus prêmios de seguro e que tipos de serviços são disponibilizados
(GRAHAM; MARVIN, 2001), como sua área é policiada e que tipo de
publicidade você recebe.

Se a vigilância enquanto triagem social está crescendo, isso não é
apenas porque alguns dispositivos novos se tornaram disponíveis. Ao
contrário, os dispositivos são procurados devido ao número crescente de
riscos percebidos e reais, e ao desejo mais completo de gerenciar
populações --- sejam essas populações formadas por cidadãos, empregados
ou consumidores. O desmantelamento do Estado de bem-estar, por exemplo,
que vem ocorrendo sistematicamente em todas as sociedades avançadas
desde seu apogeu na década de 1960, tem como efeito a individualização
dos riscos. Uma vez que o próprio conceito de Estado de bem-estar
envolve uma partilha social dos riscos, o inverso ocorre quando ele
entra em declínio. Quais são os resultados disso?

Para aqueles que ainda se encontram em extrema necessidade, por causa de
desemprego, doença, monoparentalidade ou pobreza, a vigilância é
reduzida a um meio de disciplina. Casos de fraude são buscados de
maneira mais ativa através da combinação de dados e outros meios ---
moradores de Ontário que possuem seguro-desemprego, quando cruzam a
fronteira dos Estados Unidos podem ter seus dados pessoais verificados
pela alfândega canadense para investigar duplos subsídios --- e os
critérios de elegibilidade passam a ser controlados de maneira mais
severa. Em Ohio, o departamento de serviços humanos adverte seus
clientes que ``seu número de seguridade social pode ser usado para
verificar a sua renda e/ou informações empregatícias com empregadores
anteriores ou atuais, seus recursos financeiros por meio da Receita
Federal\footnote{Nota do Tradutor: nos Estados Unidos, Internal Revenue
  Service ou IRS} , remunerações trabalhistas, benefícios por invalidez
{[}\ldots{}{]}''. O documento continua: o ``número assim como outras
informações será usado em combinações geradas pelo computador e nos
programas de avaliação ou auditoria para certificar-se de que seu
domicílio é elegível para o vale-alimentação, o almoço escolar, o ADC e
o Medicaid\footnote{Nota do Tradutor: O Medicaid é um programa de saúde
  social dos Estados Unidos oferecido a famílias de baixa renda,
  incluindo adultos, crianças, mulheres grávidas, idosos e pessoas
  necessidades especiais.}'' (GILLIOM, 2001, p. 18). Para aqueles cujos
riscos se tornaram, sobretudo, uma responsabilidade pessoal ou familiar,
as companhias de seguros vão empregar meios cada vez mais intrusivos
para verificar a saúde, emprego e outros riscos associados a cada
solicitação. Dados pessoais são constantemente procurados para tais
intercâmbios, que exigem que o conhecimento oriundo da vigilância seja
comunicado.

A individualização dos riscos, assim, fomenta os níveis já acentuados de
vigilância, subentendendo-se que a categorização automática ocorre com
crescente frequência. Agora, como observamos acima, a categorização é
endêmica e vital para a vida humana, especialmente para a vida social.
Os processos de categorização institucional, no entanto, sofreram um
grande surto na modernidade, com seu impulso analítico e racionalizante.
Todas as instituições sociais modernas, por exemplo, dependem da
diferenciação para descobrir quem conta como cidadão, quais cidadãos
podem votar, quem pode deter propriedades, quais pessoas podem se casar,
quem se graduou em qual escola, com quais qualificações, quem é
contratado por quem, e assim por diante. Muitas dessas questões eram
verificadas menos escrupulosamente antes do advento da modernidade, e
havia uma certa ambiguidade e (o que agora pode ser visto como)
indefinição de limites.

O crescimento de instituições modernas --- acima de todas o Estado-Nação
e a empresa capitalista --- significou que aqueles que eram cidadãos,
empregados e, com o tempo, consumidores viram-se detentores de
identidades institucionais ou organizacionais que tinham de ser afinadas
com suas identidades próprias. Isso não quer dizer que identidades de
grupo ou clã não existissem antes dos tempos modernos ou que a
identidade própria seja de alguma forma um dom pré-social da pessoa. Ao
contrário, é um lembrete de que as identidades organizacionais se
proliferaram ao longo dos tempos modernos e se tornaram fatores cada vez
mais significativos na determinação das oportunidades de vida. Com a
ascensão concomitante das modernas compreensões de ego, as transações
entre identidades organizacionais e pessoais tornaram-se cada vez mais
frenéticas e complexas.

O que acontece quando os computadores entram em cena? Em resumo, o poder
social de informação é reforçado. Para começar, os registros dessas
identidades organizacionais, há muito tempo relegados a arquivos mortos,
raramente perturbados, agora estão em movimento. \emph{Dublês de dados}
--- várias concatenações de dados pessoais que, queira-se ou não,
representam ``você'' dentro da burocracia ou da rede --- agora começam a
fluir como impulsos elétricos e estão sujeitos à alteração, adição,
mescla e perda enquanto viajam. De modo complementar, a vigência dos
dublês de dados depende agora de complexas infraestruturas de
informação. Isso pode ajudar a democratizar as informações; pode
igualmente conduzir a tiranias. Como John Bowker e Leigh Star dizem:

Algumas são tiranias de inércia --- burocracia --- mais do que políticas
públicas explícitas. Outras são as discretas vitórias dos construtores
de infraestrutura inscrevendo suas políticas dentro dos sistemas. Outras
ainda são quase acidentais --- sistemas que se tornam tão complexos que
nenhuma pessoa e nenhuma organização podem prever ou administrar uma boa
política. (BOWKER; STAR, 1999, p. 50)

Embora toda essa modernidade possa ter ajudado a gerar identidades
organizacionais e, agora, o dublê de dados dentro de complexas
infraestruturas de informação, ela não garantiu, todavia, que dublês e
identidades fossem classificadas de uma maneira livre de estereótipos ou
outras tipologias prejudiciais. Como já vimos, os publicitários usam os
dados geodemográficos e de comportamento dos consumidores --- que podem
em si mesmos ser inexatos --- para criar suas categorias e podem também
adicionar à combinação critérios altamente questionáveis, como
identificadores de origem racial e étnica. O policiamento de alta
tecnologia pode envolver pesquisas de dados baseados na rede sem fio,
mas não foi além de noções como ``\emph{hotspots}'' da cidade para
identificar áreas que requerem atenção especial da polícia, onde um
``comportamento indesejado'' pode ocorrer ou onde é necessário que
certos elementos sociais, sobretudo as pessoas pobres, sejam varridos
para longe do turismo. E quanto mais o policiamento passa a depender da
vigilância dos CCTV, mais outros tipos de categorias preconceituosas
entrarão em jogo. Na Grã-Bretanha, ser jovem, homem e negro é garantia
de um maior índice de controle pelas câmeras de rua (NORRIS; ARMSTRONG,
1999).

\section{Códigos de computador}

Embora os computadores não sejam necessariamente usados em todos os
tipos de vigilância --- alguns ainda acontecem no cara a cara, e alguns,
assim como a maioria dos sistemas de CCTV, ainda requerem operadores
humanos ---, a maior parte dos aparatos de vigilância nas sociedades
mais ricas e tecnológicas depende de computadores. Bancos de dados
indexados tornaram-se particularmente significativos, juntamente com as
capacidades das redes remotas. O uso atual dos computadores está se
expandindo do fixo para o móvel, permitindo que vigiantes ou vigiados
--- ou ambos --- estejam em movimento. Como em outras áreas, não é
apenas o processamento da informação que importa, mas a comunicação.

Mas que informação é processada e comunicada? As taxações e julgamentos
feitos sobre os titulares de dados dependem de critérios codificados e
são esses códigos que fazem com que os processos de vigilância funcionem
de determinadas maneiras. São eles os botões que colocam uma pessoa,
digamos, na categoria dos Abonados, e a outra na da Pressão da Cidade
Grande, uma pessoa como com riscos para a saúde, a seguinte como com
boas perspectivas. Como Bowker e Star notam: ``valores, opiniões e
retórica são congelados em códigos''. Eles aprofundaram Marx para
sugerir que: ``{[}\ldots{}{]} o software é um discurso organizacional e
de política congelado'' (BOWKER; STAR, 1999, p. 135). Em primeiro lugar,
diferentes setores ajudam a determinar a codificação. Como Ericson e
Haggerty (1997, p. 23) demonstram, as companhias de seguros desempenham
um papel cada vez maior na determinação das categorias de policiamento
(ver também ERICSON; BARRY; DOYLE, 2000). E gestores de recursos
humanos, opositores da organização sindical, podem ajudar na codificação
de e-mails ou dispositivos de monitoramento da internet.

Os códigos de computação, portanto, são extremamente importantes para as
maneiras pelas quais a vigilância funciona. Num sentido forte, sistemas
de vigilância são o que está nos códigos. Eles regulam o sistema ---
como disse Lawrence Lessig a respeito do ciberespaço, ``o código é a
lei\emph{''} (1999, p. 6). Lessig também observa que parece ter sido uma
surpresa para alguns o fato de que o ciberespaço seja necessariamente
regulamentado. Com efeito, o próprio termo ``ciberespaço'' tem óbvias
afinidades com ``cibernética'', a ciência do controle remoto, conectada
desde o início com uma visão de regulação perfeita. Paradoxalmente,
Lessig argumenta que a comercialização do ciberespaço está construindo
uma arquitetura que aperfeiçoa o controle. Dessa forma ele simplesmente
perpetua o que James R. Beniger explicou a respeito da tecnologia da
informação em geral --- ela contribui para uma ``revolução do controle''
(BENIGER, 1986). O que é verdadeiro sobre a internet é também verdadeiro
sobre outras formas de códigos de computação. E é por isso que é tão
importante que os códigos sejam analisados e avaliados nas configurações
de vigilância.Sem estarem envolvidos eles mesmos nos tipos da sociologia
da tecnologia que é totalmente necessária para o entendimento desses
códigos, certos teóricos franceses --- notoriamente Paul Virilio e
Gilles Deleuze --- observaram que os processos de ordenamento social
sofreram mudanças ao longo da última ou das duas últimas décadas. Eles
argumentam que a vigilância de hoje vai além daquela da sociedade
disciplinar de Foucault, onde pessoas são ``normalizadas'' por suas
posições categóricas, para o que Deleuze chama de ``sociedade de
controle'', em que semelhanças e diferenças são reduzidas a código
(DELEUZE, 1992). A codificação é crucial, porque se supõe que os códigos
contenham os meios de previsão, de antecipação de eventos (como crimes),
condições (como a Aids) e comportamentos (como as escolhas dos
consumidores) que ainda têm de acontecer (ver também BOGARD, 1996). Os
códigos formam conjuntos de protocolos que ajudam a alterar a
experiência cotidiana de vigilância. Como diz Virilio, obstáculos
físicos e distâncias temporais tornam-se menos relevantes em um mundo de
fluxos de informação. O velho mundo da vigilância, que dependia do
traçado da cidade (remontando aos tempos de muralhas e portões), é agora
suplantado --- ou, eu diria, suplementado --- pela vigilância mais
recente, que depende mais do que Virilio chama de ``protocolos
audiovisuais'' (VIRILIO, 1994). Virilio se refere a esse tipo de
vigilância como ``\emph{pros}pecção'' porque os códigos prometem visão
antecipada, captando eventos futuros (Id., 1989).

Esses códigos de computador se tornam mais significativos para a
vigilância cada vez que se torna possível adicionar outras dimensões aos
dados coletados e processados. A infraestrutura da informação lida com
tipos de códigos cada vez mais diversos e, conforme o faz, as
capacidades de vigilância de outras áreas, antes relativamente separadas
e distintas, são incrementadas. Os códigos da vigilância de dados {[}no
inglês, \emph{dataveillance}; ver CLARKE, 1988{]} têm sido aumentados
não só pelos protocolos audiovisuais citados por Virilio, mas também
pelos biométricos, genéticos e locacionais. Também esses carregam
consigo a bagagem de suas origens e dos valores, opiniões e retórica dos
setores interessados. Assim, por exemplo, o ``corpo codificado'' de uma
pessoa que tenta atravessar uma fronteira nacional pode descobrir que
ela já é bem-vinda ou se já está excluída com base em uma identidade que
é estabelecida (não apenas determinada; ver VAN DER PLOEG, 1999) pelos
códigos.

Nem é preciso dizer que pessoas que buscam asilo estão entre os mais
vulneráveis a tal codificação --- ou que este pode ser um nível
politicamente aceitável sobre o qual se possa criar novos sistemas de
códigos envolvendo a biometria. Foi relatado em maio de 2001, por
exemplo, que o Departamento Canadense de Imigração pretende introduzir
um cartão inteligente que pode conter identificadores como leitores de
íris ou de impressões digitais (WALTERS, 2001). Após os atentados de 11
de setembro de 2001, o clima de aceitabilidade política alterou-se de
maneira bastante radical, e os identificadores nacionais por cartões
inteligentes tornaram-se uma das técnicas de defesa contra o
``terrorismo'' mais seriamente consideradas. Eles tem a capacidade de
serem codificados para várias finalidades.

\section{Corpos móveis}

Obviamente, no século XXI mais e mais pessoas querem atravessar
fronteiras. Não só os que requerem pedido de asilo e outros refugiados,
mas os fluxos de viajantes também incluem empresários, turistas,
atletas, artistas, estudantes e assim por diante. Se a globalização é
corretamente considerada como um processo no qual o mundo se torna um só
lugar, então só se pode esperar que as fronteiras venham a se tornar
cada vez mais permeáveis. A mobilidade, tanto física quanto virtual, é
uma marca da era da informação e comunicação. Igualmente previsível, em
um mundo cada vez mais móvel, é que as práticas de vigilância evoluam de
forma paralela. Mas há diferentes aspectos nessa mudança.

Primeiramente, as formas conectadas de vigilância nunca foram tão
poderosas e são mais visíveis nas atividades das grandes corporações ---
como a Doubleclick ou a Disney ---, por um lado, ou no policiamento e na
administração governamental, por outro. Também existem ligações
potenciais e reais entre a vigilância que ocorre nas organizações
públicas e privadas. As empresas farmacêuticas, por exemplo, podem
eventualmente ganhar acesso aos bancos de dados dos sistemas de
assistência médica do governo e vice-versa (HAFNER, 2001). Mas, em
segundo lugar, isso deriva para outro tipo de organização de vigilância,
ou, como Gilles Deleuze e Félix Guattari chamaram, ``agenciamento''
(1987; ver também HAGGERTY; ERICSON, 2000). Essa vigilância não se
limita a um departamento de um governo ou corporação, mas cresce e se
expande rizomáticamente, como aquelas plantas rastejantes da horta ou
dos gramados suburbanos. Alguns sistemas de CCTV parecem ``se
esparramar'' dessa forma, movendo-se de acordo com uma lógica
imprevisível e conectada que Norris e Armstrong chamam de ``mutabilidade
expansível'' (1999).

Um terceiro aspecto da vigilância em um mundo de mobilidades é que
numerosos dispositivos novos estão disponíveis para identificar a
localização dos titulares dos dados. Eles representam um desenvolvimento
específico nas indústrias de computação e telecomunicações e são
baseados na telefonia sem fio, vídeo, dados de GPS recentemente
disponíveis e, claro, bases de dados indexaveis. Alguns são sistemas
Sistemas de Transporte Inteligente (em inglês Intelligent Transportation
Systems ou ITS), tais como pedágios com tecnologias automatizadas ou
dispositivos de navegação ou de monitoramento a bordo. Outros conectam
recursos de GPS e SIG com telefones celulares para permitir que a
localização de quem faz a chamada --- especialmente em situações de
emergência --- seja facilmente rastreada no raio de alguns metros. Essas
tecnologias representam um setor emergente, mas já há evidências
suficientes de sua utilização --- tanto nos sistemas fixos de pedágios,
quanto nos localizadores de chamadas de emergência a partir de telefones
celulares --- para sugerir que isso não é algo passageiro.

De um jeito ou de outro, a vigilância parece perfeitamente capaz de se
manter atualizada de acordo com as tendências sociais de maior
mobilidade. Afinal de contas, a vigilância depende cada vez mais das
mesmas tecnologias que, em primeiro lugar, permitiram que a mobilidade
se expandisse. Velhas noções de ordem, padrão e regularidade parecem
menos proeminentes para um mundo de mobilidades, tornando plausível a
visão de Urry de que os regimes emergentes têm mais a ver com a
``regulação das mobilidades'' (2000, p. 186), de modo similar a um vigia
regulando o gado que, caso contrário, atravessaria fronteiras segundo
sua vontade. Isso não significa que os Estados nacionais perderão seu
poder de regular (a globalização, afinal de contas, implica,
paralelamente, em uma localização), ou que organizações mais
hierarquizadas (empresas, departamentos de polícia) definharão. Ao
contrário, sugere que as informações de vigilância --- juntamente com as
pessoas de quem são extraídos os dados --- simplesmente estarão entre os
fluidos que circulam e fluem dentro e além daquilo que, outrora, se teve
convictamente por ``sociedades''.

\section{Conclusões}

As sociologias da vigilância entendida como triagem social estão pouco
desenvolvidas. Tais estudos tenderam tanto a recorrer aos recursos
teóricos liberais e às ideias inspiradas em Orwell e esquemas panópticos
e pós-estruturalistas (BOYNE, 2000), quanto a começar com as respostas
de políticas públicas mais comuns--- proteção de dados e privacidade ---
e retornar às análises partindo delas (GILLIOM, 2001, p. 8). O relato
acima sugeriu algumas limitações de tais abordagens, que, sejam quais
forem os seus pontos fortes, não lidam de fato com as questões de corpos
em desaparição, codificação, categorização e mobilidades. Há uma demanda
por muitas pesquisas de caráter etnográfico, explicativo e ético. A
etnográfica nos ajudaria a compreender os processos de codificação e de
vivência cotidiana da vigilância. A explicativa desenvolveria teorias
sobre a triagem social e o poder social da informação em configurações
de vigilância. O ético forneceria uma crítica de ponta para estudos que
claramente tocam em questões de justiça e dignidade.

A gramática dos códigos oferece ricos filões para a pesquisa social.
Explorar como os códigos são compostos e modificados, do ponto de vista
da sociologia da tecnologia e da economia política, traria pistas sobre
o ``\emph{switching power}'' (CASTELLS, 1996) nas sociedades
contemporâneas. Quais os desejos que motivam os processos de codificação
promovidos pelos que utilizam e buscam dados? A lista de candidatos
possíveis inclui controle, governança, segurança, proteção e lucro. Como
diferentes setores trazem seus interesses para pressionar os processos
de codificação? Quais tipos de interesses (gestão de riscos, seguros)
predominam? Como um crescente desinvestimento e a terceirização para
empresas comerciais afetam a busca por dados pessoais e populacionais?
Em que medida pressões comerciais incentivam a utilização de dados para
fins além daqueles para os quais foram coletados?

Do mesmo modo, entender como pessoas experimentam a vigilância na vida
cotidiana é uma tarefa central para a investigação sociológica. Alguns
estudos já foram feitos sobre, por exemplo, como jovens ``atuam'' na
frente de câmeras de rua. Mas muitos outros são necessários, sobre o
preenchimento de formulários de garantia ou de candidatura a benefícios,
a navegação na internet, o uso de cartões com códigos de barras e assim
por diante. Tal análise traria pistas sobre até que ponto esses sistemas
de vigilância aparentemente poderosos realmente funcionam e em que
medida seu poder é restringido pela negociação, dissimulação e
resistência ativa por parte dos titulares dos dados pretendidos. Sob
quais condições os atores sociais negociam seus dados pessoais por
vantagens comerciais ou de posição? Quando potenciais titulares de dados
podem simplesmente se recusar a divulgá-los? Quais classificações são
relativamente maleáveis e suscetíveis à modificação pelos titulares de
dados, e quais são mais impermeáveis à negociação ou contestação?

Os estudos sobre vigilância são marcados atualmente por uma busca
urgente de novos conceitos e teorias explicativas. Os mais fecundos e
emocionantes estão emergindo do trabalho transdisciplinar, envolvendo,
entre outras, a sociologia, economia política, história e geografia.
Entre essas, a sociologia da tecnologia é particularmente importante,
pois os estudos de vigilância têm de lidar cada vez mais com a interação
entre pessoas e máquinas. Mas isso também atrai colegas das ciências da
computação e da informação, que podem explicar, por exemplo, questões
vitais de programação. Uma vez que técnicas similares são aplicadas em
diferentes áreas --- por exemplo, o uso de bases de dados indexados no
policiamento e na vigilância para fins de publicidade ---, recursos
teóricos de uma área podem então ser emprestados a outra. Assim como os
cientistas da informação, colegas de estudos em políticas públicas e
direito também terão um papel nos estudos de vigilância, ao menos porque
o deslocamento para além da ``privacidade'' também tem implicações na
responsabilização em contextos legais e organizacionais.

A vigilância, argumentei, é intensificada em um mundo de relações
remotas, onde muitas conexões não envolvem diretamente pessoas corpóreas
e coexistentes, e onde já não vemos mais os rostos daqueles com quem
estamos ``em contato'' ou com quem nos envolvemos em trocas. Bancos de
dados indexados dependem de dados abstraídos de pessoas vivas e
corpóreas, dados esses que são posteriormente usados para representá-las
diante de uma organização. Assim, dados extraídos de pessoas --- em
caixas eletrônicos, por meio de câmeras de rua, em situações de
monitoramento no trabalho, através exames genéticos ou testes de
medicamentos, no uso do telefone celular --- são usados para criar
dublês feitos de dados, e que são, eles próprios, constantemente
mutantes e modificáveis. Mas esses dublês, criados a partir de
categorias codificadas, não são ficções virtuais inocentes ou inócuas.
Conforme circulam, eles servem para abrir e fechar as portas de
oportunidades e acesso. Eles afetam a elegibilidade para benefícios de
crédito ou estatais e concedem credenciais ou geram suspeitas. Eles
fazem uma diferença real. Eles têm ética, política.

As sociologias da vigilância sempre serão produzidas a partir de um
ponto de vista e parece-me que tais pontos de vista dificilmente serão
críticos se eles negligenciam a relação entre dados abstratos e pessoas
sociais corpóreas. As atitudes do Iluminismo, enraizadas na modernidade,
fomentaram a impessoalidade e a mediação eletrônica tem hoje exacerbado
essa situação. Repensar a importância do rosto afeta como cada um
percebe as questões ao redor das condições para a revelação pessoal (daí
o debate sobre privacidade) e também sobre a justiça em face da triagem
social automatizada. Como defendo em outro texto (LYON, 2001a), o rosto
ausente oferece possibilidades enquanto guia moral em ambos os níveis.
Com relação à triagem social, o rosto sempre resiste à mera
categorização, ao mesmo tempo em que requer que os usuários de dados
tentem estabelecer confiança e justiça. Isso não soluciona a questão
política, mas, em minha opinião, gera um forte ponto de partida ético,
que pode servir como guia para a análise crítica.

Bibliografia

BARNETT, A. Every Move You Make\ldots{} \emph{Observer}, 2000.
Disponível em:
\textless{}http://www.guardianunlimited.co.uk/Print/0,3858,4045760,00.html\textgreater{}.
Acesso em: 30 jul. 2000.

BAUMAN, Z. \emph{Liquid Modernity}. Cambridge: Polity Press, 2000.

\_\_\_\_\_\_. \emph{The Individualized Society}. Cambridge: Polity
Press, 2001.

BENIGER, J. R. \emph{The Control Revolution}: Technological and Economic
Origins of the Information Society. Cambridge, MA: Harvard University
Press, 1986.

BLACK, D. Crime Fighting's New Wave, The Toronto Star. 2001. Disponível
em:
\textless{}http://www.torstar.com/thestar/editorial/tech/20000125NEWOld\_CI-UPFRONT.htm\textgreater{}.
Acesso em: 25 jan. 2001.

BOGARD, W. \emph{The Simulation of Surveillance}: Hypercontrol in
Telematic Societies. New York: Cambridge University Press, 1996.

BOWKER, J.; STAR, L. \emph{Sorting Things Out: Classification and its
Consequences}, Cambridge, MA: MIT Press, 1999.

BOYNE, R. Post-panopticism. \emph{Economy and Society}, vol. 29, n. 2,
2000, p. 285--307.

CASTELLS, M. \emph{The Rise of the Network Society}. Oxford: Blackwell,
1996.

CLARKE, R. Information Technology and Dataveillance.
\emph{Communications of the ACM}, vol. 31, n. 5, 1988, p. 498--512.

CNN. Casinos Use Facial Recognition Technology. 2001. Disponível em:
\textless{}http://www.cnn.com/2001TECH/ptech/02/26/casino.surveillance.ap/index.html\textgreater{}.
Acesso em: 26 fev. 2001.

DELEUZE, G. Postscript on the Societies of Control. 1992. out. 59, p.
3-7.

DELEUZE, G.; GUATTARI, F. \emph{A Thousand Plateaus}: Capitalism and
Schizophrenia. Minneapolis: University of Minnesota Press, 1987.

ERICSON, R. e HAGGERTY, K. \emph{Policing the Risk Society}. Toronto:
University of Toronto Press, 1997.

ERICSON, R.; BARRY, D.; DOYLE, R. \emph{The Moral Hazards of
Neo-liberalism}: Lessons from the Private Insurance Industry.
\emph{Economy and Society}, vol. 29, n. 4, 2000, p. 532-58.

GANDY, O. \emph{The Panoptic Sort}: A Political Economy of Personal
Information. CO Boulder: Westview, 1993.

\_\_\_\_\_\_. It's Discrimination, Stupid! In: BROOK, J.; BOAL, A. I.
(Ed.). \emph{Resisting the Virtual Life}. San Francisco: City Lights,
1995.

\_\_\_\_\_\_. Coming to Terms with the Panoptic Sort. In: LYON, D.;
ZUREIK, E. (Ed.). \emph{Computers, Surveillance, and Privacy}.
Minneapolis: University of Minnesota Press, 1996.

GILLIOM, J. \emph{Overseers of the Poor}: Surveillance, Resistance, and
the Limits of Privacy, Chicago: University of Chicago Press, 2001.

GRAHAM, S.; MARVIN, S. \emph{Splintering Urbanism}: Networked
Infrastructures, Technological Mobilities, and the Urban Condition.
London; New York: Routledge, 2001.

HAFNER, K. Privacy's Guarded Prognosis. \emph{The New York Times}, New
York, 1 mar. 2001. Disponível em:
\textless{}http://www.nytimes.com/2001/03/01/technology/01MEDI.html\textgreater{}.

HAGGERTY, K.; ERICSON, R. The Surveillant Assemblage. \emph{British
Journal of Sociology}, London, vol. 51, n.4, 2000, p. 605-22.

LESSIG, L. \emph{Code and Other Laws of Cyberspace}. New York: Basic
Books, 1999.

LYON, D. \emph{The Electronic Eye}: The Rise of Surveillance Society.
Cambridge, UK: Polity; Malden, MA: Blackwell, 1994.

\_\_\_\_\_\_. Facing the Future: Seeking Ethics for Everyday
Surveillance. \emph{Ethics and Information Technology}, Dordrecht, vol.
3, n. 3, 2001a, p. 171-81.

\_\_\_\_\_\_. \emph{Surveillance Society}: Monitoring Everyday Life.
Buckingham: Open University Press, 2001b.

LYON, D.; ZUREIK, E. (Ed.). \emph{Computers, Surveillance, and Privacy}.
Minneapolis: University of Minnesota Press, 1996.

MARRON, K. New Crime Fighter Rides Along with Cops in Cruisers.
\emph{The Globe and Mail}, Toronto, 23 fev. 2001, T2.

NEWBURN, T.; HAYMAN, S. \emph{Policing, Surveillance, and Social
Control}: CCTV and the Police Monitoring of Suspects. Southampton:
Willan, 2002.

NORRIS, C.; ARMSTRONG, G. \emph{The Maximum Surveillance Society}.
London: Berg, 1999.

PERRI 6 \emph{Divided by Information}: The ``Digital Divide'' and
Implications of the New Meritocracy (com Ben Jupp). London: Demos, 2001.

REGAN, P. \emph{Legislating Privacy}: Technology, Social Values and
Public Policy. Chapel Hill: University of North Carolina Press, 1995.

ROMERO, S. Locating Devices Gain in Popularity but Raise Privacy
Concerns. \emph{The New York Times}, New York, 2001. Disponível em:
\textless{}http://www.nytimes.com/2001/03/04/technology/04LOCA.html\textgreater{}.
Acesso em: 4 mar. 2001.

ROSEN, J. A Cautionary Tale for a New Age of Surveillance (Um conto
preventivo para uma nova era de vigilância). \emph{The New York Times},
New York, 7 out. 2001.

SLEVIN, P. Cameras Caught Super bowl Crowd. \emph{Washington Post},
Washington, 2001. Disponível em:
\textless{}http://www.msnbc.com/news/524802.asp\textgreater{}. Acesso
em: 2 fev. 2001.

STEPANEK, M. Weblining. \emph{BusinessWeek}, New York, 2000. Disponível
em:
\textless{}http://www.businessweek.com:/2000/00\_14/b3675027.html\textgreater{}.
Acesso em: 3 abr. 2000.

TETRAD . Psyte Market Segments. 2001. Disponível em:
\textless{}http://www.tetrad.com/pcensus/com/py951st.html\textgreater{}.

TOMAS, P. Ontario Today Report, July 26. 2000. Disponível em:
\textless{}http://ottawa.cbc.ca/onttoday/archives/july\_26\_00.html\textgreater{}.
Acesso em: 7 maio 2001.

URRY, J. Mobile Sociology. \emph{British Journal of Sociology}, London,
vol. 51, n. 1, 2000, p. 185-203.

VAN DER PLOEG, I. The Illegal Body: Eurodac and the Politics of
Biometric Identification. \emph{Ethics and Information Technology},
Dordrecht, vol. 1, n. 4, 1999, p. 295-302.

VIRILIO, P. \emph{War and Cinema}: The Logistics of Perception. London:
Verso, 1994; \emph{The Vision Machine}. London: The British Film
Institute, 1989.

WALTERS, J. New Immigrant ID Cards are Proposed to Foil Illegals.
\emph{The Toronto Star}, Toronto, 2001. Disponível em:
\textless{}http://www.torontostar.ca/NAS\ldots{}/Article\_PrintFriendly\&c=Article\&cid=99119106461\textgreater{}.
Acesso em: 30 maio 2001.

\chapter{A fantasia do olhar
total}\label{a-fantasia-do-olhar-total}

Silvia Viana

\textbf{I}

Uma constatação atravessa o debate público a respeito das novas
tecnologias de monitoramento e vigilância: seu poder total, seja ele
atual ou potencial. Favoráveis, levando em consideração a necessidade de
segurança, ou a não menos imperiosa funcionalidade da visibilidade,
aplaudem sua ampliação até os recônditos da última viela, tanto quanto
do primeiro pensamento. Contrários sabem dos riscos de um controle que
se adensa e se espraia, seja sobre a intimidade, seja sobre a vida
pública -- ambos sob ameaça de desaparição. Onipresentes são as
evidências de tal poder: câmeras, GPSs, aparelhos biométricos, drones,
dispositivos de rastreamento de dados na internet, chips de
identificação e muito mais câmeras, em casa, no trabalho, no bar, nas
ruas... Trata-se de um olhar pulverizado, modular, flexível, que difere
do modelo panóptico por dispensar o centro de vigia; o que amplia, ainda
mais, a capacidade de captura de todos e de tudo desses todos. À imagem
desse fluxo tentacular correspondem as críticas sobre sua capacidade
incansável de assimilação e apropriação do que quer que se queira
desviante.

Contudo, mais que fruto de um desvelamento crítico do poder, essa imagem
é, do mesmo modo, exibida pelas instâncias às quais se atribui a
funcionalidade da vigilância. Basta andarmos pelas ruas de uma grande
cidade para nos darmos conta da onipresença das câmeras, e não sabemos
se as enxergamos porque se tornaram parte inevitável da paisagem ou
porque, informados pela grandiloquente propaganda estatal, as
pressentimos. Já o novo sistema operacional da Microsoft não deixa
qualquer margem para o questionamento relativo ao verdadeiro
proprietário, da máquina, bem como das operações que nela serão
realizadas, pois simplesmente não é possível acionar o computador pela
primeira vez sem antes registrar-se em um dos servidores da empresa e,
para isso, dar o aceite às famigeradas ``condições de uso'', que tudo
permitem àquele que nos provê. Surpresa para mim, que não tenho celular,
esse procedimento já é segunda natureza para todos os que não podem se
dar ao luxo de deixar a internet longe do bolso -- e, como veremos,
trata-se mesmo de luxo, longe de uma imaginada resistência. Mas se há
uma modalidade de vigilância cuja visibilidade é mais que alardeada, são
os reality shows, afinal, como não cessam de afirmar produtores,
consumidores e pesquisadores, a devassa da vida íntima é sua razão de
ser. E se já não causam aversão as notícias diárias a respeito das
novidades tecnológicas de monitoramento, tampouco têm o efeito político
de choque as denúncias de espionagem estatais. Tanto a inovação quanto a
espionagem são antecipadas pela imagem do olhar total, a fantasia
precede ambas as notícias, tomem elas a forma da mera constatação -- sob
a qual mal se esconde a autocelebração do que assim é -- ou do
escândalo. A novidade é incapaz de alterar o estado geral das coisas, a
não ser na percepção sempre renovada de que foi percorrido um centímetro
a mais no movimento das duas paredes entre as quais espreme-se o humano.
Não obstante as propaladas mudanças de práticas e discursos decorrentes
da proliferação das tecnologias de vigilância, vislumbra-se sempre e
para sempre um mesmo sujeito: acuado.

A realidade se tornou tão absurdamente sufocante que se torna natural a
explicação que parte da proliferação das tecnologias de vigilância e
monitoramento e desemboca em seus efeitos. Como olhar onipresente, elas
são a imagem plasmada, ainda que flexível, da dominação que, no entanto,
as precede e orienta. Tendo em vista a ampliação da reflexão a seu
respeito, cabe indagar a estrutura que, em primeiro lugar, configura o
desenvolvimento técnico do qual resultam e, em segundo lugar, confere
forma à vigilância, direciona o olhar. Não se trata, portanto, de
indagar essas tecnologias no ponto em que falham, e sim de questionar a
percepção que delas não consegue se desviar, assim ampliando, por vezes
à revelia das intenções, sua fantasmagoria. ~

\textbf{II}

Se as tecnologias de vigilância são fonte de ansiedade por parte de seus
críticos, o mesmo não se pode dizer a respeito da técnica da qual
derivam. A percepção de que as inovações científicas microeletrônicas
são neutras e apenas posteriormente instrumentalizadas pelas empresas ou
o Estado está entre as mais correntes, em especial entre ciberativistas.
Uma das principais narrativas nesse sentido é aquela que afirma o
potencial democratizante da internet, cuja forma intrinsecamente
rizomática permite a quebra do monopólio no que tange à produção e
circulação de informações. Dessa perspectiva, a sociedade-rede cria e
recria, permanentemente, linhas de força múltiplas e horizontais que não
obedecem a um sentido preestabelecido e que, por isso mesmo, são
passíveis de gerar contrapoderes dos mais variados. Contudo, a não ser
em casos de delírio ideológico, não se pode negar o diferencial de poder
político e econômico existente entre os núcleos de controle e as,
digamos assim, pessoas físicas. Graças a esse diferencial, as empresas,
até bem mais que os Estados, conseguem, ``capturar'' os fluxos de
informação, e isso de duas maneiras. Em primeiro lugar, a posteriori,
mediante a coleta e monitoramento de dados fornecidos involuntariamente;
mas também de modo mais sofisticado, através da ``manipulação''
publicitária, que comanda a direção dos fluxos antes mesmo que a
informação seja gerada. Nesse cenário, a queda de braço entre o controle
e a participação democrática parece ter destino marcado. E a internet e
derivados aparecem, a cada click, como uma promessa não cumprida. ~~

O cenário muda completamente de figura se, pelo contrário, pensarmos as
tecnologias de informação como realização plena daquilo o que sempre
prometeram: a ampliação simultânea do controle E da participação. E isso
não graças à neutralidade da técnica mas, pelo contrário, porque ela não
está livre de um sentido previamente estabelecido: o da produção
capitalista. É disso o que provavelmente se esquecem aqueles que, ainda
nos dias de hoje, se autocongratulam em suas batalhas diárias contra
moinhos hierárquicos, disciplinares, centralizados e rígidos, típicos
daquele cachorro morto chamado fordismo. Pois, desde seu nascimento, a
empresa flexível comporta, ela mesma, o tal rizoma. Sua morfologia
implica, por um lado, a descentralização do trabalho produtivo,
pulverizado mediante inúmeros arranjos móveis de contratação,
subcontratação ou completa ausência de contrato; por outro, um centro
coordenador desses fluxos. Assim, a chamada ``sociedade da informação'',
cuja assepsia terminológica grita três vivas ao que está dado, é função
primeira da forma-empresa, visto que seu sucesso depende menos do
produto que da conexão. Se essa puder ser ampliada até o espaço mais
impensável, como a favela, por exemplo, ótimo, se puder ser realizada de
forma gratuita, melhor ainda, se a empresa não precisar se
responsabilizar juridicamente pelo trabalho ofertado, maravilha, se, por
fim, resultar de engajamento político, êxtase. ~Assim, a democratização
do acesso, por exemplo, é uma bandeira gloriosa para um sistema que
demanda mobilização total sem que por ela precise pagar uma cocada.

A tensão que se desenrola entre controle e participação é, dessa
perspectiva, falsa. Os dois polos obedecem a um mesmo imperativo: o da
produção -- seja ela voltada para o lucro, seja a que o renegue e que,
por isso mesmo, se desdobra em produtivismo e resulta, do mesmo modo, em
acumulação. Assim funcionam os softwares livres tanto quanto o que se
cria sob a alcunha de ``novas narrativas''. Nesse campo, a aparência de
disputa política mal esconde sua verdade: trata-se de concorrência de
mercado, e não são poucas as vezes que assim é chamada sinceramente a
``tática''. Não foi outro o impulso a partir do qual surgiram algumas
das maiores corporações do ramo -- muito da lenda dos CEOs inovadores
está mais próximo de nós, pobres mortais de esquerda, do que queremos
imaginar. Pois é a partir dessa dupla configuração do capitalismo
contemporâneo que são desenvolvidas as tecnologias informacionais.
Plataformas como o Facebook, o Twiter e o Whatsup são sua síntese, não a
``captura'' de um polo pelo outro. Daí a dificuldade, até do mais
radical e paranoico entre os críticos da vigilância, em cometer o tão
sonhado suicídio virtual. Afinal, como todos sabem, precisamos ``ocupar
espaços'', ``disputar pensamentos'', ``ampliar o debate'', ``informar'',
``conectar'', em suma, participar.

\textbf{III}

Foi também em nome da democratização da produção e circulação das
informações que se criou um dos principais e mais abjetos dispositivos
de vigilância de nosso tempo: os reality shows. De fato, não há como
negar que a paralisia apática do antigo consumidor de televisão foi
substituída por uma postura ``ativa e propositiva''. Quando não se
voluntaria à participação (olha aí o termo de novo) nos programas, o
telespectador pode agir sobre a trama votando ou criando voluntariamente
espaços infinitos de debates virtuais. Também aqui, trabalha-se
gratuitamente, e de bom grado. Diante do show, e mais uma vez, os
críticos denunciam a ``manipulação'' por trás do proclamado ideal -- a
essa altura já deveria parecer, no mínimo, suspeita, a acusação, afinal,
os ``manipulados'' são sempre os outros... De qualquer modo, o \emph{Big
Brother} renega seu próprio nome ao levar a cabo um panóptico invertido,
no qual os observadores são imensamente mais numerosos que seu objeto.
Porém, se a vigilância não parte da torre de comando, qual o sentido do
olhar?

Na mesma década em que surge esse formato, foi realizada a primeira
produção que busca questioná-la. Trata-se do filme \emph{O show de
Thruman}, que amplifica o fantasma do olhar total de modo a apresentar
uma distopia do mundo atual. Thruman leva uma vida comum de americano
classe média, suburbano e bem casado; perfaz seu circuito cotidiano de
casa ao trabalho e desse de volta à casa, sem maiores sobressaltos. A
trama transcorre a partir do que ele ignora: ainda quando bebê, foi
adotado por uma empresa de mídia que, desde então, transmite sua vida 24
horas, sete dias por semana, para todo o mundo. Estranhamente, mesmo
quando, graças a falhas no sistema, Thruman começa a desconfiar da bolha
que habita, e o drama se inicia, o filme permanece quilômetros aquém da
tensão distópica prometida. O tom água com açúcar hollywoodiano
certamente tem muito a ver com isso, mas o problema central está no
ponto em que se imagina residir o terror: a vigilância. Imaginemos, por
um instante, o que aconteceria caso um programa como esse chegasse a ser
produzido. Imaginemo-nos assistindo a tudo o que uma pessoa ordinária
faz em seu cotidiano ordinário: o resultado seria o mais puro tédio,
algo semelhante ao efeito do filme \emph{sleep}, de Andy Warhol. Assim
como em grande parte das pesquisas acadêmicas, o filme falha em supor
que são as câmeras o sustentáculo dos reality shows.

O central, no entanto, é a construção que se dá a ver, a relação social
que precede e direciona o olhar. A realidade que transmitem tais
programas assume, invariavelmente, a forma da seleção. O diretor do
\emph{Big Brother} talvez tenha sido o que melhor definiu o mecanismo,
segundo ele: ``\emph{Big Brother} não é cultura, não é um programa que
propõe debates. É um jogo cruel, em que o público decide quem sai. Ele
dá o poder de o cara que está em casa ir matando pessoas, cortando
cabeças. Não é um jogo de quem ganha. Para o cara de casa, é um jogo de
quem você elimina''. Trata-se de uma seleção negativa, e isso não apenas
por ter no descarte seu motor, mas também porque sua necessidade deriva
de uma cota de eliminações predefinida, e não do que fazem ou deixam de
fazer aqueles cuja vida transcorre sobre o patíbulo. Como não se cansam
de afirmar os apresentadores, em tom lamentoso, ``infelizmente, não há
espaço para todos, alguém tem que ser eliminado''. Daí os programas
serem estruturados em torno de desafios, geralmente estúpidos, sempre
despropositados, por não estabelecerem relação alguma entre esforço e
premiação. Não obstante, a recusa é inimaginável, pois à espreita está,
a cada segundo e em todos os lugares, o paredão.

Ao contrário do que afirmam muitos de seus críticos, reality shows são
realidade, pois, também desse lado da tela, a seleção é o espectro que
ronda e sufoca. Ela já não é apenas o processo ao qual nos submetemos
para a entrada no mercado de trabalho, mas, principalmente o mecanismo
que permanentemente define se aí somos dignos de permanecer. Ainda que
saibamos da ampliação e intensificação ao infinito do trabalho, a imagem
de sua escassez é materializada pela impermanência e descontinuidade
próprias de um mundo convertido em mercado. Mais que isso, o imperativo
da sobrevivência é ritualizado por uma sanha avaliativa onipresente,
multidirecional e feroz.

A ela todas as dimensões do humano estão subsumidas, pois, como sabemos,
em nosso mundo-açougue, não passamos de capital para investimento. A
dificuldade de contabilizar o inquantificável, porém, é incapaz de fazer
recuar a necessidade demente de avaliação, pelo contrário, ela se torna
compulsiva na proporção direta de sua irrealidade. A exposição de uma
vida, em seus afetos, idiossincrasias, pensamentos e ações, em um
programa de TV, por exemplo, é falsa mercadoria; seu valor não pode ser
estabelecido segundo o tempo de trabalho socialmente necessário para sua
produção. Sua mensuração, portanto, já não é abstração, mas pura
alucinação, daí a invenção idiota das ``estalecas'', um dinheiro falso
pago por uma falsa mercadoria, não obstante, forjada em um esforço
descomunal. Idiota, mas não exclusiva, pois o conhecimento, igualmente
incomensurável, se realiza mercadologicamente como pontuação qualis X,
K, Y e outras incontáveis cifras, em nosso show de horror particular.
Assim vemos proliferar técnicas surreais que a tudo devem medir,
pontuar, ranquear, quantificar, sem que nada disso seja capaz de
fornecer parâmetros para a eliminação que a todos espreita.

Minto, há um parâmetro, universal mas negativo, a participação. Pois,
sem ela, não há avaliação possível. É conhecimento tácito entre nós que
aquele que não atualiza seu Lattes está ``fora do jogo'', quem não o
tem, sequer ``entra no jogo''. Sim, a lógica é tautológica: participamos
para sermos avaliados para podermos permanecer participando. Ou, como
bem definiu o patrão de um amigo: ``se você não produzir, você não se
sustenta; e não basta se aplicar, isso tem que aparecer, por isso
precisamos sempre atualizar nosso Lattes, se não você não se sustenta''.
Não é outro o mecanismo das ``curtidas'' em redes sociais: trata-se de
uma quantificação da inteligência, dos relacionamentos, dos sentimentos,
da vida íntima e de tudo mais que lá couber. Tanto quanto em nossa
plataforma particular de acadêmicos, sabemos que sair do Facebook
implica desaparição simbólica, ou melhor, agimos como se fosse esse
nosso destino. Pois, independentemente do nível de conhecimento crítico
que tenhamos a respeito de quaisquer umas de nossas plataformas de
comunicação, na prática, participamos, compulsivamente,
ritualisticamente.

Assim se estabelece a vigilância radial, não por exibicionismo ou
voyeurismo, mas pela necessidade fantasmagórica de não sermos
descartados. Ela não precisa ser inculcada por instâncias superiores,
brota da concorrência horizontal objetivamente estabelecida pelos
rituais de seleção negativa; e é para sempre alimentada pelo espectro do
olhar avaliativo, diante do qual precisamos manter nossos olhos reais
escancarados. Onde o inferno são os outros, o dispositivo somos nós. A
vigilância é, nesse sentido, ao mesmo tempo, vigília e a exposição do eu
não passa de disponibilização se si para o mundo em que reina, acima de
qualquer suspeita, o capitalismo.

\section{Bibliografia}

ADORNO, Theodor W. \emph{Mínima Moralha}. São Paulo: Ática, 1992.

AGAMBEN, Giorgio. \emph{O que é um dispositivo?} Disponível em:
https://periodicos.ufsc.br/index.php/Outra/article/view/12576. Última
visita: 09/05/2015

ARANTES, Paulo. \emph{O novo tempo do mundo.} São Paulo: Boitempo, 2014.

BENJAMIN, Walter. \emph{O capitalismo como religião.} São Paulo:
Boitempo, 2013.

BERNARDO, João. \emph{Democracia totalitária: teoria e prática da
empresa soberana.} São Paulo, Cortez, 2004.

DEJOURS, Cristophe. \emph{A banalização da injustiça social}. Rio de
janeiro: Editora FGV, 2000.

FUCHS, Cristian. ``Como podemos definir vigilância?'' In: \emph{Revista
Matriz,} Ano 5 -- nº 1 jul./dez. 2011, p. 109-136. ~

MARCUSE, Herbert. \emph{A ideologia da Sociedade Industrial: O homem
unidimensional}. Rio de Janeiro: Zahar Editores, 1982.

VIANA, Silvia. \emph{Rituais de sofrimento}. São Paulo: Boitempo, ~2012.

\section{RESUMOS DO LAVITS RIO 2015 CONSULTADOS}

ABDO, Alexandre Hannud. ``Descentralização e criptografia no combate à
vigilância e controle''.

ABIUSO, Federico Luis. ``De la vigilancia jerárquica a la vigilancia
informática: puentes y vinculaciones entre el neoliberalismo penal y las
tecnologías de control''.

ALBUQUERQUE, Luciana Santos Guilhon; PEDRO, Rosa Maria Leite Ribeiro.
``A Visibilidade Por Trás Das Máscaras''.

ANDRADE, Daniel Pereira. ``Vigilância e controle dos afetos no trabalho:
a questão do capital emocional''. ~

ANGELINI, Alessandro. ``The Urban Nervous System: Involuntary Action and
Political Anxiety in a `Smarter' Rio de Janeiro''.

CÓRDOVA, Yasodara; GONZALEZ, Cristiana. ``O poder troca de mãos: como a
Web pode proteger o usuário no contexto da Internet das Coisas''.

CUNHA, Adriana Pessôa da. ``Redes, Coletivos e Tecnologias de
Monitoramento: novas dinâmicas do coletivo e novas formas de controle na
era das redes''.

ENGELKE, Antonio\textbf{. ``}Midiativismo nas Jornadas de Junho:
Narrativa, Sujeito e Verdade''.

EVANGELISTA, Rafael de Almeida; VIEIRA, Miguel Said. ``A máquina de
exploração da privacidade e suas conexões sociais''.

FIGARO Roseli; NONATO, Claudia; MORAES, Laila de Albuquerque.
``Comunicação, vigilância e controle no mundo do trabalho: empresas e
poder econômico como cerceadores da liberdade de expressão e do direito
à informação''.

GONÇALVES, Cristiana de Siqueira; PEDRO, Rosa Maria Leite Ribeiro.
``Visibilidade e gestão de si: cartografando controvérsias acerca das
novas biotecnologias''.

LIMA, Patricia; SANTOS, Emanuella. ``Somos todos terroristas, até
provarmos o contrário. As implicações da vigilância em massa justificada
como medida de segurança na rede''.

PEDRO, Rosa Maria Leite Ribeiro; RODRIGUES Ana Paula da Cunha. ``Google
e Linked in: controvérsias acerca da visibilidade e vigilância na
contemporaneidade''.

ROCHA, Otávio Gomes. ``O ardil do mapeamento participativo e a renovação
das estratégias de controle territorial'.

SAETNAN, Ann Rudinow. ``The Haystack Fallacy -- or Why Big Data Provides
Little Security''.

SAMPAIO, Alice C. ``Privacidade em rede: a participação da Google e do
Facebook na transformação de um conceito''.

TELES, Edson. ``Militarização da política e democracia de segurança''

WIVES, Willian Washington. ``Vigilância e Monitoramento: efeitos micro e
macro'.
